%!
%! Copyright (C) 2015 - Present Andrea Dal Corso 
%! This file is distributed under the terms of the
%! GNU General Public License. See the file `License'
%! in the root directory of the present distribution,
%! or http://www.gnu.org/copyleft/gpl.txt .
%!
\documentclass[12pt,a4paper]{article}
\def\version{1.1.0}
\def\qe{{\sc Quantum ESPRESSO}}
\def\tpw{{\sc THERMO\_PW}}

\usepackage{html}
\usepackage{color}

\usepackage{graphicx}

\definecolor{web-blue}{rgb}{0,0.5,1.0}
\definecolor{coral}{rgb}{1.0,0.5,0.3}
\definecolor{red}{rgb}{1.0,0,0.0}
\definecolor{green}{rgb}{0.,1.0,0.0}


\textwidth = 17cm
\textheight = 24cm
\topmargin =-1 cm
\oddsidemargin = 0 cm

\def\pwx{\texttt{pw.x}}
\def\phx{\texttt{ph.x}}
\def\configure{\texttt{configure}}
\def\PWscf{\texttt{PWscf}}
\def\PHonon{\texttt{PHonon}}
\def\thermo{\texttt{thermo\_pw}}
\def\make{\texttt{make}}

\begin{document} 
\author{}
\date{}

%\def\SissaImage{./sissa_on_white.png}

\title{
%  \includegraphics[width=6cm]{\SissaImage}\\
  \vskip 1cm
  {\color{red} \Huge Thermo\_pw: Thermodynamics of crystals} \\
  \Large (version \version)
}

\maketitle

\tableofcontents

\newpage

\section{{\color{coral}Introduction}}
These notes describe the thermodynamics of crystals implemented in \tpw. 

\section{\color{coral}People}
These notes have been written by Andrea Dal Corso (SISSA - Trieste). \\
Disclaimer: I am not an expert of crystal thermodynamics. 
These notes reflect what I have understood so far and implemented in 
\texttt{thermo\_pw}. If you think that some formula is wrong, 
that I misunderstood something, or that something can be calculated more 
simply, please let me know, I would like to learn more. 
You can contact me directly: \texttt{dalcorso@sissa.it} or send an
e-mail to the \texttt{thermo\_pw-forum} mailing list.

\section{\color{coral}Overview}
The properties of a solid addressed by \tpw\ are described by the thermodynamic
theory of the elastic solid. In this theory a solid is characterized by an
equilibrium configuration that depends on the temperature $T$ and on an
external stress $\sigma$ (assumed constant) that describes the forces 
that act on the surface. The state
of the solid is characterized by a reference configuration 
(for instance the equilibrium configuration at $T=0$ and $\sigma=0$) and 
a strain $\epsilon$ which describes the deformation with respect to the 
reference configuration. 
Within the quasi-harmonic approximation the solid has an internal
energy given by  
\begin{equation}
U=U_0 + U_{el} + U_{ph}
\end{equation}
where $U_0$ is the ground state energy of the electrons in the field 
of the nuclei fixed at the equilibrium positions at the given strain and
temperature, $U_{el}$ is the additional electron energy due to the 
finite electron temperature and $U_{ph}$ is the internal energy of the 
phonons that accounts for the vibrational energy of the nuclei. 
Given the phonon frequencies $\omega_{{\bf q},\nu}$ of the solid 
(at wavevector ${\bf q}$) we can write:
\begin{equation}
U_{ph} = \sum_{{\bf q},\nu} {\hbar \omega_{{\bf q},\nu} \over 2}
+ \sum_{{\bf q},\nu} {\hbar \omega_{{\bf q},\nu} \over 
e^{\beta \hbar \omega_{{\bf q},\nu}}-1}.
\label{uvib}
\end{equation}
where $\beta=1/ k_B T$ and $k_B$ is the Boltzmann constant and $\hbar$ is the
Planck constant.
Presently $U_0$ is calculated within density functional theory (DFT) and
$U_{ph}$ within density functional perturbation theory (DFPT). 
For metals the electronic contribution to the heat capacity is not 
completely negligible. \tpw\ allows its estimate for a gas of 
independent electrons with the density of states given by the DFT bands,
but it neglects $U_{el}$ in all other calculations.
Similarly the Helmholtz free energy is $F=U_0+F_{ph}$ with:
\begin{equation}
F_{ph} = k_B T \sum_{{\bf q},\nu} \ln \left[ 2 \sinh \left({\beta \hbar 
\omega_{{\bf q},\nu} \over 2}\right) \right],
\nonumber
\end{equation}
where we have neglected the finite electron temperature. \\
Having the internal energy and the Helmholtz free energy, the entropy can 
be calculated as $S=(U-F)/T$.

\tpw\ calculates the energy, the Helmholtz free energy, and the
entropy of a unit cell of the solid for a given choice of the primitive 
Bravais lattice vectors ${\bf a}_1$, ${\bf a}_2$, and ${\bf a}_3$. These
vectors depend on a few parameters $\zeta_k$ that determine the size 
of the unit cell ($k$ is $1$ for cubic lattices and has a maximum
value $k=6$ for the triclinic lattice).
The unit cell contains $N_{at}$ atoms
whose positions are indicated by ${\bf d}_s$ with $1\le s\le N_{at}$.
These positions are assumed to be optimized so that the forces acting
on atoms vanish.
The reference configuration of the solid is
characterized by a set of parameters $\zeta_{k,0}$ and a set of parameters 
$\zeta_k$ corresponds to a solid in a strained state.
Therefore from a theoretical point of view it is easy to calculate the 
internal energy, the Helmholtz free energy, and the entropy as a 
function of strain and temperature.

Experimentally it is simpler to control the stress imposed on the solid. 
Moreover in some cases the thermodynamic quantities are measured without 
heat exchange, that is at constant entropy instead of at constant temperature. 
In the thermodynamic theory of the elastic solid it is natural to consider two
functions. The internal energy as a function of strain and entropy  
to describe adiabatic properties and the Helmholtz free energy as a function
of strain and temperature to describe isothermal properties.
Therefore we can calculate directly isothermal quantities (measured at constant
temperature), but we need thermodynamics equations to calculate
the adiabatic quantities (measured at constant entropy).

First derivatives of the Helmholtz free energy with respect to strain
give the stress acting of the solid and this gives one or more equations
of state that relate strain, stress, and temperature. 
Searching the strain for which the stress is zero we determine the 
equilibrium geometry of the solid at each temperature as the minimum of 
the Helmholtz free energy. From these information we can deduce the
thermal expansion of the solid. The first derivative of the entropy 
with respect to temperature at fixed strain multiplied by the temperature
is the heat capacity at constant strain. Second derivatives of the 
Helmholtz free energy with 
respect to strain calculated at the equilibrium geometry give the 
isothermal elastic constants at the given temperature and at zero pressure. 
From the elastic constants we can calculate the isothermal bulk modulus 
of the solid at any temperature. 
Thermal expansions and elastic constants allows the calculation of the thermal
stresses and using these quantities we can obtain the heat capacity at
constant stress, the adiabatic elastic constants and bulk moduli 
from the corresponding isothermal ones. Finally from the constant strain 
heat capacity and the thermal stresses we can calculate the anisotropic 
average Gr\"uneisen parameters.

The strain for which the stress is a given pressure (or stress) at a given
temperature minimizes the Gibbs free energy at that temperature and
pressure (or stress) and from the minimum of this function we can determine 
the equilibrium geometry of the solid as a function of temperature and 
pressure (or stress). 
From these information we can deduce the thermal expansion of the 
solid at fixed pressure (or stress). Second derivatives of the Gibbs free
energy with respect to strain give the isothermal elastic constants at the 
given temperature and pressure (or stress). From the elastic constants 
we can calculate the isothermal bulk modulus of the solid at the
given temperature and pressure (or stress).
%Thermal expansions and elastic constants allow the calculation of the thermal
%stresses at the given pressure (or stress) and using these quantities we 
%can obtain the adiabatic heat capacity, elastic constants, and bulk moduli at
%the given pressure (or stress) from the corresponding isothermal ones. 
%Finally from the constant strain heat capacity and the thermal stresses 
%we can calculate the anisotropic average Gr\"uneisen parameters at 
%the given temperature and pressure (or stress).

In this guide we define all the quantities described above and calculated 
by \tpw\ and give the formulas needed to relate the crystallographic
parameters $\zeta_k$ to the thermodynamic functions.

\newpage

\section{\color{coral}Crystal lattices}
We start with a short review of the fourteen Bravais lattices and give
our definition of the primitive lattice vectors. For each lattice we 
specify the number of parameters necessary to define the unit cell.
The volume of one unit cell is indicated with $\Omega$ and the total 
volume of the solid is $V=N\Omega$ where $N$ is the number of unit cells
of the solid. 

\subsection{\color{web-blue}Cubic system}
The cubic system comprises three Bravais lattices. The cubic lattice
is:
\begin{eqnarray}
{\bf a}_1 &=& a (1, 0, 0), \nonumber \\
{\bf a}_2 &=& a (0, 1, 0), \nonumber \\
{\bf a}_3 &=& a (0, 0, 1). \nonumber
\nonumber
\end{eqnarray}
The volume of the unit cell is $\Omega = a^3$. \\
The face-centered cubic lattice is:
\begin{eqnarray}
{\bf a}_1 &=& {a \over 2} (-1, 0, 1), \nonumber \\
{\bf a}_2 &=& {a \over 2} (0, 1, 1), \nonumber \\
{\bf a}_3 &=& {a \over 2} (-1, 1, 0). \nonumber
\nonumber
\end{eqnarray}
The volume of the unit cell is $\Omega = a^3/4$. \\
The body-centered cubic lattice is:
\begin{eqnarray}
{\bf a}_1 &=&{a \over 2} (1, 1, 1), \nonumber \\
{\bf a}_2 &=&{a \over 2} (-1, 1, 1), \nonumber \\
{\bf a}_3 &=&{a \over 2} (-1, -1, 1). \nonumber
\nonumber
\end{eqnarray}
The volume of the unit cell is $\Omega = a^3/2$. \\
All three lattices are characterized by one parameter $\zeta_1=a$.

\subsection{\color{web-blue}Tetragonal system}
The tetragonal system comprises two Bravais lattices: the tetragonal
and the centered tetragonal lattices.\\
The tetragonal lattice is:
\begin{eqnarray}
{\bf a}_1 &=& a (1, 0, 0), \nonumber \\
{\bf a}_2 &=& a (0, 1, 0), \nonumber \\
{\bf a}_3 &=& a (0, 0, {c\over a}). \nonumber
\nonumber
\end{eqnarray}
The volume of the unit cell is $\Omega = a^2c$. \\
The centered tetragonal lattice is:
\begin{eqnarray}
{\bf a}_1 &=& {a \over 2} (1, -1, {c\over a}), \nonumber \\
{\bf a}_2 &=& {a \over 2} (1, 1, {c\over a}), \nonumber \\
{\bf a}_3 &=& {a \over 2} (-1, -1, {c\over a}). \nonumber
\nonumber
\end{eqnarray}
The volume of the unit cell is $\Omega = a^2c/2$. \\
Both lattices are characterized by two parameters $\zeta_1=a$ and 
$\zeta_2=c/a$.

\subsection{\color{web-blue}Orthorhombic system}
The orthorhombic system comprises four Bravais lattices.
The orthorhombic lattice:
\begin{eqnarray}
{\bf a}_1 &=& a (1, 0, 0), \nonumber \\
{\bf a}_2 &=& a (0, {b\over a}, 0), \nonumber \\
{\bf a}_3 &=& a (0, 0, {c\over a}). \nonumber
\nonumber
\end{eqnarray}
The volume of the unit cell is $\Omega = abc$. \\
The base-centered orthorhombic lattice. This lattice can be of three
types A, B, or C, but only two types are presently used. 
The C-type is:
\begin{eqnarray}
{\bf a}_1 &=& {a \over 2} (1, {b \over a}, 0), \nonumber \\
{\bf a}_2 &=& {a \over 2} (-1, {b \over a}, 0), \nonumber \\
{\bf a}_3 &=& a  (0, 0, {c \over a}), \nonumber
\nonumber
\end{eqnarray}
while the A-type is:
\begin{eqnarray}
{\bf a}_1 &=& a  (1, 0, 0), \nonumber \\
{\bf a}_2 &=& {a \over 2} (0, {b \over a}, -{c \over a}), \nonumber \\
{\bf a}_3 &=& {a \over 2} (0, {b \over a}, {c \over a}). \nonumber 
\nonumber
\end{eqnarray}
The volume of the unit cell is $\Omega = abc/2$ for both lattices. \\
The face-centered orthorhombic lattice:
\begin{eqnarray}
{\bf a}_1 &=& {a \over 2} (1, 0, {c \over a}), \nonumber \\
{\bf a}_2 &=& {a \over 2} (1, {b \over a}, 0), \nonumber \\
{\bf a}_3 &=& {a \over 2} (0, {b \over a}, {c \over a}). \nonumber
\nonumber
\end{eqnarray}
The volume of the unit cell is $\Omega = abc/4$. \\
The body-centered orthorhombic lattice:
\begin{eqnarray}
{\bf a}_1 &=& {a \over 2} (1, {b \over a}, {c \over a}), \nonumber \\
{\bf a}_2 &=& {a \over 2} (-1, {b \over a}, {c \over a}), \nonumber \\
{\bf a}_3 &=& {a \over 2} (-1, -{b \over a}, {c \over a}). \nonumber
\nonumber
\end{eqnarray}
The volume of the unit cell is $\Omega = abc/2$. \\
All four lattices are characterized by three parameters, 
$\zeta_1=a$, $\zeta_2=b/a$ and $\zeta_3=c/a$.

\subsection{\color{web-blue}Monoclinic system}
The monoclinic system comprises two Bravais lattices:
The monoclinic and the base-centered monoclinic lattices.
Each one of these two lattices can be $c$-unique or $b$-unique depending on
which axis is perpendicular to the other two. 
The $c$-unique monoclinic lattice is:
\begin{eqnarray}
{\bf a}_1 &=& a (1, 0, 0), \nonumber \\
{\bf a}_2 &=& a ({b \over a} \cos{\gamma}, {b \over a}\sin{\gamma}, 0), \nonumber \\
{\bf a}_3 &=& a (0, 0, {c \over a}). 
\nonumber
\end{eqnarray}
The volume of the unit cell is $\Omega = abc \sin\gamma$. \\
The $b$-unique monoclinic lattice is:
\begin{eqnarray}
{\bf a}_1 &=& a (1, 0, 0), \nonumber \\
{\bf a}_2 &=& a (0, {b \over a}, 0), \nonumber \\
{\bf a}_3 &=& a ({c \over a} \cos{\beta}, 0, {c \over a}\sin{\beta}), \nonumber 
\end{eqnarray}
The volume of the unit cell is $\Omega = abc \sin\beta$. \\
The $c$-unique base-centered monoclinic lattice is:
\begin{eqnarray}
{\bf a}_1 &=& a ({1\over 2}, 0, -{c \over 2 a}), \nonumber \\
{\bf a}_2 &=& a ({b \over a} \cos{\gamma}, {b \over a}\sin{\gamma}, 0), \nonumber \\
{\bf a}_3 &=& a ({1\over 2}, 0, {c \over 2 a}). 
\nonumber
\end{eqnarray}
The volume of the unit cell is $\Omega = abc \sin\gamma/2$. \\
The $b$ unique base-centered monoclinic lattice is:
\begin{eqnarray}
{\bf a}_1 &=& a ({1\over 2}, -{b \over 2 a}, 0), \nonumber \\
{\bf a}_2 &=& a ({1\over 2}, {b \over 2 a}, 0), \nonumber \\
{\bf a}_3 &=& a ({c \over a} \cos{\beta}, 0, {c \over a}\sin{\beta}), \nonumber 
\end{eqnarray}
The volume of the unit cell is $\Omega = abc \sin\beta/2$. \\
All lattices are characterized by four parameters $\zeta_1=a$, $\zeta_2=b/a$,
$\zeta_3=c/a$ and $\zeta_4=\cos\gamma$
($c$-unique) or $\cos\beta$ ($b$-unique).

\subsection{\color{web-blue}Triclinic system}
The triclinic system comprises only the triclinic Bravais lattice:
\begin{eqnarray}
{\bf a}_1 &=& a (1, 0, 0), \nonumber \\
{\bf a}_2 &=& a ({b \over a}\cos\gamma, {b \over a}\sin\gamma, 0), \nonumber \\
{\bf a}_3 &=& a ({c \over a} \cos{\beta}, {c \over a}(\cos\alpha-\cos\beta\cos\gamma)/\sin\gamma,\nonumber \\
&\phantom{=}& {c \over a}(1+2\cos\alpha\cos\beta\cos\gamma-\cos^2\alpha-\cos^2\beta-\cos^2\gamma), \nonumber 
\end{eqnarray}
The volume of the unit cell is $\Omega=abc \sin\gamma \left( 
1+2\cos\alpha\cos\beta\cos\gamma-\cos^2\alpha-\cos^2\beta-\cos^2\gamma\right)
$.
The lattice is characterized by six parameters $\zeta_1=a$, 
$\zeta_2=b$, $\zeta_3=c$, $\zeta_4=\cos\alpha$,
$\zeta_5=\cos\beta$, $\zeta_6=\cos\gamma$.

\subsection{\color{web-blue}Hexagonal system}
The hexagonal system comprises only the hexagonal Bravais lattice:
\begin{eqnarray}
{\bf a}_1 &=& a (1, 0, 0), \nonumber \\
{\bf a}_2 &=& a (-{1 \over 2}, {\sqrt{3} \over 2}, 0), \nonumber \\
{\bf a}_3 &=& a (0, 0, {c\over a}), \nonumber
\nonumber
\end{eqnarray}
The volume of the unit cell is $\sqrt{3} a^2 c /2$. \\
The lattice is characterized by two parameters $\zeta_1=a$ 
and $\zeta_2=c/a$.

\subsection{\color{web-blue}Trigonal system}
The trigonal system comprises two Bravais lattices: the hexagonal Bravais 
lattice described above or the rhombohedral Bravais lattice:
\begin{eqnarray}
{\bf a}_1 &=& a ({\sqrt{3}\over 2}\sin{\theta}, -{1\over 2} \sin{\theta},
          \cos{\theta}), 
\nonumber \\
{\bf a}_2 &=& a (0, \sin{\theta}, \cos{\theta}), 
\nonumber \\
{\bf a}_3 &=& a (-{\sqrt{3}\over 2} \sin{\theta}, -{1\over 2} \sin{\theta},
         \cos{\theta}), 
\nonumber \\
\nonumber
\end{eqnarray}
where $\sin{\theta}=\sqrt{2\over 3}\sqrt{1-\cos{\alpha}}$
and $\cos{\theta}=\sqrt{1\over 3}\sqrt{1 + 2 \cos{\alpha}}$. \\
The volume of the unit cell is $\Omega={3 \sqrt{3} \over 2} a^3 \sin^2\theta
\cos\theta$. \\
The lattice is characterized by two parameters $\zeta_1=a$ and 
$\zeta_2=\cos\alpha$.

\newpage

\section{\color{coral}The strain tensor}
The strain is defined by a $3\times 3$ tensor
$\epsilon$. Applying a strain $\epsilon$ to a solid, 
the three primitive vectors ${\bf a}_1$, ${\bf a}_2$, and ${\bf a}_3$
becomes 
${\bf a}'_1$, ${\bf a}'_2$, and ${\bf a}'_3$, where:
\begin{equation}
{\bf a}'_{j,i} = {\bf a}_{j,i} +  \sum_{k=1}^3 \epsilon_{j,k} {\bf a}_{k,i}.
\end{equation}
Here $i=1,2,3$ is the index of the primitive vector, while $k$ and $j$ run
on the three Cartesian coordinates. We consider only symmetric tensors
$\epsilon_{j,k}=\epsilon_{k,j}$ removing any global rotation. 
In order to make the connection with the classical elasticity theory, we
call ${\bf X}$ the coordinates of an unstrained solid and
${\bf x}({\bf X})$ the position of each point ${\bf X}$ in
the strained solid. Calling ${\bf u}= {\bf x} -{\bf X}$ the displacements
of each point, the standard theory of elasticity defines the $3\times3$ matrix
$\alpha_{i,j}=\delta_{i,j}+u_{i,j}$ where $u_{i,j}= {\partial u_i \over 
\partial X_j}$ are the displacement gradients and 
${\bf x}_i = \alpha_{i,j} {\bf X}_j$. With our definition $\epsilon_{j,k}$
corresponds to the symmetrized tensor $\epsilon_{j,k}={1\over 2} ( u_{j,k} +
u_{k,j})$. Notice that $\epsilon_{j,k}$ are not the Lagrangian strains
$\eta_{j,k}={1\over 2} ( u_{j,k} + u_{k,j} + \sum_l u_{l,j}u_{l,k})$, but 
differ for terms quadratic in $u_{i,j}$. 

In general, a strain changes both the size 
and the shape of the unit cell possibly changing the Bravais lattice. \\
In this section we consider some strains of interest for our applications: 
\begin{equation}
\epsilon_A=\left( \begin{array}{ccc}
\epsilon_{1,1}, & 0, & 0 
\\
0,& \epsilon_{1,1}, & 0 
\\
0,& 0, & \epsilon_{1,1}
\end{array}
\right),
\end{equation}
is an isotropic strain that modifies the $x$, $y$, and $z$ directions in the
same way,
\begin{equation}
\epsilon_B=\left( \begin{array}{ccc}
\epsilon_{1,1},& 0, & 0 
\\
0, & \epsilon_{1,1}, &  0 
\\
0, & 0, &  0
\end{array}
\right),
\epsilon_{B'}=\left( \begin{array}{ccc}
\epsilon_{1,1},& 0, & 0 
\\
0, & 0, &  0
\\
0, & 0, & \epsilon_{1,1}
\end{array}
\right),
\epsilon_{B''}=\left( \begin{array}{ccc}
0, & 0, & 0 
\\
0, & \epsilon_{2,2}, &  0
\\
0, & 0, & \epsilon_{2,2}
\end{array}
\right),
\end{equation}
are three strains that modify the $xy$, $xz$, and $yz$ plane 
isotropically, without changing the other direction.
The following three strains modify the length in one direction, 
$x$, $y$, or $z$, so they are compressions or dilatations in these directions:
\begin{equation}
\epsilon_C=\left( \begin{array}{ccc}
\epsilon_{1,1}, & 0, & 0 
\\
0, & 0, & 0 
\\
0, & 0, & 0
\end{array}
\right),
\end{equation}

\begin{equation}
\epsilon_D=\left( \begin{array}{ccc}
0, & 0, & 0 
\\
0, &  \epsilon_{2,2}, & 0 
\\
0, & 0, & 0
\end{array}
\right),
\end{equation}

\begin{equation}
\epsilon_E=\left( \begin{array}{ccc}
0, & 0, & 0 
\\
0, & 0, & 0 
\\
0, & 0, & \epsilon_{3,3}
\end{array}
\right).
\end{equation}
Then we consider shear strains first in all three directions:
\begin{equation}
\epsilon_F=\left( \begin{array}{ccc}
0, & \epsilon_{1,2}, & \epsilon_{1,2}
\\
\epsilon_{1,2}, & 0, & \epsilon_{1,2} 
\\
\epsilon_{1,2}, & \epsilon_{1,2}, & 0
\end{array}
\right),
\end{equation}
with the possibility to apply a different sign in one direction
\begin{equation}
\epsilon_{F'}=\left( \begin{array}{ccc}
0, & -\epsilon_{1,2}, & \epsilon_{1,2}
\\
-\epsilon_{1,2}, & 0, & \epsilon_{1,2} 
\\
\epsilon_{1,2}, & \epsilon_{1,2}, & 0
\end{array}
\right),
\epsilon_{F''}=\left( \begin{array}{ccc}
0, & \epsilon_{1,2}, & -\epsilon_{1,2}
\\
\epsilon_{1,2}, & 0, & \epsilon_{1,2} 
\\
-\epsilon_{1,2}, & \epsilon_{1,2}, & 0
\end{array}
\right),
\epsilon_{F'''}=\left( \begin{array}{ccc}
0, & \epsilon_{1,2}, & \epsilon_{1,2}
\\
\epsilon_{1,2}, & 0, & -\epsilon_{1,2} 
\\
\epsilon_{1,2}, & -\epsilon_{1,2}, & 0
\end{array}
\right),
\end{equation}
then only in the $xy$ plane:
\begin{equation}
\epsilon_G=\left( \begin{array}{ccc}
0, & \epsilon_{1,2}, & 0
\\
\epsilon_{1,2}, & 0, & 0
\\
0, & 0, & 0
\end{array}
\right),
\end{equation}
in the $xz$ plane:
\begin{equation}
\epsilon_H=\left( \begin{array}{ccc}
0, & 0, & \epsilon_{1,3}
\\
0, & 0, & 0
\\
\epsilon_{1,3}, & 0, & 0
\end{array}
\right),
\end{equation}
and finally in the $yz$ plane:
\begin{equation}
\epsilon_I=\left( \begin{array}{ccc}
0, & 0, & 0
\\
0, & 0, & \epsilon_{2,3} 
\\
0, & \epsilon_{2,3}, & 0
\end{array}
\right).
\end{equation}
A generic strain in the $xy$ plane is
\begin{equation}
\epsilon_L=\left( \begin{array}{ccc}
\epsilon_{1,1}, & \epsilon_{1,2}, & 0 
\\
\epsilon_{2,1}, & \epsilon_{2,2}, & 0 
\\
0, & 0, & 0
\end{array}
\right).
\end{equation}
For some lattices it is also of interest to apply
\begin{equation}
\epsilon_M=\left( \begin{array}{ccc}
\epsilon_{1,1}, & 0, & 0 
\\
0, & \epsilon_{1,1}, & 0 
\\
0, & 0, & \epsilon_{3,3}
\end{array}
\right),
\end{equation}
a strain that modifies isotropically the plane $xy$ and also the $z$ direction.
\begin{equation}
\epsilon_N=\left( \begin{array}{ccc}
\epsilon_{1,1}, & 0, & 0 
\\
0, & -\epsilon_{1,1}, & 0 
\\
0, & 0, & 0
\end{array}
\right),
\end{equation}
is a shear strain that elongates (or compresses) the $x$ direction and
compresses (or elongates) of the same amount the $y$ direction.
We can define similar strains for the $x$ and $z$ and for 
the $y$ and $z$ directions.
\begin{equation}
\epsilon_O=\left( \begin{array}{ccc}
\epsilon_{1,1}, & 0, & 0 
\\
0, & 0, & 0 
\\
0, & 0, & -\epsilon_{1,1}
\end{array}
\right),
\end{equation}

\begin{equation}
\epsilon_P=\left( \begin{array}{ccc}
0, & 0, & 0 
\\
0, & \epsilon_{2,2}, & 0 
\\
0, & 0, & -\epsilon_{2,2}
\end{array}
\right).
\end{equation}
The strain is a symmetric tensors. Introducing the Voigt notation 
$\epsilon_1=\epsilon_{1,1}$,
$\epsilon_2=\epsilon_{2,2}$, $\epsilon_3=\epsilon_{3,3}$, but
$\epsilon_4=2 \epsilon_{2,3}$, $\epsilon_5=2 \epsilon_{1,3}$, and
$\epsilon_6=2 \epsilon_{1,2}$, a generic strain is:
\begin{equation}
\epsilon=\left( \begin{array}{ccc}
\epsilon_1, & \epsilon_6/2, & \epsilon_5/2 
\\
\epsilon_6/2, & \epsilon_2, & \epsilon_4/2  
\\
\epsilon_5/2, & \epsilon_4/2, & \epsilon_{3}
\end{array}
\right).
\end{equation}

For each system (with the exception of the monoclinic and triclinic) 
we discuss how the Bravais lattice changes for a given strain limiting
to the strains of interest for the following.

\subsection{\color{web-blue}Cubic system}
The cell remains cubic only applying an isotropic strain $\epsilon_A$ that 
changes the cubic lattice constant $a_c$ to $a=(1+\epsilon_{1,1})a_c$.

Applying $\epsilon_B$, the simple (body-centered) cubic
unit cell becomes (centered) tetragonal with:
\begin{eqnarray}
a&=&a_c (1+\epsilon_{1,1}), \\
c&=&a_c.
\end{eqnarray}
The face-centered cubic lattice becomes a centered tetragonal lattice with
\begin{eqnarray}
a&=&a_c {\sqrt{2} \over 2} (1+\epsilon_{1,1}), \\
c&=&a_c.
\end{eqnarray}
However the Cartesian reference system of the tetragonal cell
is rotated with respect to that of the face-centered cubic cell. 
A point of Cartesian coordinates ${\bf r}=(x_1,x_2,x_3)$ 
in the reference system of the face-centered cubic cell 
has coordinates 
${\bf r}=(X_1,X_2,X_3)$ in the reference system of the centered tetragonal cell
where $X_i=\sum_j R^A_{i,j} x_j$ and the rotation matrix $R_{i,j}$ is 
\begin{equation}
R^A=\left( \begin{array}{ccc}
1/\sqrt{2}, & -1/\sqrt{2}, & 0 
\\
1/\sqrt{2}, & 1/\sqrt{2}, & 0 
\\
0, & 0, & 1
\end{array}
\right).
\end{equation}

Similarly, by applying $\epsilon_E$ the simple (body-centered) cubic
cell becomes a (centered) tetragonal cell with
\begin{eqnarray}
a&=&a_c,   \\
c&=&a_c (1+\epsilon_{3,3}).
\end{eqnarray}
The face-centered cubic lattice becomes a centered tetragonal lattice with
\begin{eqnarray}
a&=&a_c {\sqrt{2}\over 2},   \\
c&=&a_c (1+\epsilon_{3,3}).
\end{eqnarray}
The reference system of the centered tetragonal
cell is rotated with respect to the reference system of the face-centered
cubic cell and the change of coordinates can be done with the same rotation 
matrix written above.

Among the shear strain $\epsilon_F$
transforms the simple cubic unit cell into a rhombohedral cell with
parameters:
\begin{eqnarray}
a_r&=& a_c \sqrt{1+2\epsilon_{1,2}^2}, \\
\cos\alpha &=& {\epsilon_{1,2} (\epsilon_{1,2} + 2) \over 
1 + 2 \epsilon_{1,2}^2 }.
\end{eqnarray}
The reference system of the rhombohedral cell is rotated with respect to
the reference system of the cubic cell. The change of coordinates can
be written as above with the rotation matrix:
\begin{equation}
R^B=\left( \begin{array}{ccc}
1/\sqrt{2}, & 0, & -1/\sqrt{2}
\\
-1/\sqrt{6}, & 2/\sqrt{6}, & -1/\sqrt{6}
\\
1/\sqrt{3}, & 1/\sqrt{3}, & 1/\sqrt{3}
\end{array}
\right).
\end{equation}

The face-centered cubic cell is transformed by $\epsilon_{F'''}$
in a rhombohedral unit cell with parameters:
\begin{eqnarray}
a_r&=& {a_c \over 2} \sqrt{6 \epsilon_{1,2}^2-4 \epsilon_{1,2} +2}, \\
\cos\alpha &=& {5\epsilon_{1,2}^2- 6\epsilon_{1,2} + 1 \over 
6\epsilon_{1,2}^2 - 4\epsilon_{1,2}+2 },
\end{eqnarray}
The reference system of the rhombohedral cell is rotated with respect to
the reference system of the face-centered cubic cell. 
The change of coordinates can be written as above with the rotation matrix:
\begin{equation}
R^C=\left( \begin{array}{ccc}
1/\sqrt{2}, & 1/\sqrt{2}, & 0 \\
-1/\sqrt{6}, & 1/\sqrt{6}, & -2/\sqrt{6}
\\
-1/\sqrt{3}, & 1/\sqrt{3}, & 1/\sqrt{3}
\end{array}
\right).
\end{equation}

The body-centered cubic cell is transformed by 
$\epsilon_{F'''}$ in a rhombohedral unit cell
with parameters:
\begin{eqnarray}
a_r&=& {a_c \over 2} \sqrt{ 4\epsilon_{1,2}^2 + 4\epsilon_{1,2} + 3 }, \\
\cos\alpha &=& -{4\epsilon_{1,2}+1 \over 4\epsilon_{1,2}^2+4\epsilon_{1,2}+3}.
\end{eqnarray}
The reference system of the rhombohedral cell is rotated with respect to
the reference system of the body-centered cubic cell. 
The change of coordinates can be written as above with the rotation matrix 
$R^C$.

\subsection{\color{web-blue}Tetragonal system}
Applying $\epsilon_A$, or $\epsilon_B$, or $\epsilon_E$ to a 
(centered) tetragonal cell
does not change its shape. $\epsilon_A$ changes $a_t$ into 
$a_t(1+\epsilon_{1,1})$ and $c_t$ into $c_t(1+\epsilon_{1,1})$ so that the
ration $c_t/a_t$ remains constant. 
$\epsilon_B$ changes $a_t$ into $a_t(1+\epsilon_{1,1})$
while $c_t$ does not change and $\epsilon_E$ changes  
$c_t$ into $c_t(1+\epsilon_{3,3})$ while $a_t$ does not change.

Applying $\epsilon_{B'}$ to a simple tetragonal cell transforms it into an  
orthorhombic cell with
\begin{eqnarray}
a&=& a_t (1+\epsilon_{1,1}), \\
b&=& a_t, \\
c&=& c_t (1+\epsilon_{1,1}).
\end{eqnarray}
Applying $\epsilon_{B'}$ to a centered tetragonal cell transforms it
into a body-centered orthorhombic cell with the same
parameters. 

Similarly, applying $\epsilon_C$ to a simple tetragonal cell transforms 
it into an  orthorhombic cell with
\begin{eqnarray}
a&=&a_t (1+\epsilon_{1,1}), \\
b&=&a_t, \\
c&=&c_t,
\end{eqnarray}
and applying $\epsilon_C$ to a centered tetragonal cell transforms it
into a body-centered orthorhombic cell with the same
parameters.\\

$\epsilon_G$ applied to the simple tetragonal cell transforms it into a 
base-centered orthorhombic cell, with parameters:
\begin{eqnarray}
a&=&\sqrt{2} a_t (1-\epsilon_{1,2}), \\
b&=&\sqrt{2} a_t (1+\epsilon_{1,2}) , \\
c&=&c_t.
\end{eqnarray}
The reference system of the orthorhombic cell is rotated with respect to
the reference system of the tetragonal cell. 
The change of coordinates can be written as above with the rotation matrix 
$R^A$.
$\epsilon_G$ applied to the centered tetragonal cell transforms it into
a face-centered orthorhombic cell with the same parameters written above
and the same change of reference system.

$\epsilon_H$ applied to the simple tetragonal cell transforms it into a 
simple monoclinic (b-unique) cell with:
\begin{eqnarray}
a&=&a_t \sqrt{1+\epsilon_{1,3}^2}, \\
b&=&a_t, \\
c&=&c_t \sqrt{1+\epsilon_{1,3}^2}, \\
\cos\beta &=& {2 \epsilon_{1,3} \over 1 + \epsilon_{1,3}^2}.
\end{eqnarray}
The reference system of the simple monoclinic cell is rotated with respect to
the reference system of the tetragonal cell. 
The change of coordinates can be written as above with the rotation matrix 
\begin{equation}
R^D=\left( \begin{array}{ccc}
\cos\phi, & 0, & -\sin\phi \\
0, & 1, & 0 \\
\sin\phi, & 0, & \cos\phi  
\end{array}
\right)
\end{equation}
where $\tan \phi=-\epsilon_{1,3}$.

Applying $\epsilon_H$ transforms the centered tetragonal cell into 
a base-centered $b$-unique monoclinic cell with
\begin{eqnarray}
a&=&\sqrt{(a_t^2+c_t^2)(1+\epsilon_{1,3}^2)-4\epsilon_{1,3}a_tc_t}, \\
b&=&a_t, \\
c&=&a_t \sqrt{1+\epsilon_{1,3}^2}, \\
\cos\beta &=& {a_t (1+\epsilon_{1,3}^2) -2 \epsilon_{1,3} c_t \over 
\sqrt{(a_t^2+c_t^2)(1+\epsilon_{1,3}^2)-4\epsilon_{1,3}a_tc_t} 
\sqrt{(1+\epsilon_{1,3}^2)}}.
\end{eqnarray}
The reference system of the base-centered monoclinic cell is rotated with 
respect to the reference system of the centered tetragonal cell. 
The change of coordinates can be written as above with the rotation matrix 
$R^D$ and the angle $\phi$ equal to
\begin{equation}
{\rm tg} \phi = -{\epsilon_{1,3} a_t - c_t \over a_t - \epsilon_{3,1} c_t}.
\end{equation}

\subsection{\color{web-blue}Orthorhombic system}
Applying $\epsilon_A$, $\epsilon_B$, $\epsilon_{B'}$, $\epsilon_{B''}$, 
$\epsilon_C$, $\epsilon_D$, or $\epsilon_E$, to a (base-centered, 
face-centered, or body-centered) orthorhombic cell does not change its shape. 
$\epsilon_A$ changes $a_o$ into $a_o(1+\epsilon_{1,1})$, $b_o$ into
$b_o(1+\epsilon_{1,1})$ and $c_o$ into $c_o(1+\epsilon_{1,1})$.
$\epsilon_B$ changes $a_o$ into $a_o(1+\epsilon_{1,1})$ and $b_o$ into
$b_o(1+\epsilon_{1,1})$ while $c_o$ remains constant.
$\epsilon_{B'}$ changes $a_o$ into $a_o(1+\epsilon_{1,1})$ and $c_o$ into
$c_o(1+\epsilon_{1,1})$ while $b_o$ remains constant.
$\epsilon_{B''}$ changes $b_o$ into $b_o(1+\epsilon_{1,1})$ and $c_o$ into
$c_o(1+\epsilon_{1,1})$ while $a_o$ remains constant.
$\epsilon_C$ changes $a_o$ into 
$a_o(1+\epsilon_{1,1})$, while $b_o$ and $c_o$ remain constant.
$\epsilon_D$ changes $b_o$ into $b_o(1+\epsilon_{2,2})$ while
$a_o$ and $c_o$ remain constant.
$\epsilon_E$ changes $c_o$ into $c_o(1+\epsilon_{3,3})$ while $a_o$ and
$b_o$ remain constant.

Applying $\epsilon_G$ to an simple orthorhombic cell transforms it into a 
simple monoclinic ($c$-unique) cell with
\begin{eqnarray}
a&=&a_o \sqrt{1+\epsilon_{1,2}^2}, \\
b&=&b_o \sqrt{1+\epsilon_{1,2}^2}, \\
c&=&c_o, \\
\cos\gamma &=& {2 \epsilon_{1,2} \over 1 + \epsilon_{1,2}^2}.
\end{eqnarray}
The reference system of the simple monoclinic cell is rotated with respect to
the reference system of the orthorhombic cell. 
The change of coordinates can be written as above with the rotation matrix 
\begin{equation}
R^E=\left( \begin{array}{ccc}
\cos\phi, & -\sin\phi, & 0 \\
\sin\phi, &  \cos\phi, & 0 \\
0, & 0, & 1 \\
\end{array}
\right),
\end{equation}
where $\tan \phi=-\epsilon_{1,2}$.

Similarly, applying $\epsilon_H$ to a simple (base-centered) 
orthorhombic lattice transforms
it in a simple (base-centered) monoclinic lattice ($b$-unique) with 
\begin{eqnarray}
a&=&a_o \sqrt{1+\epsilon_{1,3}^2}, \\
b&=&b_o, \\
c&=&c_o \sqrt{1+\epsilon_{1,3}^2}, \\
\cos\beta &=& {2 \epsilon_{1,3} \over 1 + \epsilon_{1,3}^2}. 
\end{eqnarray}
The reference system of the simple (base-centered) monoclinic cell is 
rotated with respect to the reference system of the simple (base-centered) 
orthorhombic cell. 
The change of coordinates can be written as above with the rotation matrix 
$R^D$ and $\tan \phi=-\epsilon_{1,3}$.

$\epsilon_I$ transforms the simple (base-centered) orthorhombic lattice 
in a simple (base-centered) monoclinic lattice that 
however is $a$-unique and must be rotated to be simulated.
For instance we can simulate the strained orthorhombic cell with a monoclinic
$c$-unique lattice with
\begin{eqnarray}
a&=&b_o \sqrt{1+\epsilon_{2,3}^2}, \\
b&=&c_o \sqrt{1+\epsilon_{2,3}^2}, \\
c&=&a_o, \\
\cos\gamma &=& {2 \epsilon_{2,3} \over 1 + \epsilon_{2,3}^2}. 
\end{eqnarray}
The reference system of the simple (base-centered) monoclinic cell is 
rotated with respect to
the reference system of the simple (base-centered) orthorhombic cell. 
The change of coordinates can be written as above with the rotation matrix 
\begin{equation}
R^F=\left( \begin{array}{ccc}
0, & \cos \phi, &  -\sin \phi
\\
0, & \sin \phi, & \cos \phi 
\\
1, & 0, & 0
\end{array}
\right),
\end{equation}
and ${\rm tg} \phi =-\epsilon_{2,3}$. 

The application of $\epsilon_G$ to a base-centered orthorhombic lattice
transforms it into a $c$-unique simple monoclinic lattice with parameters:
\begin{eqnarray}
a&=&a_o \sqrt{1+\epsilon_{1,2}^2}, \\
b&=&{1\over 2} \sqrt{(a_o^2+b_o^2)(1+\epsilon_{1,2}^2)+4a_ob_o\epsilon_{1,2}}, \\
c&=&c_o,  \\
\cos\gamma &=& {a_o (1+\epsilon_{1,2}^2)+2a_ob_o\epsilon_{1,2} \over
\sqrt{1+\epsilon_{1,2}^2} \sqrt{(a_o^2+b_o^2)
(1+\epsilon_{1,2}^2)+4a_ob_o\epsilon_{1,2}}}. 
\end{eqnarray}
The reference system of the simple monoclinic cell is rotated with respect to
the reference system of the orthorhombic cell. 
The change of coordinates can be written as above with the rotation matrix 
$R^E$ and ${\rm tg}\phi=-\epsilon_{1,2}$.

The application of $\epsilon_G$ to a face-centered orthorhombic lattice
transforms it into a $c$-unique base-centered monoclinic lattice with 
parameters:
\begin{eqnarray}
a&=&a_o \sqrt{1+\epsilon_{1,2}^2}, \\
b&=&{1\over 2} \sqrt{(a_o^2+b_o^2)(1+\epsilon_{1,2}^2)+4a_ob_o\epsilon_{1,2}}, \\
c&=&c_o,  \\
\cos\gamma &=& {a_o (1+\epsilon_{1,2}^2)+2a_ob_o\epsilon_{1,2} \over
\sqrt{1+\epsilon_{1,2}^2} \sqrt{(a_o^2+b_o^2)
(1+\epsilon_{1,2}^2)+4a_ob_o\epsilon_{1,2}}}. 
\end{eqnarray}
The reference system of the base-centered monoclinic cell is rotated 
with respect to the reference system of the face-centered orthorhombic cell. 
The change of coordinates can be written as above with the rotation matrix 
$R^E$ and ${\rm tg}\phi=-\epsilon_{1,2}$.

The application of $\epsilon_H$ to a face-centered orthorhombic lattice
transforms it into a $b$-unique base-centered monoclinic lattice with
parameters:
\begin{eqnarray}
a&=&a_o \sqrt{1+\epsilon_{1,3}^2}, \\
b&=&b_o,  \\
c&=&{1\over 2} \sqrt{(a_o^2+c_o^2)(1+\epsilon_{1,3}^2)+4a_oc_o\epsilon_{1,3}}, \\
\cos\gamma &=& {a_o (1+\epsilon_{1,3}^2)+2a_oc_o\epsilon_{1,3} \over
\sqrt{1+\epsilon_{1,3}^2} \sqrt{(a_o^2+c_o^2)
(1+\epsilon_{1,3}^2)+4a_oc_o\epsilon_{1,3}}}. 
\end{eqnarray}
The reference system of the base-centered monoclinic cell is rotated 
with respect to the reference system of the face-centered orthorhombic cell. 
The change of coordinates can be written as above with the rotation matrix 
$R^D$ and ${\rm tg}\phi=-\epsilon_{1,3}$.

The application of $\epsilon_I$ to a face-centered orthorhombic lattice
transforms it into a base-centered monoclinic lattice that however is 
$a$-unique, so we have to rotate it. For instance we can transform it
in a base-centered $c$-unique monoclinic lattice with parameters:
\begin{eqnarray}
a&=&b_o \sqrt{1+\epsilon_{2,3}^2}, \\
b&=&{1\over 2} \sqrt{(b_o^2+c_o^2)(1+\epsilon_{2,3}^2)+4b_oc_o
\epsilon_{2,3}}, \\
c&=&a_o,  \\
\cos\gamma &=& {b_o^2 (1+\epsilon_{2,3}^2)+2b_oc_o\epsilon_{2,3} \over
\sqrt{1+\epsilon_{2,3}^2} \sqrt{(b_o^2+c_o^2)
(1+\epsilon_{2,3}^2)+4b_oc_o\epsilon_{2,3}}}. 
\end{eqnarray}
The reference system of the base-centered monoclinic cell is rotated 
with respect to the reference system of the face-centered orthorhombic cell. 
The change of coordinates can be written as above with the rotation matrix 
$R^F$ and 
${\rm tg}\phi=-\epsilon_{2,3}$.

The application of $\epsilon_G$ to a body-centered orthorhombic lattice
transforms it into a base-centered $c$-unique monoclinic lattice with 
parameters:
\begin{eqnarray}
a&=&\sqrt{(a_o^2+b_o^2)(1+\epsilon_{1,2}^2)-4a_ob_o\epsilon_{1,2}}, \\
a&=&a_o \sqrt{1+\epsilon_{1,2}^2}, \\
c&=&c_o,  \\
\cos\gamma &=& {a_o (1+\epsilon_{1,2}^2)-2a_ob_o\epsilon_{1,2} \over
\sqrt{1+\epsilon_{1,2}^2} \sqrt{(a_o^2+b_o^2)
(1+\epsilon_{1,2}^2)-4a_ob_o\epsilon_{1,2}}}, 
\end{eqnarray}
The reference system of the base-centered monoclinic cell is rotated 
with respect to the reference system of the body-centered orthorhombic cell. 
The change of coordinates can be written as above with the rotation matrix 
$R^E$ and
\begin{equation}
{\rm tg} \phi = -{ \epsilon a_o - b_o \over a_o - \epsilon b_o }.
\end{equation}

The application of a strain $\epsilon_H$ to a body-centered orthorhombic
lattice transforms it into a base-centered $b$-unique monoclinic lattice 
\begin{eqnarray}
a&=&\sqrt{(a_o^2+c_o^2)(1+\epsilon_{1,3}^2)-4\epsilon_{1,3}a_oc_o}, \\
b&=&b_o, \\
c&=&a_o \sqrt{1+\epsilon_{1,3}^2}, \\
\cos\beta &=& {a_o (1+\epsilon_{1,3}^2) -2 \epsilon_{1,3} c_o \over 
\sqrt{(a_o^2+c_o^2)(1+\epsilon_{1,3}^2)-4\epsilon_{1,3}a_oc_o} 
\sqrt{(1+\epsilon_{1,3}^2)}}.
\end{eqnarray}
The reference system of the base-centered monoclinic cell is rotated 
with respect to the reference system of the body-centered orthorhombic cell. 
The change of coordinates can be written as above with the rotation matrix 
$R^D$ and
\begin{equation}
{\rm tg} \phi = -{\epsilon_{1,3} a_o - c_o \over a_o - \epsilon_{3,1} c_o}.
\end{equation}

The application of a strain $\epsilon_I$ to a body-centered orthorhombic
lattice transforms it into a base-centered monoclinic lattice that however
is $a$-unique. In order to simulate it we must rotate it so that it
becomes for instance a base-centered $c$-unique monoclinic lattice with
parameters
\begin{eqnarray}
a&=&\sqrt{(b_o^2+c_o^2)(1+\epsilon_{2,3}^2)-4\epsilon_{2,3}b_oc_o}, \\
b&=&b_o \sqrt{1+\epsilon_{2,3}^2}, \\
c&=&a_o \\
\cos\gamma &=& {b_o (1+\epsilon_{2,3}^2) -2 \epsilon_{2,3} c_o \over 
\sqrt{(b_o^2+c_o^2)(1+\epsilon_{2,3}^2)-4\epsilon_{2,3}b_oc_o} 
\sqrt{(1+\epsilon_{2,3}^2)}}.
\end{eqnarray}
The reference system of the base-centered monoclinic cell is rotated 
with respect to the reference system of the body-centered orthorhombic cell. 
The change of coordinates can be written as above with the rotation matrix 
$R^F$ and
and 
\begin{equation}
{\rm tg} \phi = -{\epsilon_{2,3} b_o - c_o \over b_o - \epsilon_{3,2} c_o}.
\end{equation}

\subsection{\color{web-blue} Monoclinic system}
Applying $\epsilon_A$, $\epsilon_C$, $\epsilon_D$, $\epsilon_E$, or
$\epsilon_G$ to a $c$-unique simple (base-centered) monoclinic cell does 
not change its shape.

Applying $\epsilon_A$ to a $c$-unique simple (base-centered) monoclinic 
lattice changes
$a_m$ to $a_m(1+\epsilon_{1,1})$, $b_m$ to $b_m(1+\epsilon_{1,1})$
and $c_m$ to $c_m(1+\epsilon_{1,1})$ while $\cos \gamma_m$
does not change.

Applying $\epsilon_C$ to a $c$-unique simple (base-centered) monoclinic lattice 
changes $a_m$, $b_m$, $c_m$, and $\cos \gamma_m$ to 
\begin{eqnarray}
a&=&a_m(1+\epsilon_{1,1}), \\
b&=&b_m \sqrt{1+2 \epsilon_{1,1}\cos^2\gamma_m
+\epsilon_{1,1}^2 \cos^2\gamma_m}, \\
c&=&c_m \\
\cos\gamma &=& {(1+\epsilon_{1,1}) \cos\gamma_m \over 
\sqrt{1+2 \epsilon_{1,1}\cos^2\gamma_m
+\epsilon_{1,1}^2 \cos^2\gamma_m}},
\end{eqnarray}

Applying $\epsilon_D$ to a $c$-unique simple (base-centered) monoclinic lattice 
changes $a_m$, $b_m$, $c_m$, and $\cos \gamma_m$ to 
\begin{eqnarray}
a&=&a_m, \\
b&=&b_m \sqrt{1+2 \epsilon_{2,2}\sin^2\gamma_m
+\epsilon_{2,2}^2 \sin^2\gamma_m}, \\
c&=&c_m \\
\cos\gamma &=& {\cos\gamma_m \over 
\sqrt{1+2 \epsilon_{2,2}\sin^2\gamma_m
+\epsilon_{2,2}^2 \sin^2\gamma_m}}.
\end{eqnarray}

Applying $\epsilon_E$ to a $c$-unique simple (base-centered) monoclinic 
lattice changes
only $c_m$ to $c_m(1+\epsilon_{3,3})$ while all the other crystal
parameter do not change.

Applying $\epsilon_G$ to a $c$-unique simple (base-centered) monoclinic lattice 
changes $a_m$, $b_m$, $c_m$, and $\cos \gamma_m$ to 
\begin{eqnarray}
a&=&a_m \sqrt{1+\epsilon_{1,2}^2}, \\
b&=&b_m \sqrt{1+\epsilon_{1,2}^2
+4\epsilon_{1,2} \cos\gamma_m \sin\gamma_m}, \\
c&=&c_m, \\
\cos\gamma &=& {\cos\gamma_m (1+\epsilon_{1,2}^2) + 2 \epsilon_{1,2} 
\sin \gamma_m \over \sqrt{1+\epsilon_{1,2}^2}
\sqrt{1+\epsilon_{1,2}^2+4\epsilon_{1,2} \cos\gamma_m \sin\gamma_m}}.
\end{eqnarray}
The reference system of the strained simple (base-centered) monoclinic 
lattice is rotated 
with respect to the reference system of the unstrained simple 
(base-centered) monoclinic lattice. 
The change of coordinates can be written as above with the rotation matrix 
$R^E$ and ${\rm tg}\phi=-\epsilon_{1,2}$.

Applying $\epsilon_H$ to a $c$-unique simple monoclinic lattice transforms
it into a triclinic cell with parameters:
\begin{eqnarray}
a&=&a_m \sqrt{1+\epsilon_{1,3}^2}, \\
b&=&b_m \sqrt{1+\epsilon_{1,3}^2 \cos^2\gamma_m}, \\
c&=&c_m \sqrt{1+\epsilon_{1,3}^2}, \\
\cos\alpha &=& {2 \epsilon_{1,3} \cos\gamma_m \over 
\sqrt{1+\epsilon_{1,3}^2} \sqrt{1+\epsilon_{1,3}^2 \cos^2\gamma_m}}, \\
\cos\beta &=& {2 \epsilon_{1,3}^2 \over 1+\epsilon_{1,3}^2}, \\
\cos\gamma &=& {\cos\gamma_m \sqrt{1+\epsilon_{1,3}^2} \over
\sqrt{1+\epsilon_{1,3}^2 \cos^2\gamma_m}}.
\end{eqnarray}
The reference system of the triclinic lattice is rotated 
with respect to the reference system of the simple 
monoclinic lattice. 
The change of coordinates can be written as above with the rotation matrix 
$R^D$ and ${\rm tg}\phi=-\epsilon_{1,3}$.

Applying $\epsilon_I$ to a $c$-unique simple monoclinic lattice transforms
it into a triclinic cell with parameters:
\begin{eqnarray}
a&=&a_m,  \\
b&=&b_m \sqrt{1+\epsilon_{2,3}^2 \sin^2\gamma_m}, \\
c&=&c_m \sqrt{1+\epsilon_{2,3}^2}, \\
\cos\alpha &=& {2 \epsilon_{2,3} \sin\gamma_m \over 
\sqrt{1+\epsilon_{2,3}^2} \sqrt{1+\epsilon_{2,3}^2 \sin^2\gamma_m}}, \\
\cos\beta &=& 0, \\
\cos\gamma &=& { \cos\gamma_m  \over
\sqrt{1+\epsilon_{2,3}^2 \sin^2\gamma_m}}.
\end{eqnarray}
The reference system of the triclinic lattice is rotated 
with respect to the reference system of the simple 
monoclinic lattice. 
The change of coordinates can be written as above with the rotation matrix 
\begin{equation}
R^G=\left( \begin{array}{ccc}
1, & 0, & 0 \\
0, & \cos \phi, &  -\sin \phi
\\
0, & \sin \phi, & \cos \phi 
\end{array}
\right),
\end{equation}
where  ${\rm tg}\phi=-\epsilon_{2,3}$.

Applying $\epsilon_H$ to a $c$-unique base-centered monoclinic lattice 
transforms it into a triclinic lattice with parameters:
\begin{eqnarray}
a&=&a_m \sqrt{1+\epsilon_{1,3}^2}, \\
b&=&b_m \sqrt{1+\epsilon_{1,3}^2 \cos^2\gamma_m}, \\
c&=&{1\over 2} \sqrt{(a_m^2+c_m^2)(1+\epsilon_{1,3}^2)+4\epsilon_{1,3}a_mc_m},\\
\cos\alpha &=& {\cos\gamma_m (a_m (1+\epsilon_{1,3}^2)+2\epsilon_{1,3}c_m) 
\over 
\sqrt{1+\epsilon_{1,3}^2 \cos^2\gamma_m} \sqrt{(a_m^2+c_m^2)(1+\epsilon_{1,3}^2)+4\epsilon_{1,3}a_mc_m}}, \\
\cos\beta &=& {a_m (1+\epsilon_{1,3}^2)+2\epsilon_{1,3}c_m 
\over 
\sqrt{1+\epsilon_{1,3}^2} \sqrt{(a_m^2+c_m^2)(1+\epsilon_{1,3}^2)+4\epsilon_{1,3}a_mc_m}}, \\
\cos\gamma &=& {\cos\gamma_m \sqrt{1+\epsilon_{1,3}^2} \over
\sqrt{1+\epsilon_{1,3}^2 \cos^2\gamma_m}}.
\end{eqnarray}
The reference system of the triclinic lattice is rotated 
with respect to the reference system of the base-centered
monoclinic lattice. 
The change of coordinates can be written as above with the rotation matrix 
$R^D$ and ${\rm tg}\phi=-\epsilon_{1,3}$.

Applying $\epsilon_I$ to a $c$-unique base-centered monoclinic lattice 
transforms it into a triclinic lattice with parameters:
\begin{eqnarray}
a&=&a_m, \\
b&=&b_m \sqrt{1+\epsilon_{2,3}^2 \sin^2\gamma_m}, \\
c&=&{1\over 2} \sqrt{a_m^2+c_m^2(1+\epsilon_{2,3}^2)},\\
\cos\alpha &=& {a_m\cos\gamma_m +2\epsilon_{2,3}c_m \sin \gamma_m 
\over 
\sqrt{1+\epsilon_{2,3}^2 \sin^2\gamma_m} 
\sqrt{a_m^2+c_m^2(1+\epsilon_{2,3}^2)}}, \\
\cos\beta &=& {a_m  
\over 
\sqrt{a_m^2+c_m^2(1+\epsilon_{2,3}^2)}}, \\
\cos\gamma &=& {\cos\gamma_m \over
\sqrt{1+\epsilon_{2,3}^2 \sin^2\gamma_m}}.
\end{eqnarray}
The reference system of the triclinic lattice is rotated 
with respect to the reference system of the base-centered
monoclinic lattice. 
The change of coordinates can be written as above with the rotation matrix 
$R^G$ and ${\rm tg}\phi=-\epsilon_{2,3}$.

Applying $\epsilon_A$, $\epsilon_C$, $\epsilon_D$, $\epsilon_E$, and
$\epsilon_H$ to a $b$-unique simple (base-centered) monoclinic lattice
does not change its shape.

Applying $\epsilon_A$ to a $b$-unique simple (base-centered) monoclinic 
lattice changes
$a_m$ to $a_m(1+\epsilon_{1,1})$, $b_m$ to $b_m(1+\epsilon_{1,1})$
and $c_m$ to $c_m(1+\epsilon_{1,1})$ while $\cos \beta_m$
does not change.

Applying $\epsilon_C$ to a $b$-unique simple (base-centered) monoclinic lattice 
changes $a_m$, $b_m$, $c_m$, and $\cos \beta_m$ to 
\begin{eqnarray}
a&=&a_m(1+\epsilon_{1,1}), \\
b&=&b_m \\
c&=&c_m \sqrt{1+2 \epsilon_{1,1}\cos^2\beta_m
+\epsilon_{1,1}^2 \cos^2\beta_m}, \\
\cos\beta &=& {(1+\epsilon_{1,1}) \cos\beta_m \over 
\sqrt{1+2 \epsilon_{1,1}\cos^2\beta_m
+\epsilon_{1,1}^2 \cos^2\beta_m}},
\end{eqnarray}

Applying $\epsilon_D$ to a $b$-unique simple (base-centered) monoclinic 
lattice changes
only $b_m$ to $b_m(1+\epsilon_{2,2})$ while all the other crystal
parameter do not change.

Applying $\epsilon_E$ to a $b$-unique simple (base-centered) monoclinic lattice 
changes $a_m$, $b_m$, $c_m$, and $\cos \beta_m$ to 
\begin{eqnarray}
a&=&a_m, \\
b&=&b_m \\
c&=&c_m \sqrt{1+2 \epsilon_{3,3}\sin^2\beta_m
+\epsilon_{3,3}^2 \sin^2\beta_m}, \\
\cos\beta &=& {\cos\beta_m \over \sqrt{1+2 \epsilon_{3,3}\sin^2\beta_m
+\epsilon_{3,3}^2 \sin^2\beta_m}}.
\end{eqnarray}

Applying $\epsilon_H$ to a $b$-unique simple (base-centered) monoclinic lattice 
changes $a_m$, $b_m$, $c_m$, and $\cos \beta_m$ to 
\begin{eqnarray}
a&=&a_m \sqrt{1+\epsilon_{1,3}^2}, \\
b&=&b_m, \\
c&=&c_m \sqrt{1+\epsilon_{1,3}^2
+4\epsilon_{1,3} \cos\beta_m \sin\beta_m}, \\
\cos\beta &=& {\cos\beta_m (1+\epsilon_{1,3}^2) + 2 \epsilon_{1,3} 
\sin \beta_m \over \sqrt{1+\epsilon_{1,3}^2}
\sqrt{1+\epsilon_{1,3}^2+4\epsilon_{1,3} \cos\beta_m \sin\beta_m}}.
\end{eqnarray}
The reference system of the strained simple (base-centered) monoclinic 
lattice is rotated 
with respect to the reference system of the unstrained simple 
(base-centered) monoclinic lattice. 
The change of coordinates can be written as above with the rotation matrix 
$R^D$ and ${\rm tg}\phi=-\epsilon_{1,3}$.

Applying $\epsilon_G$ to a $b$-unique simple monoclinic lattice transforms
it into a triclinic lattice with parameters:
\begin{eqnarray}
a&=&a_m \sqrt{1+\epsilon_{1,2}^2}, \\
b&=&b_m \sqrt{1+\epsilon_{1,2}^2}, \\
c&=&c_m \sqrt{1+\epsilon_{1,2}^2 \cos^2\beta_m}, \\
\cos\alpha &=& {2 \epsilon_{1,2} \cos\beta_m \over 
\sqrt{1+\epsilon_{1,2}^2} \sqrt{1+\epsilon_{1,2}^2 \cos^2\beta_m}}, \\
\cos\beta &=& {\cos\beta_m \sqrt{1+\epsilon_{1,2}^2} \over
\sqrt{1+\epsilon_{1,2}^2 \cos^2\beta_m}}, \\
\cos\gamma &=& {2 \epsilon_{1,2}^2 \over 1+\epsilon_{1,2}^2}.
\end{eqnarray}
The reference system of the triclinic lattice is rotated 
with respect to the reference system of the simple 
monoclinic lattice. 
The change of coordinates can be written as above with the rotation matrix 
$R^E$ and ${\rm tg}\phi=-\epsilon_{1,2}$.

Applying $\epsilon_I$ to a $b$-unique simple monoclinic lattice transforms
it into a triclinic lattice with parameters:
\begin{eqnarray}
a&=&a_m,  \\
b&=&b_m \sqrt{1+\epsilon_{2,3}^2}, \\
c&=&c_m \sqrt{1+\epsilon_{2,3}^2 \sin^2\beta_m}, \\
\cos\alpha &=& {2 \epsilon_{2,3} \sin\beta_m \over 
\sqrt{1+\epsilon_{2,3}^2} \sqrt{1+\epsilon_{2,3}^2 \sin^2\beta_m}}, \\
\cos\beta &=& { \cos\beta_m  \over
\sqrt{1+\epsilon_{2,3}^2 \sin^2\beta_m}}, \\
\cos\gamma &=& 0. 
\end{eqnarray}
The reference system of the triclinic lattice is rotated 
with respect to the reference system of the simple 
monoclinic lattice. 
The change of coordinates can be written as above with the rotation matrix 
$R^G$ and ${\rm tg}\phi=-\epsilon_{2,3}$.

Applying $\epsilon_G$ to a $b$-unique base-centered monoclinic lattice 
transforms it into a triclinic lattice with parameters:
\begin{eqnarray}
a&=&a_m \sqrt{1+\epsilon_{1,2}^2}, \\
b&=&{1\over 2} \sqrt{(a_m^2+b_m^2)(1+\epsilon_{1,2}^2)+4\epsilon_{1,2}a_mb_m},\\
c&=&c_m \sqrt{1+\epsilon_{1,2}^2 \cos^2\beta_m}, \\
\cos\alpha &=& {\cos\beta_m (a_m (1+\epsilon_{1,2}^2)+2\epsilon_{1,2}b_m) 
\over 
\sqrt{1+\epsilon_{1,2}^2 \cos^2\beta_m} \sqrt{(a_m^2+b_m^2)(1+\epsilon_{1,2}^2)+4\epsilon_{1,2}a_mb_m}}, \\
\cos\beta &=& {\cos\beta_m \sqrt{1+\epsilon_{1,2}^2} \over
\sqrt{1+\epsilon_{1,2}^2 \cos^2\beta_m}}, \\
\cos\gamma &=& {a_m (1+\epsilon_{1,2}^2)+2\epsilon_{1,2}b_m 
\over 
\sqrt{1+\epsilon_{1,2}^2} \sqrt{(a_m^2+b_m^2)(1+\epsilon_{1,2}^2)+4\epsilon_{1,2}a_mb_m}}.
\end{eqnarray}
The reference system of the triclinic lattice is rotated 
with respect to the reference system of the base-centered
monoclinic lattice. 
The change of coordinates can be written as above with the rotation matrix 
$R^E$ and ${\rm tg}\phi=-\epsilon_{1,2}$.

Applying $\epsilon_I$ to a $b$-unique base-centered monoclinic lattice 
transforms it into a triclinic lattice with parameters:
\begin{eqnarray}
a&=&a_m, \\
b&=&{1\over 2} \sqrt{a_m^2+b_m^2(1+\epsilon_{2,3}^2)},\\
c&=&c_m \sqrt{1+\epsilon_{2,3}^2 \sin^2\beta_m}, \\
\cos\alpha &=& {a_m\cos\beta_m +2\epsilon_{2,3}b_m \sin \beta_m 
\over 
\sqrt{1+\epsilon_{2,3}^2 \sin^2\beta_m} 
\sqrt{a_m^2+b_m^2(1+\epsilon_{2,3}^2)}}, \\
\cos\beta &=& {\cos\beta_m \over
\sqrt{1+\epsilon_{2,3}^2 \sin^2\beta_m}}, \\
\cos\gamma &=& {a_m  
\over 
\sqrt{a_m^2+b_m^2(1+\epsilon_{2,3}^2)}}.
\end{eqnarray}
The reference system of the triclinic lattice is rotated 
with respect to the reference system of the base-centered
monoclinic lattice. 
The change of coordinates can be written as above with the rotation matrix 
$R^G$ and ${\rm tg}\phi=-\epsilon_{2,3}$.

\subsection{\color{web-blue}Triclinic system}
Any strain applied to a triclinic system will keep the shape of the
unit cell. Only the crystal parameters change, but there is no advantage
to use a rotated triclinic cell with respect to the cell obtained
simply applying the strain directly to the primitive vectors. Therefore
we do not discuss this case any further.

\subsection{\color{web-blue}Hexagonal system}

Applying $\epsilon_A$, $\epsilon_B$, or $\epsilon_E$, to an hexagonal cell 
does not change its shape.
Applying $\epsilon_A$, $a_h$ becomes $a_h(1+\epsilon_{1,1})$ and
$c_h$ becomes $c_h(1+\epsilon_{1,1})$ so the ratio $c_h/a_h$ does not
change.
Applying $\epsilon_B$, $a_h$ becomes $a_h(1+\epsilon_{1,1})$ and
$c_h$ does not change, while applying $\epsilon_E$, $c_h$ 
becomes $c_h(1+\epsilon_{3,3})$ and $a_h$ does not change.

Applying $\epsilon_{B'}$ to an hexagonal lattice, it becomes
a base-centered orthorhombic cell with:
\begin{eqnarray}
a&=&a_h (1+\epsilon_{1,1}), \\
b&=&a_h \sqrt{3}, \\
c&=&c_h (1+\epsilon_{1,1}).
\end{eqnarray}

Applying $\epsilon_C$ to an hexagonal lattice, it becomes a base-centered 
orthorhombic lattice with:
\begin{eqnarray}
a&=&a_h (1+\epsilon_{1,1}), \\
b&=&a_h \sqrt{3}, \\
c&=&c_h.
\end{eqnarray}

Applying $\epsilon_H$ to an hexagonal lattice, it becomes a base-centered 
$b$-unique monoclinic cell with
\begin{eqnarray}
a&=&a_h \sqrt{1 + \epsilon_{1,3}^2}, \\
b&=&a_h \sqrt{3}, \\
c&=&c_h \sqrt{1 + \epsilon_{1,3}^2}, \\
\cos\alpha &=& {2 \epsilon_{1,3} \over 1+\epsilon_{1,3}^2}.
\end{eqnarray}
The reference system of the base-centered monoclinic cell is rotated 
with respect to the reference system of the hexagonal cell. 
The change of coordinates can be written as above with the rotation matrix 
$R^D$ and ${\rm tg}\phi=-\epsilon_{1,3}$.

\subsection{\color{web-blue}Trigonal system}

Applying $\epsilon_A$ or $\epsilon_E$ to a rhombohedral lattice does
not change its shape. 

Applying $\epsilon_A$ changes $a_r$ to $a_r (1+\epsilon_{1,1})$ while the
rhombohedral angle does not change.

Applying $\epsilon_E$ changes both $a_r$ and the rhombohedral angle
$\alpha_r$. We have 
\begin{eqnarray}
a&=&a_t \sqrt{\sin^2\theta_r + (1 + \epsilon_{1,1})^2 \cos^2\theta_r}, \\
\cos\alpha &=& {
2(1 + \epsilon_{1,1})^2 \cos^2\theta_r -\sin^2\theta_r  \over
2 \left[\sin^2\theta_r+ (1 + \epsilon_{1,1})^2 \cos^2\theta_r\right]}
\end{eqnarray}
where 
$\sin{\theta_r}=\sqrt{2\over 3}\sqrt{1-\cos{\alpha_r}}$
and $\cos{\theta_r}=\sqrt{1\over 3}\sqrt{1 + 2 \cos{\alpha_r}}$. \\

Applying $\epsilon_C$ to a rhombohedral lattice transforms it into a
base-centered $b$-unique monoclinic lattice with parameters:
\begin{eqnarray}
a&=&a_r \sqrt{1 + 3 \cos^2 \theta_r}, \\
b&=&a_r \sqrt{3} (1 + \epsilon_{1,1}) \sin \theta_r, \\
c&=&a_r, \\
\cos\beta &=& {3 \cos^2 \theta_r -1 \over \sqrt{1 + 3 \cos^2 \theta_r} }.
\end{eqnarray}
The reference system of the base-centered monoclinic lattice is rotated 
with respect to the reference system of the rhombohedral lattice.
The change of coordinates can be written as above with the rotation matrix 
\begin{equation}
R^H=\left( \begin{array}{ccc}
0, & -{\sin\theta_r \over \sqrt{1 + 3 \cos^2 \theta_r}}, &  {2\cos\theta_r \over \sqrt{1 + 3 \cos^2 \theta_r}}\\
1, & 0, & 0  
\\
0, & {2\cos\theta_r \over \sqrt{1 + 3 \cos^2 \theta_r}}, & 
{\sin\theta_r \over \sqrt{1 + 3 \cos^2 \theta_r}} 
\end{array}
\right).
\end{equation}

Applying $\epsilon_I$ to a rhombohedral lattice transforms it into a
base-centered $b$-unique monoclinic lattice with parameters:
\begin{eqnarray}
a&=&a_r \sqrt{(1+\epsilon_{2,3}^2)(1 + 3 \cos^2 \theta_r)-
8\epsilon_{2,3}\sin \theta_r\cos \theta_r}, \\
b&=&a_r \sqrt{3}\sin \theta_r, \\
c&=&a_r \sqrt{1+\epsilon_{2,3}^2+4\epsilon_{2,3}\sin \theta_r\cos \theta_r}, \\
\cos\beta &=& {(1+\epsilon_{2,3}^2) (3 \cos^2 \theta_r-1) 
+ 2 \epsilon_{2,3} \sin \theta_r\cos \theta_r
\over \sqrt{(1+\epsilon_{2,3}^2)(1 + 3 \cos^2 \theta_r)-
8\epsilon_{2,3}\sin \theta_r\cos \theta_r} \sqrt{1+\epsilon_{2,3}^2+4\epsilon_{2,3}\sin \theta_r\cos \theta_r}}.
\end{eqnarray}

The reference system of the base-centered monoclinic lattice is rotated 
with respect to the reference system of the rhombohedral lattice.
The change of coordinates can be written as above with the rotation matrix 
\begin{equation}
R^I=\left( \begin{array}{ccc}
0, & a, &  b\\
1, & 0, & 0  
\\
0, & b, & -a 
\end{array}
\right)
\end{equation}
where
\begin{eqnarray}
a&=& {2 \epsilon_{2,3} \cos \theta_r - \sin \theta_r \over 
\sqrt{(1+\epsilon_{2,3}^2)(1 + 3 \cos^2 \theta_r)-
8\epsilon_{2,3}\sin \theta_r\cos \theta_r}}, \\
b&=& {2 \cos \theta_r - \epsilon_{2,3} \sin \theta_r \over 
\sqrt{(1+\epsilon_{2,3}^2)(1 + 3 \cos^2 \theta_r)-
8\epsilon_{2,3}\sin \theta_r\cos \theta_r}}.
\end{eqnarray}

\newpage
\section{\color{coral}The stress tensor}

Consider a small volume $dV$ of the solid with the form of a prism with 
edges parallel to $x_1$, $x_2$, and $x_3$. Let us call $S_i$ one of
the two surfaces perpendicular to $x_i$ of area $dS_i$. If on this surface 
there is a force ${\bf f}_i=(f_{1,i}, f_{2,i}, f_{3,i})$ where the components
are along $x_1$, $x_2$, and $x_3$, we say that on the solid a
stress: 
\begin{equation}
\sigma_{j,i}=f_{j,i} / dS_i, 
\end{equation}
is acting.
We consider only uniform stresses
that do not depend on the position of the volume $dV$ on the solid.
The force on the other surface perpendicular to $x_i$ is 
$-{\bf f}_i=-(f_{1,i}, f_{2,i}, f_{3,i})$. With this definition a positive
stress pulls outward on the surface of the volume.

The stress is a symmetric tensor so it is possible to use the Voigt
notation. We have
$\sigma_1=\sigma_{1,1}$,
$\sigma_2=\sigma_{2,2}$, $\sigma_3=\sigma_{3,3}$, $\sigma_4=\sigma_{2,3}$,
$\sigma_5=\sigma_{1,3}$, and $\sigma_6=\sigma_{1,2}$.

A uniform pressure applied to the solid is a particular stress that has
only diagonal components. In Voigt notation we have
$\sigma_1=-p$, $\sigma_2=-p$, $\sigma_3=-p$, while the other components
vanish. Conversely we can compute the pressure as:
\begin{equation}
p=-{1\over 3} (\sigma_1 + \sigma_2 + \sigma_3).
\end{equation}
The negative sign indicates that a positive pressure reduces the
volume of the solid so it pushes on the surfaces.

\newpage
\section{\color{coral}Thermodynamic functions}

The thermodynamic functions can be introduced by using the first law
of thermodynamics and the definition of the entropy ($S$). The heat 
given reversibly to the solid can be written as $dQ=TdS$ where $T$ is
the temperature of the solid and $dS$ its entropy change. The change 
of the internal energy is given by $dU = TdS - dW$ where $dW$ is 
the work done by the solid. If the strain changes by $d\epsilon$ and 
the stress is $\sigma$ then $dW=-V \sum_i \sigma_i d\epsilon_i$, where $V$
is the volume of the unperturbed solid. \\
Therefore
\begin{equation}
dU=TdS + V \sum_{i=1}^6 \sigma_i d\epsilon_i.
\end{equation}
In our approach, this function is not particularly suited for 
calculations because we do not compute the internal energy as a function 
of entropy but as a function of temperature (see Eq.~\ref{uvib}). However
at $T=0$ ($S=0$) we have $\sigma_i={1 \over V} 
{\partial U \over \partial \epsilon_i}$ and, neglecting zero point motion,
we can calculate the derivative of $U_0$. 
Here the derivatives with 
respect to $\epsilon_i$ must be done keeping all the other strain 
components fixed. \\
At $\sigma=0$ the equilibrium strain minimizes $U$. \\ 
At finite stress $\sigma$, 
the equations of state $\sigma_i ={1\over V} 
{\partial U \over \partial \epsilon_i}$
can be obtained by minimizing
\begin{equation}
H= U - V \sum_{i=1}^6 \sigma_i \epsilon_i.
\end{equation}
We have:
\begin{equation}
{1\over V }{\partial H \over \partial \epsilon_i} = {1\over V} 
{\partial U \over \partial \epsilon_i}- 
\sigma_i = 0.
\end{equation}
$H$ is called enthalpy of the solid. \\
For dealing with finite temperatures, it is convenient to introduce the 
Helmholtz free energy by writing $F=U-TS$ so that
\begin{equation}
dF=-S dT + V \sum_{i=1}^6 \sigma_i d\epsilon_i.
\end{equation}
At fixed temperature, the equations of state of the solid are
\begin{equation}
\sigma_i= {1\over V }{\partial F \over \partial \epsilon_i}\Bigg|_T.
\end{equation}
and at zero stress, the strain minimizes $F$.
Here the volume $V$ is the volume at temperature $T$. \\
At finite temperature $T$ and stress $\sigma$, the equations of state
 $\sigma_i={1\over V}
{\partial F\over \partial \epsilon_i}\Big|_T$ can be obtained by 
minimizing the function:
\begin{equation}
G= F - V \sum_{i=1}^6 \sigma_i \epsilon_i.
\end{equation}
We have:
\begin{equation}
{1\over V} {\partial G \over \partial \epsilon_i}\Bigg |_T = {1\over V} 
{\partial F \over \partial \epsilon_i}\Bigg |_T - \sigma_i = 0.
\end{equation}
$G$ is called the Gibbs free energy of the solid.

\newpage
\section{\color{coral}Equilibrium geometry and thermal expansion}
To calculate the equilibrium geometry at $T=0$ and $\sigma=0$ it
is sufficient to minimize the internal energy with respect to strain:
\begin{equation}
{\partial U \over \partial \epsilon_i}=0.
\end{equation}
Neglecting the contribution of $U_{ph}$ (zero point motion) we can minimize 
$U_0$.\\
In practice this equation is difficult to use and one limits to consider
only strains that do not modify the shape of the unit cell. The
vanishing derivatives are imposed only for these strains.
This is equivalent to calculate $U$ as a function of the lattice parameters 
$\zeta_k$ and to check the minimum condition:
\begin{equation}
{\partial U \over \partial \zeta_k}=0.
\end{equation}
Similarly, at the temperature $T$, the equations that define the lattice 
parameters are:
\begin{equation}
{\partial F \over \partial \zeta_k}=0.
\end{equation}
Note that at $T=0$ this equation contains the effect of zero point nuclear
motion. By identifying the position of the minimum at each temperature 
we obtain the parameters $\zeta_k (T)$ that can be compared with
results of X-ray diffraction experiments.

The volume thermal expansion is defined as the variation of the crystal
volume when the temperature changes. Therefore using the parameters
$\zeta_k (T)$ we can calculate the volume $V(T)$
with the expressions written above.  
Numerical differentiation of $V(T)$ gives the volume thermal expansion:
\begin{equation}
\beta = {1 \over V(T)} {d V \over d T}.
\end{equation}

The calculation of the thermal expansion tensor is slightly more
complex. At each temperature the solid has equilibrium parameters
$\zeta_k(T)$. The thermal expansion tensor $\alpha_{i}$ is defined 
as the strain $d\epsilon_{i}(T)$ induced by a change of temperature $dT$. 
We can write
\begin{equation}
\alpha_{i} = {d \epsilon_{i}(T) \over  d T }.
\end{equation}
We have only the crystal parameters $\zeta_k (T)$ and their derivatives
with respect to $T$. These are sufficient to deduce the thermal expansion
$\alpha_{\bf l}(T)$ where ${\bf l}$ is a versor ${\bf l}=(l_1,l_2,l_3)$ 
perpendicular to a set of lattice planes. Considering 
a given set of lattice planes we can write the thermal expansion in
the direction perpendicular to the lattice planes in terms of the tensor 
components $\alpha_{i,j}$
of the thermal expansion as:
\begin{equation}
\alpha_{{\bf l}} = \sum_{ij} \alpha_{i,j} l_i l_j.
\end{equation}
Along the direction ${\bf l}$ we can write:
\begin{equation}
\alpha_{{\bf l}} = {1 \over d_{\bf l}} {d d_{\bf l} \over d T},
\end{equation}
where $d_{\bf l}$ is the interplanar distance for that set of lattice planes.
In the most complex case, for a triclinic solid we need six directions
to set up a linear system whose solution are the six independent components
of the thermal expansion tensor, while for the other crystal systems 
the thermal expansion can be determined in a simpler manner.
We discuss this point separately for each crystal system.

\subsection{\color{web-blue}Cubic system}
In the cubic system $k=1$ and $\zeta_1=a$. 
The thermal expansion tensor is diagonal and the only nonzero components
are all equal: $\alpha_1=\alpha_2=\alpha_3$. We can consider the 
lattice planes perpendicular to ${\bf l}=(0,0,1)$ and the thermal expansion
in the direction ${\bf l}$ is $\alpha_3$. The distance between planes
$d_{\bf l}=a$ in the simple cubic and $d_{\bf l}=a/2$ in the face-centered
and the body-centered cubic lattices so in all cases:  
\begin{equation}
\alpha_{i} =  {1\over a(T) } {d a(T) \over dT}.
\end{equation}

\subsection{\color{web-blue}Tetragonal and hexagonal systems}
In the tetragonal and hexagonal systems $k=2$, $\zeta_1=a$, $\zeta_2=c/a$.
The only nonzero components of the thermal expansion are 
$\alpha_{1}=\alpha_{2}$ and $\alpha_{3}$. We can consider the
lattice planes perpendicular to ${\bf l}=(1,0,0)$ whose distance
is $d_{\bf l}=a$ in the simple tetragonal lattice and $d_{\bf l}=a/2$ in 
centered tetragonal and hexagonal case. 
For this direction we have $\alpha_{\bf l}=\alpha_1$ and 
\begin{equation}
\alpha_1 = {1\over a(T)} {d a \over dT}.
\end{equation}
Then we can consider the planes perpendicular to ${\bf l}=(0,0,1)$.
The distance between these planes is $d_{\bf l}=c$ for the simple
tetragonal and hexagonal lattices and $d_{\bf l}=c/2$ for the 
centered tetragonal lattice. Since $\alpha_{\bf l}=\alpha_3$ we have
\begin{equation}
\alpha_3 = {1\over c(T)} {d c \over dT}. 
\end{equation}
Choosing $a$ and $c/a$ as parameters,
we have to use the equations $c(T)=a(T) (c/a(T))$ and 
\begin{equation}
{dc \over d T} = {da \over d T} c/a(T) + a(T) {d c/a \over dT},
\end{equation}
and we have
\begin{equation}
\alpha_1 = \alpha_2= {1\over a(T)} {d a \over dT}, \ \ \ 
\alpha_3 = {1\over a(T)} {da \over d T}  + {1\over c/a(T)} {d c/a \over dT}.
\end{equation}

\subsection{\color{web-blue}Trigonal system}
In the trigonal system $k=2$, $\zeta_1=a$, $\zeta_2=\cos\alpha$. 
The thermal expansion tensor has three diagonal components 
$\alpha_1=\alpha_2$ and $\alpha_3$. We can consider the rhombohedral
lattice as a centered hexagonal lattice with three points per cell
and parameters $a_h$ and $c_h$ related to the rhobohedral parameters
by
\begin{equation}
a_h= a \sqrt{2} \sqrt{1-\cos\alpha}
\end{equation}
and 
\begin{equation}
c_h= a \sqrt{3} \sqrt{1+2\cos\alpha}.
\end{equation}
Considering first the planes perpendicular to the direction
${\bf l}=(1,0,0)$ we have that the distance between planes is
$d_{\bf l}=a_h$ and $\alpha_{\bf l}=\alpha_1$. Therefore
\begin{equation}
\alpha_1 = {1\over a_h(T)} {d a_h \over dT}.
\end{equation}
Then we can consider the planes perpendicular to the direction
${\bf l}=(0,0,1)$ where the distance between planes is
$d_{\bf l}=c_h/3$ and $\alpha_{\bf l}=\alpha_3$. Therefore
\begin{equation}
\alpha_3 = {1\over c_h(T)} {dc_h \over d T}.
\end{equation}
Choosing as parameters $a$ and $\cos \alpha$ we must use the relationships:
\begin{equation}
{d a_h \over d T}= {a_h \over a} {d a \over d T} - {a_h \over 2(1-\cos\alpha)}{d \cos\alpha \over d T},
\end{equation}
\begin{equation}
{d c_h \over d T}= {c_h \over a} {d a \over d T} + {c_h \over 1+2\cos\alpha}{d \cos\alpha \over d T},
\end{equation}
to obtain
\begin{eqnarray}
\alpha_1 &=& \alpha_2= {1\over a(T)} {d a \over dT} - {1 \over 2(1-\cos\alpha)}{d \cos\alpha \over d T}, \\
\alpha_3 &=& {1\over a(T)} {d a \over dT} + {1 \over 1+2\cos\alpha}
{d \cos\alpha \over d T}.
\end{eqnarray}

\subsection{\color{web-blue}Orthorhombic systems}
In the orthorhombic system $k=3$, $\zeta_1=a$, $\zeta_2=b/a$, 
$\zeta_3=c/a$. The nonzero components of the thermal expansion are 
$\alpha_{1}$, $\alpha_{2}$, and $\alpha_{3}$. 
Reasoning as in the previous cases and choosing as directions
${\bf l}=(1,0,0)$, ${\bf l}=(0,1,0)$ and ${\bf l}=(0,0,1)$ we find:  
\begin{eqnarray}
\alpha_{1} &=&  {1\over a(T) } {d a \over dT}, \\ 
\alpha_{2} &=&  {1\over b(T) } {d b \over dT}, \\
\alpha_{3} &=&  {1\over c(T) } {d c \over dT}.
\end{eqnarray}
which is valid for the simple orthorhombic lattice as well as for the
base-centered, the face-centered, and the body-centered orthorhombic lattices.
Choosing $a$, $b/a$, and $c/a$ as parameters we find:
\begin{eqnarray}
\alpha_1 &=& {1\over a(T)} {d a \over dT}, \\
\alpha_2 &=& {1\over a(T)} {da \over d T}  + {1\over b/a(T)} {d b/a \over dT}, \\
\alpha_3 &=& {1\over a(T)} {da \over d T}  + {1\over c/a(T)} {d c/a \over dT}.
\end{eqnarray}

%\subsection{\color{web-blue}Monoclinic systems}
%In the monoclinic system $k=4$, $\zeta_1=a$, $\zeta_2=b$, $\zeta_3=c$, 
%$\zeta_4=\cos\gamma$ (c-unique) of $\zeta_4=\cos\beta$ (b-unique). 
%For the c-unique cell it is simple to define 
%\begin{equation}
%\alpha_{3,3}= {1\over c } {d c \over dT}
%\end{equation}
%For the other two.

%\subsection{\color{web-blue}Triclinic systems}

\newpage
\section{\color{coral}The heat capacity}
The heat capacity measures the variation of the temperature of the solid
when a certain quantity of heat is given to it at constant strain:
\begin{equation}
C_\epsilon= T {d S \over dT}\Big|_\epsilon = { d U \over d T}\Big|_\epsilon.
\end{equation}
Since only the entropy of the ionic displacements is taken into consideration,
the only term that is derived is $U_{ph}$. Once the phonon dispersions 
have been calculated this quantity can be obtained by deriving Eq.~\ref{uvib}.

Note that for metals the electronic contribution, while not large, is important 
and must be accounted for before comparing with experiment. Presently,
this term is calculated in \tpw\ for a model of independent electrons with
the same density of states (DOS) of the DFT bands. This contribution is 
calculated for a metal together with the DOS (see below).

\newpage
\section{\color{coral}The elastic constants}

When a crystal is deformed with a given strain $\epsilon_i$ forces
appear to bring it to the equilibrium configuration. These
forces can be described by the stress tensor. For small strain the 
stress tensor is proportional to the strain and we have:
\begin{equation}
\sigma_{i} =\sum_{il} C_{i,l} \epsilon_l 
\end{equation}
$C_{i,l}$ is the tensor of elastic constants. 
At zero temperature we can write:
\begin{equation}
C_{i,l}= {\partial \sigma_i \over \partial \epsilon_{l}}=
{1 \over V}{\partial^2 U \over \partial \epsilon_i \partial \epsilon_l}
.
\end{equation}
The nonzero components of the elastic constant tensor for each crystal point
group can be derived using group theory. The form of the elastic constants
tensor depends on the Laue class obtained by adding an inversion center to
the group operations. From this relationship we find that for finite
and small strains the total energy can be written as:
\begin{equation}
U= {V \over 2} \sum_{i,l} \epsilon_i C_{i,l} \epsilon_l.
\end{equation}
By choosing the form of the strain $\epsilon$ and fitting the
energy with a quadratic function we can calculate the elastic constants.
Calculating the stress tensor we can also use the definition
of elastic constants and fit linearly the stress as a function of strain. \\
The procedure depends on the Laue class of the solid. We discuss separately
the different Bravais lattices below.

\subsection{\color{web-blue}Cubic system}

The nonzero components of the elastic constants tensor for the cubic system
are independent on the Laue class and are $C_{11}=C_{22}=C_{33}$,
$C_{12}=C_{13}=C_{23}$ and $C_{44}=C_{55}=C_{66}$. [1]
Applying a strain $\epsilon_E$ the cell becomes tetragonal and we have:
\begin{equation}
\sigma_{1} = C_{12} \epsilon_3,
\end{equation}
\begin{equation}
\sigma_{3} = C_{11} \epsilon_3.
\end{equation}
To calculate $C_{44}$ we can apply $\epsilon_F$ that
transforms the cubic cell into a rhombohedral cell as described above. In
this case we obtain
\begin{equation}
\sigma_{4} = C_{44} \epsilon_4,
\end{equation}
and $\sigma_{5} = \sigma_{6} = \sigma_{4}$. \\
In order to calculate the elastic constant using only the energy
we can consider a strain $\epsilon_A$ that gives:
\begin{equation}
U_A = {3 V \over 2} (C_{11} + 2C_{12}) \epsilon_1^2.
\end{equation}
The energy for a strain $\epsilon_E$ gives
\begin{equation}
U_E = {V \over 2} C_{11} \epsilon_3^2,
\end{equation}
and from these two relationships we can obtain both $C_{11}$ and $C_{12}$.
Finally to calculate $C_{44}$, we can apply a strain $\epsilon_F$ 
which gives:
\begin{equation}
U_F = {3 V \over 2} C_{44} \epsilon_4^2.
\end{equation}
By fitting the energy with a quadratic function for these three cases,
we can calculate the elastic constants as:
\begin{eqnarray}
C_{11}&=&{1 \over V} {d^2 U_E \over d \epsilon_3^2}, \\ 
C_{12}&=&{1 \over 6 V} {d^2 U_A \over d \epsilon_1^2}  - {1\over 2} C_{11}, \\
C_{44}&=&{1 \over 3 V} {d^2 U_F \over d \epsilon_4^2}.
\end{eqnarray}
but the use of the energy instead of stress requires three fits instead
of two. 

\subsection{\color{web-blue}Tetragonal system}
In this case there are two possibilities depending on the Laue class.
For the point groups $C_{4v} (4mm)$, $D_{2d} (\bar42m)$, $D_4 (422)$ and
$D_{4h} (4/mmm)$ (Laue class $D_{4h} (4/mmm)$) the nonzero components of the 
elastic constants tensor are
$C_{11}=C_{22}$, $C_{12}$, $C_{13}=C_{23}$, $C_{33}$, $C_{44}=C_{55}$, 
$C_{66}$.
For the point groups $C_{4} (4)$, $S_{4} (\bar4)$, $C_{4h} (4/m)$ (Laue class
$C_{4h} (4/m)$) in addition to the components of the elastic constants tensor
discussed above also the $C_{16}=-C_{26}$ components are non zero.

By doing a strain $\epsilon_E$ that conserves the shape of the tetragonal
cell we obtain
\begin{equation}
\sigma_{1} = C_{13} \epsilon_3,
\end{equation}
\begin{equation}
\sigma_{3} = C_{33} \epsilon_3,
\end{equation}
and we obtain $C_{13}$ and $C_{33}$.
To obtain the other components of the tensor we must change the shape of the 
unit cell. $\epsilon_C$ transforms the tetragonal cell in an orthorhombic cell 
and gives:
\begin{equation}
\sigma_{1} = C_{11} \epsilon_1, 
\end{equation}
\begin{equation}
\sigma_{2} = C_{12} \epsilon_1.
\end{equation}
Applying $\epsilon_G$ we obtain:
\begin{equation}
\sigma_{6} = C_{66} \epsilon_6.
\end{equation}
Finally applying the strain $\epsilon_H$ we obtain
\begin{equation}
\sigma_{5} = C_{55} \epsilon_5,
\end{equation}
that gives $C_{55}$. \\
For the Laue class $C_{4h} (4/m)$ the strain $\epsilon_G$ gives also 
\begin{equation}
\sigma_{1} = C_{16} \epsilon_6.
\end{equation}

To use the energy as a function of strain 
we need the following strains and relationships. 
By applying $\epsilon_C$ we get:
\begin{equation}
U_C= {V \over 2} C_{11} \epsilon_1^2.
\end{equation}
By applying $\epsilon_E$ we get:
\begin{equation}
U_E= {V \over 2} C_{33} \epsilon_3^2.
\end{equation}
By applying $\epsilon_B$ we get
\begin{equation}
U_B= {2 V \over 2} (C_{11} + C_{12}) \epsilon_1^2.
\end{equation}
By applying $\epsilon_{B'}$ we get
\begin{equation}
U_{B'}= {V \over 2} (C_{11} + C_{33} + 2 C_{13}) \epsilon_1^2.
\end{equation}
These strains transform the tetragonal into an orthorhombic cell 
and allow us to get four elastic constants. 
To obtain the last two elastic constants we 
must apply $\epsilon_G$ and we get:
\begin{equation}
U_G= {V \over 2} C_{66} \epsilon_6^2,
\end{equation}
applying $\epsilon_H$ we get:
\begin{equation}
U_H= {V \over 2} C_{55} \epsilon_5^2.
\end{equation}
Solving these equations we obtain:
\begin{eqnarray}
C_{11}&=&{1 \over V} {d^2 U_C \over d \epsilon_1^2}, \\ 
C_{33}&=&{1 \over V} {d^2 U_E \over d \epsilon_3^2}, \\ 
C_{12}&=&{1 \over 2 V} {d^2 U_B \over d \epsilon_1^2} - C_{11}, \\
C_{13}&=&{1 \over 2 V} {d^2 U_{B'} \over d \epsilon_1^2} - 
{(C_{11} +C_{33})\over 2}, \\
C_{44}&=&{1 \over V} {d^2 U_G \over d \epsilon_4^2}, \\
C_{66}&=&{1 \over V} {d^2 U_H \over d \epsilon_6^2}.
\end{eqnarray}
For the groups that require $C_{16}$ we must apply simultaneously $\epsilon_C$
and $\epsilon_G$ which give:
\begin{equation}
U_{C+G}={V \over 2} (C_{11}+2C_{16}+C_{66})\epsilon^2,
\end{equation}
from which we obtain:
\begin{equation}
C_{16}={1 \over 2V} {d^2 U_{C+G} \over d \epsilon^2} - {(C_{11}+C_{66})\over 2}.
\end{equation}

\subsection{\color{web-blue}Orthorhombic system}
In this system all solids belong to the $D_{2h} (mmm)$ Laue class so
the non zero elements of the elastic constant tensor are the same for all 
point groups and are
$C_{11}$, $C_{12}$, $C_{13}$, $C_{22}$, $C_{23}$,
$C_{33}$, $C_{44}$, $C_{55}$, and $C_{66}$. \\
Applying $\epsilon_C$ we obtain
\begin{eqnarray}
\sigma_1&=&C_{11} \epsilon_1,  \\
\sigma_2&=&C_{12} \epsilon_1,  \\
\sigma_3&=&C_{13} \epsilon_1.
\end{eqnarray}
Applying $\epsilon_D$ we obtain
\begin{eqnarray}
\sigma_2&=&C_{22} \epsilon_2,  \\
\sigma_3&=&C_{23} \epsilon_2.
\end{eqnarray}
Finally applying $\epsilon_E$ we obtain
\begin{equation}
\sigma_3=C_{33} \epsilon_3.
\end{equation}
To obtain the other components of the elastic constant tensor,
we have to modify the shape of the orthorhombic cell. \\
Applying $\epsilon_G$ we have
\begin{equation}
\sigma_6=C_{66} \epsilon_6.
\end{equation}
Applying $\epsilon_H$ we have
\begin{equation}
\sigma_5=C_{55} \epsilon_5.
\end{equation}
Finally applying $\epsilon_I$ we have
\begin{equation}
\sigma_4=C_{44} \epsilon_4.
\end{equation}

To use the energy as a function of strain we need the following strains
and relationships. 
By applying $\epsilon_C$ we get:
\begin{equation}
U_C= {V \over 2} C_{11} \epsilon_1^2.
\end{equation}
By applying $\epsilon_D$ we get:
\begin{equation}
U_D= {V \over 2} C_{22} \epsilon_2^2.
\end{equation}
By applying $\epsilon_E$ we get:
\begin{equation}
U_E= {V \over 2} C_{33} \epsilon_3^2.
\end{equation}
By applying $\epsilon_B$ we get:
\begin{equation}
U_B= {V \over 2} (C_{11} + C_{22} + 2 C_{12}) \epsilon_1^2.
\end{equation}
By applying $\epsilon_{B'}$ we get: 
\begin{equation}
U_{B'}= {V \over 2} (C_{11} + C_{33} + 2 C_{13}) \epsilon_1^2.
\end{equation}
By applying $\epsilon_{B''}$ we get:
\begin{equation}
U_{B''}= {V \over 2} (C_{22} + C_{33} + 2 C_{23}) \epsilon_2^2.
\end{equation}
These strains conserve the orthorhombic cell and allow us to get
six elastic constants. To obtain the last three elastic constants we 
apply $\epsilon_G$ to obtain
\begin{equation}
U_G= {V \over 2} C_{66} \epsilon_6^2,
\end{equation}
$\epsilon_H$ to obtain
\begin{equation}
U_H= {V \over 2} C_{55} \epsilon_5^2,
\end{equation}
and $\epsilon_I$ to obtain
\begin{equation}
U_I= {V \over 2} C_{44} \epsilon_4^2.
\end{equation}
Solving these equations we have:
\begin{eqnarray}
C_{11}&=&{1 \over V} {d^2 U_C \over d \epsilon_1^2}, \\ 
C_{22}&=&{1 \over V} {d^2 U_D \over d \epsilon_2^2}, \\ 
C_{33}&=&{1 \over V} {d^2 U_E \over d \epsilon_3^2}, \\ 
C_{12}&=&{1 \over 2 V} {d^2 U_B \over d \epsilon_1^2} - 
{(C_{11}+C_{22})\over 2}, \\
C_{13}&=&{1 \over 2 V} {d^2 U_{B'} \over d \epsilon_1^2} - 
{(C_{11}+C_{33})\over 2}, \\
C_{23}&=&{1 \over 2 V} {d^2 U_{B''} \over d \epsilon_2^2} - 
{(C_{22}+C_{33})\over 2}, \\
C_{44}&=&{1 \over V} {d^2 U_I \over d \epsilon_4^2}, \\
C_{55}&=&{1 \over V} {d^2 U_H \over d \epsilon_5^2}, \\
C_{66}&=&{1 \over V} {d^2 U_G \over d \epsilon_6^2}.
\end{eqnarray}

\subsection{\color{web-blue}Monoclinic system}
The non zero components of the elastic constants tensor depend on the choice
of the unit cell $b$-unique or $c$-unique. 
$C_{11}$, $C_{12}$, $C_{13}$, $C_{22}$, $C_{23}$, $C_{33}$, $C_{44}$, 
$C_{55}$, and $C_{66}$ are always different from zero, while for the
$b$-unique cell also the $C_{15}$, $C_{25}$, $C_{35}$, and $C_{46}$ components
are different from zero. For the $c$-unique cell, instead, $C_{16}$, $C_{26}$, 
$C_{36}$, and $C_{45}$ are different from zero. \\
Applying $\epsilon_C$ we obtain
\begin{eqnarray}
\sigma_1&=&C_{11} \epsilon_1,  \\
\sigma_2&=&C_{12} \epsilon_1,  \\
\sigma_3&=&C_{13} \epsilon_1.  
\end{eqnarray}
Moreover
\begin{eqnarray}
\sigma_5&=&C_{15} \epsilon_1,
\end{eqnarray}
for the $b$-unique cell and 
\begin{eqnarray}
\sigma_6&=&C_{16} \epsilon_1,
\end{eqnarray}
for the $c$-unique cell. \\
Applying $\epsilon_D$ we obtain
\begin{eqnarray}
\sigma_2&=&C_{22} \epsilon_2,  \\
\sigma_3&=&C_{23} \epsilon_2.  
\end{eqnarray}
Moreover
\begin{eqnarray}
\sigma_5&=&C_{25} \epsilon_2,
\end{eqnarray}
for the $b$-unique cell and 
\begin{eqnarray}
\sigma_6&=&C_{26} \epsilon_2,
\end{eqnarray}
for the $c$-unique cell. \\
Finally applying $\epsilon_E$ we obtain:
\begin{eqnarray}
\sigma_3&=&C_{33} \epsilon_3.
\end{eqnarray}
Moreover 
\begin{eqnarray}
\sigma_5&=&C_{35} \epsilon_3,
\end{eqnarray}
for the $b$-unique cell and 
\begin{eqnarray}
\sigma_6&=&C_{36} \epsilon_3,
\end{eqnarray}
for the $c$-unique cell. \\
Applying $\epsilon_I$ we obtain
\begin{eqnarray}
\sigma_4&=&C_{44} \epsilon_4.
\end{eqnarray}
Moreover
\begin{eqnarray}
\sigma_6&=&C_{46} \epsilon_4,
\end{eqnarray}
for the $b$-unique cell and
\begin{eqnarray}
\sigma_5&=&C_{45} \epsilon_4,
\end{eqnarray}
for the $c$-unique cell. \\
Applying $\epsilon_H$ we obtain:
\begin{eqnarray}
\sigma_5&=&C_{55} \epsilon_5,
\end{eqnarray}
and applying $\epsilon_G$ we obtain
\begin{eqnarray}
\sigma_6&=&C_{66} \epsilon_6.
\end{eqnarray}

\subsection{\color{web-blue}Triclinic system}
In this case all the components of the elastic constant tensors are non zero.
We can apply $\epsilon_C$ and obtain:
\begin{eqnarray}
\sigma_1&=&C_{11} \epsilon_1,  \\
\sigma_2&=&C_{12} \epsilon_1,  \\
\sigma_3&=&C_{13} \epsilon_1,  \\
\sigma_4&=&C_{14} \epsilon_1,  \\
\sigma_5&=&C_{15} \epsilon_1,  \\
\sigma_6&=&C_{16} \epsilon_1,  \\
\end{eqnarray}
where we have used the symmetry of the elastic constants matrix.
Applying $\epsilon_D$ we obtain:
\begin{eqnarray}
\sigma_2&=&C_{22} \epsilon_2,  \\
\sigma_3&=&C_{23} \epsilon_2,  \\
\sigma_4&=&C_{24} \epsilon_2,  \\
\sigma_5&=&C_{25} \epsilon_2,  \\
\sigma_6&=&C_{26} \epsilon_2.  \\
\end{eqnarray}
Applying $\epsilon_E$ we obtain:
\begin{eqnarray}
\sigma_3&=&C_{33} \epsilon_3,  \\
\sigma_4&=&C_{34} \epsilon_3,  \\
\sigma_5&=&C_{35} \epsilon_3,  \\
\sigma_6&=&C_{36} \epsilon_3.  \\
\end{eqnarray}
Applying $\epsilon_I$ we obtain:
\begin{eqnarray}
\sigma_4&=&C_{44} \epsilon_4,  \\
\sigma_5&=&C_{45} \epsilon_4,  \\
\sigma_6&=&C_{46} \epsilon_4.  \\
\end{eqnarray}
Applying $\epsilon_H$ we obtain:
\begin{eqnarray}
\sigma_5&=&C_{55} \epsilon_5,  \\
\sigma_6&=&C_{56} \epsilon_5.  \\
\end{eqnarray}
Applying $\epsilon_G$ we obtain:
\begin{eqnarray}
\sigma_6&=&C_{66} \epsilon_6.  \\
\end{eqnarray}
Clearly in all cases, independent of the lattice, these equations can be
applied to calculate the elastic constants, neglecting any symmetry
information, but the number of calculations is larger than that needed
when symmetry is used.

For using the energy in the calculation we use the following
relationships. By applying $\epsilon_C$ we get:
\begin{equation}
U_C= {V \over 2} C_{11} \epsilon_1^2.
\end{equation}
By applying $\epsilon_D$ we get:
\begin{equation}
U_D= {V \over 2} C_{22} \epsilon_2^2.
\end{equation}
By applying $\epsilon_E$ we get:
\begin{equation}
U_E= {V \over 2} C_{33} \epsilon_3^2.
\end{equation}
By applying $\epsilon_B$ we get:
\begin{equation}
U_B= {V \over 2} (C_{11} + C_{22} + 2 C_{12}) \epsilon_1^2.
\end{equation}
By applying $\epsilon_{B'}$ in which only $\epsilon_1$ and
$\epsilon_3=\epsilon_1$ are non zero, we get:
\begin{equation}
U_{B'}= {V \over 2} (C_{11} + C_{33} + 2 C_{13}) \epsilon_1^2.
\end{equation}
By applying $\epsilon_{B''}$ in which only $\epsilon_2$ and
$\epsilon_3=\epsilon_2$ are non zero, we get:
\begin{equation}
U_{B''}= {V \over 2} (C_{22} + C_{33} + 2 C_{23}) \epsilon_2^2,
\end{equation}
and in this way we get six elastic constant. Continuing as in the
orthorhombic case we can
apply $\epsilon_G$ and we get:
\begin{equation}
U_G= {V \over 2} C_{66} \epsilon_6^2.
\end{equation}
By applying $\epsilon_H$ we obtain:
\begin{equation}
U_H= {V \over 2} C_{55} \epsilon_5^2.
\end{equation}
By applying $\epsilon_I$ we obtain:
\begin{equation}
U_I= {V \over 2} C_{44} \epsilon_4^2.
\end{equation}
The triclinic case twelve components. Applying simultaneously two strains
we can get combinations of other elastic constants with those already
calculated. For instance applying $\epsilon_C + \epsilon_G$ and taking
$\epsilon_1=\epsilon_6=\epsilon$ we obtain:
\begin{equation}
U_{C+G}= {V \over 2} (C_{11}+2 C_{16} + C_{66}) \epsilon^2.
\end{equation}
Similarly applying $\epsilon_C + \epsilon_H$ and taking $\epsilon_1=
\epsilon_5=\epsilon$ we obtain:
\begin{equation}
U_{C+H}= {V \over 2} (C_{11}+2 C_{15} + C_{55}) \epsilon^2.
\end{equation}
Similarly applying $\epsilon_C + \epsilon_I$ and taking
$\epsilon_1=\epsilon_4=\epsilon$ we obtain:
\begin{equation}
U_{C+I}= {V \over 2} (C_{11}+2 C_{14} + C_{44}) \epsilon^2.
\end{equation}
Then we can apply $\epsilon_D + \epsilon_G$ taking
$\epsilon_2=\epsilon_6=\epsilon$ and we obtain:
\begin{equation}
U_{D+G}= {V \over 2} (C_{22}+2 C_{26} + C_{66}) \epsilon^2.
\end{equation}
Similarly applying $\epsilon_D + \epsilon_H$ taking
$\epsilon_2=\epsilon_5=\epsilon$ and we obtain:
\begin{equation}
U_{D+H}= {V \over 2} (C_{22}+2 C_{25} + C_{55}) \epsilon^2.
\end{equation}
Similarly applying $\epsilon_D + \epsilon_I$ taking
$\epsilon_2=\epsilon_4=\epsilon$ and we obtain:
\begin{equation}
U_{D+I}= {V \over 2} (C_{22}+2 C_{24} + C_{44}) \epsilon^2.
\end{equation}
Then we can apply $\epsilon_E + \epsilon_G$ taking
$\epsilon_3=\epsilon_6=\epsilon$ and we obtain:
\begin{equation}
U_{E+G}= {V \over 2} (C_{33}+2 C_{36} + C_{66}) \epsilon^2.
\end{equation}
Similarly applying $\epsilon_E + \epsilon_H$ taking
$\epsilon_3=\epsilon_5=\epsilon$ and we obtain:
\begin{equation}
U_{E+H}= {V \over 2} (C_{33}+2 C_{35} + C_{55}) \epsilon^2,
\end{equation}
and applying $\epsilon_E + \epsilon_I$ taking
$\epsilon_3=\epsilon_4=\epsilon$ and we obtain:
\begin{equation}
U_{E+I}= {V \over 2} (C_{33}+2 C_{34} + C_{44}) \epsilon^2.
\end{equation}
In order to calculate the last three components we need to apply
simultaneously two shear strains. Applying $\epsilon_G + \epsilon_H$ 
and taking $\epsilon_6=\epsilon_5=\epsilon$ we obtain:
\begin{equation}
U_{G+H}= {V \over 2} (C_{55}+2 C_{56} + C_{66}) \epsilon^2.
\end{equation}
Similarly applying $\epsilon_G + \epsilon_I$
and taking $\epsilon_6=\epsilon_4=\epsilon$
\begin{equation}
U_{G+I}= {V \over 2} (C_{44}+2 C_{46} + C_{66}) \epsilon^2.
\end{equation}
Finally applying $\epsilon_H + \epsilon_I$
and taking $\epsilon_5=\epsilon_4=\epsilon$
\begin{equation}
U_{H+I}= {V \over 2} (C_{44}+2 C_{45} + C_{55}) \epsilon^2.
\end{equation}
Solving these equations we obtain:
\begin{eqnarray}
C_{11}&=&{1 \over V} {d^2 U_C \over d \epsilon_1^2}, \\ 
C_{22}&=&{1 \over V} {d^2 U_D \over d \epsilon_2^2}, \\ 
C_{33}&=&{1 \over V} {d^2 U_E \over d \epsilon_3^2}, \\ 
C_{12}&=&{1 \over 2 V} {d^2 U_B \over d \epsilon_1^2} - 
                             {(C_{11}+ C_{22})\over 2}, \\
C_{13}&=&{1 \over 2 V} {d^2 U_{B'} \over d \epsilon_1^2} - 
                             {(C_{11} + C_{33})\over 2}, \\
C_{23}&=&{1 \over 2 V} {d^2 U_{B''} \over d \epsilon_2^2} - 
                             {(C_{22} - C_{33})\over 2}, \\
C_{44}&=&{1 \over 3 V} {d^2 U_I \over d \epsilon_4^2}, \\
C_{55}&=&{1 \over 3 V} {d^2 U_H \over d \epsilon_5^2}, \\
C_{66}&=&{1 \over 3 V} {d^2 U_G \over d \epsilon_6^2}, \\
C_{14}&=&{1 \over 2 V} {d^2 U_{C+I} \over d \epsilon^2} - {(C_{11} + C_{44})\over 2}, \\
C_{15}&=&{1 \over 2 V} {d^2 U_{C+H} \over d \epsilon^2} - {(C_{11} + C_{55})\over 2}, \\
C_{16}&=&{1 \over 2 V} {d^2 U_{C+G} \over d \epsilon^2} - {(C_{11} + C_{66})\over 2}, \\
C_{24}&=&{1 \over 2 V} {d^2 U_{D+I} \over d \epsilon^2} - {(C_{22} + C_{44})\over 2}, \\
C_{25}&=&{1 \over 2 V} {d^2 U_{D+H} \over d \epsilon^2} - {(C_{22} + C_{55})\over 2}, \\
C_{26}&=&{1 \over 2 V} {d^2 U_{D+G} \over d \epsilon^2} - {(C_{22} + C_{66})\over 2}, \\
C_{34}&=&{1 \over 2 V} {d^2 U_{E+I} \over d \epsilon^2} - {(C_{33} + C_{44})\over 2}, \\
C_{35}&=&{1 \over 2 V} {d^2 U_{E+H} \over d \epsilon^2} - {(C_{33} + C_{55})\over 2}, \\
C_{36}&=&{1 \over 2 V} {d^2 U_{E+G} \over d \epsilon^2} - {(C_{33} + C_{66})\over 2}, \\
C_{45}&=&{1 \over 2 V} {d^2 U_{H+I} \over d \epsilon^2} - {(C_{44} + C_{55})\over 2}, \\
C_{46}&=&{1 \over 2 V} {d^2 U_{G+I} \over d \epsilon^2} - {(C_{44} + C_{66})\over 2}, \\
C_{56}&=&{1 \over 2 V} {d^2 U_{G+H} \over d \epsilon^2} - {(C_{55} + C_{66})\over 2}.
\end{eqnarray}

\subsection{\color{web-blue}Hexagonal system}
In this case the form of the elastic constants tensor is independent of
the Laue class. The nonzero components are $C_{11}=C_{22}$, $C_{12}$, 
$C_{13}=C_{23}$, $C_{33}$, $C_{44}=C_{55}$, $C_{66}={1\over 2} 
(C_{11}-C_{12})$. 
$C_{11}$, $C_{12}$ and $C_{13}$ can be obtained by doing a strain $\epsilon_C$,
transforming the hexagonal cell into a base-centered orthorhombic
cell. We have:
\begin{eqnarray}
\sigma_1&=&C_{11} \epsilon_1,  \\
\sigma_2&=&C_{12} \epsilon_1,  \\
\sigma_3&=&C_{13} \epsilon_1.
\end{eqnarray}
The strain $\epsilon_E$ conserves the shape of the hexagonal cell and
gives $C_{33}$:
\begin{eqnarray}
\sigma_3&=&C_{33} \epsilon_3.  
\end{eqnarray}
In order to obtain $C_{44}=C_{55}$ we can apply a strain $\epsilon_H$ 
that transforms the hexagonal cell into a base-centered monoclinic cell.
We have
\begin{eqnarray}
\sigma_5&=&C_{55} \epsilon_5. 
\end{eqnarray}

To use the energy as a function of strain 
we need the following strains and relationships. \\
By applying $\epsilon_C$, we get $C_{11}$:
\begin{equation}
U_C= {V \over 2} C_{11} \epsilon_1^2.
\end{equation}
By applying $\epsilon_E$, we get $C_{33}$:
\begin{equation}
U_E= {V \over 2} C_{33} \epsilon_3^2.
\end{equation}
By applying $\epsilon_{B'}$ in which only $\epsilon_1$ and 
$\epsilon_3=\epsilon_1$ are non zero, we have:
\begin{equation}
U_{B'}= {V \over 2} (C_{11} + C_{33} + 2 C_{13}) \epsilon_1^2,
\end{equation}
and using the previously calculated elastic constants, we can obtain $C_{13}$.
By applying $\epsilon_A$, we get
\begin{equation}
U_A= {V \over 2} (2C_{11} + C_{33} + 2 C_{12} + 4 C_{13}) \epsilon_1^2,
\end{equation}
and using the previously calculated elastic constants, we can obtain $C_{12}$.
Finally applying $\epsilon_H$, we obtain $C_{55}$:
\begin{equation}
U_H= {V \over 2} C_{55} \epsilon_5^2.
\end{equation}
Solving these equations we obtain:
\begin{eqnarray}
C_{11}&=&{1 \over V} {d^2 U_C \over d \epsilon_1^2}, \\ 
C_{33}&=&{1 \over V} {d^2 U_E \over d \epsilon_3^2}, \\ 
C_{13}&=&{1 \over 2 V} {d^2 U_{B'} \over d \epsilon_1^2} - 
{(C_{11} + C_{33})\over 2}, \\
C_{12}&=&{1 \over 2 V} {d^2 U_A \over d \epsilon_1^2} - C_{11} - 
{C_{33} \over 2} - 2 C_{13}, \\
C_{55}&=&{1 \over V} {d^2 U_I \over d \epsilon_5^2}. \\
\end{eqnarray}

\subsection{\color{web-blue}Trigonal system}
In this case there are two possibilities depending on the Laue class. For
the point groups $D_3(32)$, $C_{3v}(3m)$ and $D_{3d}(\bar 3m)$ (Laue class
$D_{3d}(\bar 3m)$) the nonzero components of the elastic constant tensor
are $C_{11}=C_{22}$, $C_{12}$, $C_{13}=C_{23}$, $C_{33}$, $C_{44}=C_{55}$, 
$C_{66}={1\over 2} (C_{11}-C_{12})$, $C_{14}$, $C_{24}=-C_{14}$, and 
$C_{56}=C_{14}$. For the point groups $C_3(3)$ and $S_6(\bar 3)$ (Laue
class $S_6(\bar 3)$) there are other three nonzero components $C_{15}$, 
$C_{25}=-C_{15}$ and $C_{46}=-C_{15}$. 
By doing a strain $\epsilon_C$ we obtain:
\begin{equation}
\sigma_1=C_{11} \epsilon_1,
\end{equation}
\begin{equation}
\sigma_2=C_{12} \epsilon_1,
\end{equation}
\begin{equation}
\sigma_3=C_{13} \epsilon_1,
\end{equation}
\begin{equation}
\sigma_4=C_{14} \epsilon_1.
\end{equation}
Applying $\epsilon_E$ we have:
\begin{equation}
\sigma_3=C_{33} \epsilon_3,
\end{equation}
Finally, applying $\epsilon_G$ we have:
\begin{equation}
\sigma_4=C_{44} \epsilon_4,
\end{equation}
For the class $S_6(\bar 3)$ we have also:
\begin{equation}
\sigma_5=C_{15} \epsilon_1.
\end{equation}


\newpage
\section{\color{coral}Pressure dependent elastic constants}

The elastic constants of a solid on which we apply a pressure $p$ can still
be defined from the linear relationship between applied strain and 
induced stress: 
\begin{equation}
\sigma_{ij} =\sigma^{(0)}_{ij}+\sum_{kl} B_{ij,kl} \epsilon_{kl}.
\end{equation}
However since the solid is already in a strained state
to balance the pressure $p$, the elastic constants defined as
\begin{equation}
C_{ij,kl}= 
{1 \over V}{\partial^2 U \over \partial \epsilon_{ij} \partial \epsilon_{kl}},
\end{equation}
do not coincide with $B_{ij,kl}$. Actually there is a simple 
relationship between the two:
\begin{equation}
B_{ij,kl} = C_{ij,kl} + {1\over 2} p (2 \delta_{ij} \delta_{kl} - 
 \delta_{il} \delta_{jk} - \delta_{ik} \delta_{jl} ).
\end{equation}
Since experimental measurements of elastic constants are usually done 
using the stress-strain relationship it is common to give the numerical
values of $B_{ij,kl}$ and to call them the elastic constants of a
solid under pressure. We can write out the relationship for the Voigt 
components:
\begin{eqnarray}
B_{44} &=& C_{44} - {1\over 2} p, \\
B_{55} &=& C_{55} - {1\over 2} p, \\
B_{66} &=& C_{66} - {1\over 2} p, \\
B_{12} &=& C_{12} + p, \\
B_{13} &=& C_{13} + p, \\
B_{23} &=& C_{23} + p, 
\end{eqnarray}
while for all the other components $B_{ij}=C_{ij}$.
Note however that for a generic stress, as opposed to 
an isotropic pressure, the $B_{ij,kl}$ have not the same symmetry properties 
of $C_{ij,kl}$. In general they cannot be written using the Voigt notation.

%\section{\color{coral}Temperature dependent elastic constants}
%In principle to calculate the elastic constants at each temperature it
%is sufficient to substitute the Helmholtz free energy to the total energy
%in the equations of the previous section. There are however two complications
%that make the calculation more difficult. One is that the stress is not
%readily available at finite temperatures and therefore we can use only the
%energy expressions. The second is that, at each temperature, the strain must be
%applied taking as equilibrium (unstrained) geometry the one that 
%minimizes the free energy at temperature $T$ and the volume that appears in
%the equations should be the equilibrium one at temperature $T$.
%We have:
%\begin{equation}
%C_{i,l}^T= {1 \over V(T)}{\partial^2 F \over \partial \epsilon_i \partial \epsilon_l}
%\Bigg|_T.
%\end{equation}
%Since we cannot make an elastic constant calculation for each temperature, it
%is convenient to calculate the Helmholtz free energy in a regular grid in
%a space whose dimension are the parameters $\zeta_k$ that define the 
%unstrained lattice plus an additional parameter that define the strain necessary
%to calculate the given elastic constant. Sometimes this strain can be applied
%simply changing the parameters $\zeta_k$, but in some cases it is an
%additional parameter that increases of one the dimension of 
%the parameter space, or equivalently lowers the symmetry of the cell so
%that we need more parameters to determine it (however all the additional
%parameters are a function of the strain amplitude).
%The free energy is then fitted by a quadratic polynomial in this extended space
%at each temperature and the minimum of the quadratic polynomial gives
%the equilibrium geometry at that temperature. In principle for a solid
%that is stable at that temperature the minimum is obtained at
%zero strain. Then the same free energy values are fitted with a quartic
%polynomial and the second derivatives of the free energy with respect
%to strain are calculated at the minimum.
%It is important to use a fourth-order polynomial to calculate the
%second derivatives. A second order polynomial is sufficient to determine
%the minimum but not the second derivatives. Under pressure instead of the
%Helmholtz free energy we minimize the Gibbs free energy. The second
%derivatives of the Gibbs free energy gives:
%\begin{equation}
%B_{i,l}^T= {1 \over V(T)}{\partial^2 G \over \partial \epsilon_i \partial \epsilon_l}
%\Bigg|_T.
%\end{equation}
%This equation is sufficient to calculate the elastic constant at a 
%given pressure and temperature, but not at a given generic stress. In the
%latter case $B_{i,l}^T$ cannot be written using the Voigt notation.
%
%Let us consider separately the different Bravais lattices.
%
%\subsection{\color{web-blue}Cubic system}
%The application of $\epsilon_A$ is equivalent to the change of the lattice 
%constant $a$, so the calculation of $C_{11}+2 C_{12}$ can be done by fitting
%$F(a)$ with the quadratic function
%$F(a) = g_1 + g_3 (a-a_0)^2$ where $a_0$ is the parameter $a$ that minimizes
%the free energy at each temperature. Since
%\begin{equation}
%F(\epsilon_A)= {3 V(T) \over 2} \epsilon_1^2 (C_{11}+2C_{12}) 
%\end{equation}
%and $\epsilon_1=(a-a_0)/a_0$ we have:
%\begin{equation}
%C^T_{11}+2C^T_{12}= {1 \over 3 V(T)}  a_0^2 2 g_3
%\end{equation}
%
%Let us now consider $\epsilon_E$. To apply this strain we have to transform
%the cubic cell in a tetragonal cell. At $T=0$ it was sufficient to change only
%the parameter $c/a$ keeping $a$ fixed to calculate $C_{11}$, now instead we
%have to vary both $a$ and $c/a$. Fitting the free energy
%$F(a, \epsilon_3) = g_1 + g_3 (a-a_0)^2 + g_5 \epsilon_3^2 
%+ g_6 (a-a_0) \epsilon_3$.
%At the minimum $a_0$ and $\epsilon_{30}$ we should have
%$F(\epsilon_3) = {V\over 2} C_{11} \epsilon_3^2$.
%Since the strain $\epsilon_E$ does not change $a$ we have simply:
%\begin{equation}
%C^T_{11} = {1 \over V(T)} 2 g_5
%\end{equation}
%but $g_5$ must be obtained from a two-dimensional fit of the free energy
%as a function of both $a$ and $\epsilon_3$ at each temperature.
%
%Let us now consider $\epsilon_F$. To apply this strain we have to transform the
%unit cell in a trigonal cell and we must study a strain 
%$\epsilon_4=\epsilon_5=\epsilon_6=\epsilon$. In this case we must vary both 
%$a_t$ and $\cos\alpha$ of the trigonal cell.
%Using the relationships between trigonal and strained cubic cell we can write
%the free energy as:
%\begin{equation}
%F(a, \epsilon) = g_1 + g_3 (a-a_0)^2 + g_5 \epsilon^2 + g_6 (a-a_0) \epsilon.
%\end{equation}
%where $a$ is the size of the cubic cell and we already used the fact that
%the minimum should be obtained for $\epsilon=0$. We therefore compute the free energy
%in a uniform mesh of values of $a$ and $\epsilon$ and from the coefficients
%of the interpolating polynomial and the equation:
%\begin{equation}
%F(\epsilon) = { 3 V \over 2} C_{44} \epsilon^2
%\end{equation}
%we obtain 
%\begin{equation}
%C_{44}= { 1\over 3 V } 2 g_5 
%\end{equation}
%If the fit is made with $N_g$ free energy evaluations per parameter, the cost of
%this procedure is $N_g + 2 N_g^2$ free energy evaluations.
%
%\subsection{\color{web-blue}Hexagonal system}
%The application of $\epsilon_C$ means that we transform the hexagonal cell
%into a base-centered orthorhombic cell. This cell is defined by three 
%parameters $a_o$, $b_o/a_o$ and $c_o/a_o$. We can fit the Helmholtz free 
%energy with a quadratic function of three parameters $a$, $c/a$, and 
%$\epsilon_1$, where $a$ and $c/a$ are the parameters of the hexagonal
%cell as:
%\begin{eqnarray}
%F(a, c/a, \epsilon_1) = g_1 &+& g_3 (a-a_0)^2 + g_5 (c/a-c_0/a_0)^2
%+ g_6 (a-a_0) (c/a-c_0/a_0) \nonumber \\ &+& g_8 \epsilon_1^2 + g_9 (a-a_0) \epsilon_1
%+g_{10} (c/a-c_0/a_0) \epsilon_1, 
%\label{quadratic_3}
%\end{eqnarray}
%The conversion between the orthorhombic parameters and the hexagonal ones
%plus strain has been given above.
%At the minimum of the free energy $\epsilon_1=0$ while $a$ and $c/a$ have
%the equilibrium values at the temperature $T$. At the minimum 
%\begin{equation}
%F(\epsilon_1) = { V \over 2} C_{11} \epsilon_1^2
%\end{equation}
%so we obtain
%\begin{equation}
%C_{11}= { 1\over V } 2 g_8 
%\end{equation}
%
%In order to calculate $C_{33}$ at any temperature we can apply $\epsilon_E$.
%We keep the hexagonal cell and change $a$ and $c/a$ fitting the free energy 
%as a function of $a$ and $\epsilon_3$ with a second order polynomial. We can 
%write:
%\begin{equation}
%F(a, \epsilon_3) = g_1 + g_3 (a-a_0)^2 + 
%                         g_5 (\epsilon_3-\epsilon_{30})^2 + 
%                         g_6 (a-a_0) (\epsilon_3-\epsilon_{30})
%\end{equation}
%$a_0$ and $\epsilon_{30}$ will give the equilibrium configuration at
%each temperature. $\epsilon_{30}$ will depend on the reference configuration
%chosen to make the fit.
%At the minimum:
%\begin{equation}
%F(\epsilon_3) = { V \over 2} C_{33} (\epsilon_3-\epsilon_{30})^2
%\end{equation}
%so that 
%\begin{equation}
%C_{33}= { 1\over V } 2 g_5 
%\end{equation}
%
%Then we can calculate the combination $C_{11}+C_{33}+2C_{13}$ by repeating
%the procedure seen above for the strain $\epsilon_{B'}$ and writing the
%orthorhombic parameters in terms of $a$, $c/a$ and $\epsilon_1$. This time
%the functional dependence of the orthorhombic parameters will be the
%one appropriate to the strain $\epsilon_{B'}$, and we get:
%\begin{equation}
%C_{11}+C_{33}+2C_{13}= { 1\over V } 2 g_8 
%\end{equation}
%
%Then we calculate the combination $2C_{11}+C_{33}+2C_{12}+4C_{13}$ by 
%applying a strain $\epsilon_A$. We can write the free energy as a function
%of $a$ and $\epsilon_1$:
%\begin{equation}
%F(a, \epsilon_1) = g_1 + g_3 (a-a_0)^2 + 
%                         g_5 (\epsilon_1-\epsilon_{10})^2 + 
%                         g_6 (a-a_0) (\epsilon_1-\epsilon_{10})
%\end{equation}
%and at the minimum $a_0$ and $\epsilon_1$ gives the values of $a$ and $c/a$
%that correspond to the given temperature and at the minimum:
%\begin{equation}
%2C_{11}+C_{33}+2C_{12}+4C_{13}= { 1\over V } 2 g_5 
%\end{equation}
%
%Finally we have to make a strain $\epsilon_H$ to determine $C_{55}$. Since
%now both $a$ and $c/a$ might vary, we link the four parameters of the 
%monoclinic lattice with $a$, $c/a$ and $\epsilon_5$. A quadratic fit of
%this function of three parameters similar to Eq.~\ref{quadratic_3}
%will give:
%\begin{equation}
%C_{55}= { 1\over V } 2 g_8 
%\end{equation}
%
%\section{\color{coral}Pressure and temperature dependent elastic constants}
%
%Under pressure instead of using the Helmholtz free energy we can minimize 
%the Gibbs free energy. The second
%derivatives of the Gibbs free energy are the elastic constants:
%\begin{equation}
%B_{i,l}^T= {1 \over V(T)}{\partial^2 G \over \partial 
%\epsilon_i \partial \epsilon_l}
%\Bigg|_T.
%\end{equation}
%This equation is sufficient to calculate the elastic constant at a
%given pressure and temperature, but not at a given generic stress. In the
%latter case $B_{i,l}^T$ cannot be written using the Voigt notation.
%In general if the interval of parameters used to explore the free energy
%surface is sufficiently large we can fit the Gibbs free energy
%with a second order polynomial and find the minimum at any pressure
%without additional energy or phonon calculations. However the quadratic
%fit is sufficient to determine the new minimum for a range of pressures,
%but not to determine the second derivatives. To have sufficient accuracy 
%the Gibbs free energy must be fitted with a quartic
%polynomial and the second derivatives of the free energy with respect
%to strain of this polynomial must be calculated at the minimum.
%The range of pressures for which this approach works can be tested changing
%the starting configuration.

\newpage
\section{\color{coral}The elastic compliances}
The relationship between stress and strain can be inverted. A stress 
applied to a solid produces a strain proportional to the stress:
\begin{equation}
\epsilon_{i} =\sum_{il} S_{i,l} \sigma_l, 
\end{equation}
where the tensor $S_{i,l}$ is known as elastic compliances. It is the
inverse of the elastic constant tensor: 
\begin{equation}
S_{i,l} = C_{i,l}^{-1}.
\end{equation}


\newpage
\section{\color{coral}Bulk modulus and compressibility}

The isothermal compressibility is defined as the derivative of the 
equilibrium volume with respect to pressure divided by the volume,
calculated at constant temperature.
\begin{equation}
K_T = - {1 \over V} {d V \over d p}\Bigg|_T.
\end{equation}
The bulk modulus is the inverse of the compressibility. 

In order to calculate it we should calculate the change of volume
due to an external pressure $p$. Since the change of volume can be
written as 
\begin{equation}
dV = V (\epsilon_1 + \epsilon_2 + \epsilon_3)
\end{equation}
and we can calculate $\epsilon$ from the elastic compliances and the
stress $\sigma_1=-p$, $\sigma_2=-p$, $\sigma_3=-p$ we have
\begin{equation}
B_T = {1 \over S_{1,1}+S_{2,2}+S_{3,3}+2 S_{1,2}+2 S_{1,3}+2 S_{2,3}}.
\end{equation}
%In some cases it might be useful to express the bulk modulus in
%terms of the elastic constants, for instance when not all the 
%elastic constants are available and it is not possible to calculate 
%the elastic compliances. Moreover it is interesting to express
%the bulk modulus in terms of the derivatives of the free energy with
%respect to the crystal parameters $\zeta_k$.
%This depends on the Laue class and on the Bravais lattice, so we discuss
%this in the following.

\newpage
\section{\color{coral}Thermal expansion from Gr\"uneisen parameters}
Thermal expansion can be calculated from the numerical derivatives of the 
phonon frequencies with respect to strain, doing analytically the
derivative with respect to the temperature.
From the definition of thermal expansion:
\begin{equation}
\alpha_i= \left( {d \epsilon_i \over d T}\right)_{\sigma},
\end{equation}
we can write:
\begin{equation}
\alpha_i= - \sum_j \left({d \epsilon_i \over d \sigma_j}\right)_{T} 
\left({d \sigma_j \over d T} \right)_{\epsilon} = - \sum_j S_{i,j}
\left({d \sigma_j \over d T} \right)_{\epsilon}.
\end{equation}
Using the fact that 
\begin{equation}
\left({d \sigma_j \over d T} \right)_{\epsilon} = {1\over V} {d \over d T} \left( {d F \over d \epsilon_j}\right),
\end{equation}
where $F$ is the free energy, we obtain 
\begin{equation}
\left({d \sigma_j \over d T} \right)_{\epsilon} = -\sum_{{\bf q},\nu}
\gamma_{{\bf q},\nu}^j c_{{\bf q},\nu},
\end{equation}
where
\begin{equation}
\gamma_{{\bf q},\nu}^j = -{1\over \omega_{{\bf q},\nu}} {\partial
\omega_{{\bf q},\nu} \over \partial \epsilon_j} 
\end{equation}
are the modes Gr\"uneisen parameters and 
\begin{equation}
c_{{\bf q},\nu} = {\hbar \omega_{{\bf q},\nu} \over V} {\partial
\over \partial T } \left[{1 \over e^{\beta \hbar \omega_{{\bf q},\nu}} - 1}
\right]
\end{equation}
are the mode contribution to the heat capacity.
The final expression of the thermal expansion is therefore:
\begin{equation}
\alpha_i = \sum_j S_{i,j} \sum_{{\bf q},\nu}
\gamma_{{\bf q},\nu}^j c_{{\bf q},\nu}.
\end{equation}
An accurate evaluation of this sum would require the temperature dependent
elastic compliances and the Gr\"uneisen parameters should be calculated
at the equilibrium volume at each temperature, but an approximate expression 
can be obtained neglecting this dependence and using the $T=0$ values
of the volume and of the compliances. 

%Again we have the problem that the phonon frequencies and hence their
%derivatives can be calculated only with respect to the crystal parameters,
%while we need the derivatives with respect to the strain. We discuss
%this point for each crystal system.
%
%\subsection{\color{web-blue}Cubic system}
%Assuming that the phonon frequencies ${\omega}_{{\bf q}\nu}$ are a function 
%of the set of strains $\{\epsilon_i\}$, it seems that the calculation of 
%the derivative of the frequency with respect to one strain component 
%$\epsilon_i$ require the simulation of a noncubic cell.  
%However we can consider in strain space, a six-dimensional space, the line 
%$\epsilon_1=\epsilon_2=\epsilon_3=\epsilon$ and 
%$\epsilon_4=\epsilon_5=\epsilon_6=0$ and consider the derivative of phonon
%frequencies along this line. Moving along this line means to keep the cell
%cubic and change the lattice constant $a$ so that $d\epsilon={da \over a}$
%therefore:
%\begin{equation}
%{d \omega_{{\bf q}\nu} \over d \epsilon} =a{d \omega_{{\bf q}\nu} \over d a},
%\end{equation}
%but along the line we have:
%\begin{equation}
%{d \omega_{{\bf q}\nu} \over d \epsilon} = {\partial \omega_{{\bf q}\nu} 
%\over \partial \epsilon_1} + {\partial \omega_{{\bf q}\nu} 
%\over \partial \epsilon_2} + {\partial \omega_{{\bf q}\nu} 
%\over \partial \epsilon_3} 
%\end{equation}
%Now using the fact that in a cubic system the three partial derivatives 
%are equal we have:
%\begin{equation}
%{\partial \omega_{{\bf q}\nu} 
%\over \partial \epsilon_1} = {a \over 3} {d \omega_{{\bf q}\nu} \over d a}.
%\end{equation}
%
%\subsection{\color{web-blue}Tetragonal and hexagonal systems}
%In these systems we can consider two lines in strain space. 
%The line $\epsilon_1=\epsilon_2=\epsilon_3=\epsilon$, and
%$\epsilon_4=\epsilon_5=\epsilon_6=0$ that means a change of $a$ at fixed
%$c/a$, and the line $\epsilon_3=\epsilon$ and 
%$\epsilon_1=\epsilon_2=\epsilon_4=\epsilon_5 =\epsilon_6=0$ that means 
%a change of $c/a$ at fixed $a$.
%Along the first line $d\epsilon={d a \over a}$, while along the second
%$d\epsilon={d c/a \over c/a}$.
%Along the first line we have:
%\begin{equation}
%{d \omega_{{\bf q}\nu} \over d \epsilon} = {\partial \omega_{{\bf q}\nu} 
%\over \partial \epsilon_1} + {\partial \omega_{{\bf q}\nu} 
%\over \partial \epsilon_2} + {\partial \omega_{{\bf q}\nu} 
%\over \partial \epsilon_3} = a {d \omega_{{\bf q}\nu} 
%\over d a},
%\end{equation}
%while along the second
%\begin{equation}
%{d \omega_{{\bf q}\nu} \over d \epsilon} = {\partial \omega_{{\bf q}\nu} 
%\over \partial \epsilon_3} = {c\over a} {d \omega_{{\bf q}\nu} 
%\over d c/a}.
%\end{equation}
%From the second equation we obtain ${\partial \omega_{{\bf q}\nu} 
%\over \partial \epsilon_3}$ that inserted in the first equation gives
%\begin{equation}
%2 {\partial \omega_{{\bf q}\nu} \over \partial \epsilon_1} = a  
%{d \omega_{{\bf q}\nu} \over d a} -
%{c\over a} {d \omega_{{\bf q}\nu} \over d c/a}
%\end{equation}
%where we used the fact that in tetragonal and hexagonal systems
%${\partial \omega_{{\bf q}\nu} \over \partial \epsilon_1}
%={\partial \omega_{{\bf q}\nu} \over \partial \epsilon_2}$.
% 
%\subsection{\color{web-blue}Trigonal system}
%We can use the analogy with the hexagonal system and the relationship
%between $a_h$ and $x=c_h/a_h$ and the parameters $a$ and $\cos \alpha$ of the
%trigonal system. 
%We have
%\begin{equation}
%a={a_h \over 3} \sqrt{3 + x^2}
%\end{equation}
%and
%\begin{equation}
%\cos\alpha={ 2 x^2 - 3 \over 6 + 2 x^2}.
%\end{equation}
%Using the relationship found at previous point for the hexagonal system:
%\begin{equation}
%{\partial \omega_{{\bf q}\nu} 
%\over \partial \epsilon_3} = x {d \omega_{{\bf q}\nu} 
%\over d x}
%\end{equation}
%and
%\begin{equation}
%{\partial \omega_{{\bf q}\nu} 
%\over \partial \epsilon_1} = 
%{1\over 2} \left(a_h {\partial \omega_{{\bf q}\nu} \over \partial a_h} - 
%{\partial \omega_{{\bf q}\nu} \over \partial 
%\epsilon_3} \right)
%\end{equation}
%and transforming to the trigonal variables using ${\partial a \over \partial
%a_h}={a\over a_h}$, ${\partial \cos \alpha \over \partial
%a_h}=0$, ${\partial a \over \partial x}={a_h^2 x \over 9 a}$ and
%${\partial \cos \alpha \over \partial
%x}={ a_h^4 x \over 9 a^4}$ we obtain:
%\begin{equation}
%{\partial \omega_{{\bf q}\nu} 
%\over \partial \epsilon_3} = {1\over 3} (1 + 2 \cos \alpha)
%\left[ a {\partial \omega_{{\bf q}\nu} \over \partial a} +
%2 (1-\cos\alpha) {\partial \omega_{{\bf q}\nu} \over \partial \cos \alpha}
%\right],
%\end{equation}
%and
%\begin{equation}
%{\partial \omega_{{\bf q}\nu} \over \partial \epsilon_1} = 
%{1\over 2} \left(a {\partial \omega_{{\bf q}\nu} \over \partial a} - {\partial 
%\omega_{{\bf q}\nu} \over \partial 
%\epsilon_3} \right) 
%\end{equation}
%
%\subsection{\color{web-blue}Orthorhombic system}
%In this system we can consider the three lines 
%$\epsilon_1=\epsilon_2=\epsilon_3=\epsilon$ and
%$\epsilon_4=\epsilon_5 =\epsilon_6=0$ obtained varying $a$ at fixed
%$b/a$ and $c/a$, 
%$\epsilon_2=\epsilon$ and
%$\epsilon_1=\epsilon_3=\epsilon_4=\epsilon_5 =\epsilon_6=0$, 
%obtained varying $b/a$ at fixed $a$ and $c/a$,
%and 
%$\epsilon_3=\epsilon$ and
%$\epsilon_1=\epsilon_2=\epsilon_4=\epsilon_5 =\epsilon_6=0$
%obtained varying $c/a$ at fixed $a$ and $b/a$.
%Along these three lines we have $d\epsilon={da \over a}$, 
%$d\epsilon={db \over b}$, and $d\epsilon={dc \over c}$, respectively.
%Along the first line we have
%\begin{equation}
%{d \omega_{{\bf q}\nu} \over d \epsilon} = {\partial \omega_{{\bf q}\nu} 
%\over \partial \epsilon_1} + {\partial \omega_{{\bf q}\nu} 
%\over \partial \epsilon_2} + {\partial \omega_{{\bf q}\nu} 
%\over \partial \epsilon_3} = a {d \omega_{{\bf q}\nu} 
%\over d a},
%\end{equation}
%along the second
%\begin{equation}
%{d \omega_{{\bf q}\nu} \over d \epsilon} = {\partial \omega_{{\bf q}\nu} 
%\over \partial \epsilon_2} = {b\over a} {d \omega_{{\bf q}\nu} 
%\over d b/a},
%\end{equation}
%and along the third
%\begin{equation}
%{d \omega_{{\bf q}\nu} \over d \epsilon} = {\partial \omega_{{\bf q}\nu} 
%\over \partial \epsilon_3} = {c\over a} {d \omega_{{\bf q}\nu} 
%\over d c/a}.
%\end{equation}
%Using the second and the third equation to calculate 
%${\partial \omega_{{\bf q}\nu} \over \partial \epsilon_2}$ and
%${\partial \omega_{{\bf q}\nu} \over \partial \epsilon_3}$, we can
%substitute in the first equation
%\begin{equation}
%{\partial \omega_{{\bf q}\nu} \over \partial \epsilon_1} 
% = a {d \omega_{{\bf q}\nu} \over da} - {b\over a} {d \omega_{{\bf q}\nu} 
%\over d b/a} - {c\over a} {d \omega_{{\bf q}\nu} 
%\over d c/a}
%\end{equation}
%
\newpage
\section{\color{coral}Average Gr\"uneisen parameter}

The product of the thermal expansion coefficient $\beta$, the bulk modulus
$B_T$, and the volume of unit cell $V$, divided by the constant volume
heat capacity $C_V$ is the average Gr\"uneisen parameter $\gamma$
that we can calculate as:
\begin{equation}
\gamma = {V B_T \beta \over C_V},
\end{equation}
and this can be calculated even for anisotropic solids if we have
calculated the bulk modulus as written above.

For anisotropic solids we can generalize this concept by introducing
the thermal stresses, as products of the elastic constants and the
thermal expansion:
\begin{equation}
b_{ij} = -\sum_{lm} C_{ij,lm} \alpha_{lm}.
\end{equation}
Multiplying this quantity by the ratio between the volume and the
constant strain heat capacity we obtain the generalized Gr\"uneisen
parameters:
\begin{equation}
\gamma_{ij} =  {V b_{ij} \over C_\epsilon}.
\end{equation}

\newpage
\section{\color{coral}Constant stress heat capacity}

The difference between the heat capacity calculated at constant stress
and at constant strain is given by:
\begin{equation}
C^\sigma - C^\epsilon = - T V \sum_{ij} \alpha_{ij} b_{ij}.
\end{equation}

\newpage
\section{\color{coral}Adiabatic quantities}

Similarly the difference between adiabatic and isothermal 
elastic compliances is given by (Voigt notation):
\begin{equation}
S^S_{i,l} - S^T_{i,l} = - {T V \alpha_{i} \alpha_{l} \over C^\sigma},
\end{equation}
while the difference between adiabatic and isothermal elastic constants
is given by
\begin{equation}
C^S_{ij,lm} - C^T_{ij,lm} =  {T V b_{ij} b_{lm} \over C^\epsilon}.
\end{equation}

\newpage

\section{\color{coral}Finite temperature electronic excitation energy}
The electronic excitation energy due to the finite temperature $U_{el}$ is
neglected in semiconductors and insulators. For metals it gives a finite
contribution to the heat capacity that we estimate by a model of independent 
electrons whose energy-wave vector dispersion is given by the band structure. 
We assume Fermi-Dirac occupations of the available electronic levels:
\begin{equation}
f(E,T,\mu) = \tilde \theta({\mu - E \over k_B T}),
\end{equation}
where $\mu$ is the chemical potential and the function $\tilde \theta(x)$ is
\begin{equation}
\tilde \theta(x) = {1 \over 1 + e^{-x}}.
\end{equation}
Given the density of electronic states $N(E)$,
we determine the chemical potential $\mu(T)$ at each temperature
from the condition
\begin{equation}
N_{el} = \int_{-\infty}^\infty N(E) f(E,T,\mu) dE,
\end{equation}
where $N_{el}$ is the number of electrons per unit cell.
The electronic excitation energy $U_{el}$ is given by
\begin{equation}
U_{el}=\int_{-\infty}^\infty E N(E) f(E,T,\mu) dE - 
\int_{-\infty}^{E_F} E N(E) dE,
\end{equation}
where $E_F$ is the Fermi energy.
The electronic entropy is given:
\begin{equation}
S_{el}=-k_B\int_{-\infty}^\infty \left[ f(E,T,\mu) \ln f(E,T,\mu) 
+ (1 - f(E,T,\mu)) \ln (1 - f(E,T,\mu)) \right] N(E) dE.
\end{equation}
The electron Helmholtz free-energy is calculated by 
$F_{el}=U_{el}-TS_{el}$.
Finally the constant strain electronic specific heat is estimated by:
\begin{equation}
C_{\eta,el}=\int_{-\infty}^\infty (E - \mu) N(E) 
{\partial f(E,T,\mu) \over \partial T} dE,  
\end{equation}
where the partial derivative with respect to the temperature:
\begin{equation}
{\partial f(E,T,\mu) \over \partial T} = {E -\mu \over k_B T^2} \ \tilde 
\theta'({\mu - E \over k_B T}),
\end{equation}
indicates that, in computing the derivative, we neglect the temperature 
dependence of the chemical potential, and
\begin{equation}
\tilde \theta'(x) = {1 \over 1 + e^x+ e^{-x}}.
\end{equation}

\newpage
\section{\color{coral}Appendix}
It is often useful to express the primitive vectors of the centered
lattices in term of the primitive vectors of the conventional lattices
and viceversa. We report in this appendix the relationships that result
from the definitions of the primitive vectors given above.
We indicate with a prime the primitive vectors of the centered lattices. \\
Face-centered cubic lattice:
\begin{eqnarray}
{\bf a}_1'&=&{1\over 2} (-{\bf a}_1 + {\bf a}_3), \\
{\bf a}_2'&=&{1\over 2} ({\bf a}_2 + {\bf a}_3), \\
{\bf a}_3'&=&{1\over 2} (-{\bf a}_1 + {\bf a}_2). 
\end{eqnarray}

\begin{eqnarray}
{\bf a}_1&=& -{\bf a}_1' + {\bf a}_2'- {\bf a}_3', \\
{\bf a}_2&=& -{\bf a}_1' + {\bf a}_2'+ {\bf a}_3', \\
{\bf a}_3&=& {\bf a}_1' + {\bf a}_2'- {\bf a}_3' .
\end{eqnarray}
Body-centered cubic lattice:
\begin{eqnarray}
{\bf a}_1'&=&{1\over 2} ({\bf a}_1 + {\bf a}_2 + {\bf a}_3), \\
{\bf a}_2'&=&{1\over 2} (-{\bf a}_1 + {\bf a}_2 + {\bf a}_3), \\
{\bf a}_3'&=&{1\over 2} (-{\bf a}_1 - {\bf a}_2 + {\bf a}_3). 
\end{eqnarray}

\begin{eqnarray}
{\bf a}_1&=&  {\bf a}_1' - {\bf a}_2', \\
{\bf a}_2&=&  {\bf a}_2' - {\bf a}_3', \\
{\bf a}_3&=& {\bf a}_1' + {\bf a}_3'.
\end{eqnarray}
Centered tetragonal:
\begin{eqnarray}
{\bf a}_1'&=&{1\over 2} ({\bf a}_1 - {\bf a}_2 + {\bf a}_3), \\
{\bf a}_2'&=&{1\over 2} ({\bf a}_1 + {\bf a}_2 + {\bf a}_3), \\
{\bf a}_3'&=&{1\over 2} (-{\bf a}_1 - {\bf a}_2 + {\bf a}_3). 
\end{eqnarray}

\begin{eqnarray}
{\bf a}_1&=&  {\bf a}_1' - {\bf a}_3', \\
{\bf a}_2&=& -{\bf a}_1' + {\bf a}_2', \\
{\bf a}_3&=& {\bf a}_2' + {\bf a}_3'.
\end{eqnarray}
Base-centered orthorhombic:
\begin{eqnarray}
{\bf a}_1'&=&{1\over 2} ({\bf a}_1 + {\bf a}_2), \\
{\bf a}_2'&=&{1\over 2} (-{\bf a}_1 + {\bf a}_2), \\
{\bf a}_3'&=&{\bf a}_3. 
\end{eqnarray}

\begin{eqnarray}
{\bf a}_1&=&  {\bf a}_1' - {\bf a}_2', \\
{\bf a}_2&=&  {\bf a}_1' + {\bf a}_2', \\
{\bf a}_3&=& {\bf a}_3'.
\end{eqnarray}
Face-centered orthorhombic:
\begin{eqnarray}
{\bf a}_1'&=&{1\over 2} ({\bf a}_1 + {\bf a}_3), \\
{\bf a}_2'&=&{1\over 2} ({\bf a}_1 + {\bf a}_2), \\
{\bf a}_3'&=&{1\over 2} ({\bf a}_2 + {\bf a}_3), \\
\end{eqnarray}

\begin{eqnarray}
{\bf a}_1&=&  {\bf a}_1' + {\bf a}_2' - {\bf a}_3', \\
{\bf a}_2&=& -{\bf a}_1' + {\bf a}_2' + {\bf a}_3', \\
{\bf a}_3&=&  {\bf a}_1' - {\bf a}_2' + {\bf a}_3'.
\end{eqnarray}
Body-centered orthorhombic:
\begin{eqnarray}
{\bf a}_1'&=&{1\over 2} ({\bf a}_1 + {\bf a}_2 + {\bf a}_3), \\
{\bf a}_2'&=&{1\over 2} (-{\bf a}_1 + {\bf a}_2 + {\bf a}_3), \\
{\bf a}_3'&=&{1\over 2} (-{\bf a}_1 - {\bf a}_2 + {\bf a}_3). 
\end{eqnarray}

\begin{eqnarray}
{\bf a}_1&=&  {\bf a}_1' - {\bf a}_2', \\
{\bf a}_2&=&  {\bf a}_2' - {\bf a}_3', \\
{\bf a}_3&=& {\bf a}_1' + {\bf a}_3'.
\end{eqnarray}
Base-centered monoclinic ($b$-unique):
\begin{eqnarray}
{\bf a}_1'&=&{1\over 2} ({\bf a}_1 - {\bf a}_2), \\
{\bf a}_2'&=&{1\over 2} ({\bf a}_1 + {\bf a}_2), \\
{\bf a}_3'&=&{\bf a}_3. 
\end{eqnarray}

\begin{eqnarray}
{\bf a}_1&=&  {\bf a}_1' + {\bf a}_2', \\
{\bf a}_2&=& -{\bf a}_1' + {\bf a}_2', \\
{\bf a}_3&=& {\bf a}_3'.
\end{eqnarray}
Base-centered monoclinic ($c$-unique):
\begin{eqnarray}
{\bf a}_1'&=&{1\over 2} ({\bf a}_1 - {\bf a}_3), \\
{\bf a}_2'&=&{\bf a}_2, \\
{\bf a}_3'&=&{1\over 2} ({\bf a}_1 + {\bf a}_3). 
\end{eqnarray}

\begin{eqnarray}
{\bf a}_1&=& {\bf a}_1' + {\bf a}_3', \\
{\bf a}_2&=& {\bf a}_2', \\
{\bf a}_3&=& -{\bf a}_1' + {\bf a}_3'.
\end{eqnarray}

Rombohedral cell (in term of the hexagonal cell):
\begin{eqnarray}
{\bf a}_1'&=&{1\over 3} ({\bf a}_1 - {\bf a}_2 + {\bf a}_3), \\
{\bf a}_2'&=&{1\over 3} ({\bf a}_1 + 2 {\bf a}_2 + {\bf a}_3), \\
{\bf a}_3'&=&{1\over 3} (-2 {\bf a}_1 - {\bf a}_2 + {\bf a}_3). 
\end{eqnarray}

\begin{eqnarray}
{\bf a}_1&=& {\bf a}_1' - {\bf a}_3', \\
{\bf a}_2&=& - {\bf a}_1' + {\bf a}_2', \\
{\bf a}_3&=& {\bf a}_1' + {\bf a}_2' + {\bf a}_3'.
\end{eqnarray}

\newpage

\section{\color{coral}Bibliography}
\begin{enumerate}

\item
J.F. Nye, Physical properties of crystals, Oxford University press, 
Oxford (1957)

\item
D.C. Wallace, Thermodynamics of crystals, J. Wiley and Sons, (1972),
Dover edition (1998).

\end{enumerate}

\end{document}
