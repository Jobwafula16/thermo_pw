\documentclass[12pt,a4paper]{article}
\def\version{0.1.0}
\def\qe{{\sc Quantum ESPRESSO}}

\usepackage{html}

\usepackage{graphicx}

\textwidth = 17cm
\textheight = 24cm
\topmargin =-1 cm
\oddsidemargin = 0 cm

\def\pwx{\texttt{pw.x}}
\def\phx{\texttt{ph.x}}
\def\configure{\texttt{configure}}
\def\PWscf{\texttt{PWscf}}
\def\PHonon{\texttt{PHonon}}
\def\thermo{\texttt{thermo\_pw}}
\def\make{\texttt{make}}

\begin{document} 
\author{}
\date{}

\def\qeImage{../../Doc/quantum_espresso.pdf}
\def\democritosImage{../../Doc/democritos.pdf}
\def\SissaImage{./sissa.pdf}

\title{
%  \includegraphics[width=6cm]{\SissaImage}\\
  \vskip 1cm
  \Huge User's Guide for the \thermo\ package \smallskip
  \Large (version \version)
}

\maketitle

\tableofcontents

\section{Introduction}

This guide covers the usage and installation of the \thermo\ package. 
It assumes that you know the contents of the general User's Guide for \qe\ 
and of the User's Guide for \PWscf\ and for \PHonon. 
If not, please locate the general User's Guide in directory 
\texttt{Doc/} of your \qe\ distribution,
the User's Guide for \PWscf\ in \texttt{PW/Doc/} and the User's Guide
for \PHonon\ in \texttt{PHonon/Doc/}, or consult the web site:
\texttt{http://www.quantum-espresso.org}.

\thermo\ is a set of drivers to compute material properties. 
The same calculations can be done by using the \qe\ package, but they would 
require several steps. These tasks are usually carried out by \texttt{bash}
scripts. \thermo\ allows to avoid the writing of complex \texttt{bash}
scripts for the most common material properties. It can be considered as a set 
of scripts written in \texttt{FORTRAN}. At low level \thermo\ uses the \qe\ 
routines and writes the same files. If needed, these files could be read
by the codes of the \qe\ package. 

\thermo\ has the following directory structure, contained in a subdirectory 
\texttt{THERMO\_PW/} of the main \qe\ tree:

\begin{tabular}{ll}
\texttt{Doc/}      & : contains the user\_guide and input data description \\
\texttt{examples/} & : some running examples \\
\texttt{lib/}      & : source files for modules used by \thermo\ calculations \\
\texttt{src/}      & : source files for \thermo\ calculations \\
\end{tabular}\\

The \thermo\ package can perform the following types of calculations:
\begin{itemize}
\item Computation of the total energy.

\item Total energy as a function of the kinetic energy cut-off.

\item Total energy as a function of {\bf k}-points and smearing.

\item Band structure calculation at fixed geometry.

\item Phonon frequencies at fixed geometry.

\item Phonon dispersions at fixed geometry and computation of the harmonic
thermodynamical properties: vibrational energy, vibrational free energy,
vibrational entropy, and constant volume molar heat capacity as a function
of temperature.

\item Murnaghan fit of the total energy as function of the lattice parameter.

\item Band structure at the Murnaghan minimum.

\item Phonon frequencies at the Murnaghan minimum.

\item Phonon dispersions and harmonic vibrational thermodynamic quantities
at the Murnaghan minimum.

\item Phonon dispersions at several geometries and calculation of the
anharmonic vibrational properties: lattice parameter, bulk modulus,
pressure derivative of the bulk modulus as a function of temperature.
Gruneisen parameters. Thermal expansion as a function of temperature.
Isobaric molar heat capacity as a function of temperature.
Isoentropic bulk modulus as a function of temperature.
Average Gruneisen parameter as a function of temperature.

\item Frozen ions and total elastic constants.

\item Frozen ions and total piezoelectric tensor.
\end{itemize}

\thermo\ can run in parallel using all the parallel modes available in
the \qe\ package. On top of that \thermo\ can be run using several images.
The image parallelization is used when possible in an asyncronous way.
One processor takes the role of master and distributes the work 
to all the images that carry it out independently. Presently 
the total energies of several geometries for a Murnaghan fit are
calculated independently, and in parallel if there are several images.  
The phonon calculation is carried out in parallel, each image doing 
a single representation.

\section{People}
The \thermo\ code has been developed and is maintained by Andrea Dal Corso 
(SISSA). It is mainly an experimental tool to test new ideas for \qe\ 
organization. It is an open source code distributed, as it is, within the GPL 
licence.  

\section{Installing, Compiling, and Running}

\subsection{Installing}

\thermo\ is a package tightly bound to \qe. It cannot be compiled without
the \qe\ routines. For instruction on how to download and compile \qe, please 
refer to the general User's Guide, available in file \texttt{Doc/user\_guide.pdf}
under the main \qe\ directory, or in web site \\
\texttt{http://www.quantum-espresso.org}.

Once \qe\ is correctly configured, you need to download the \PHonon\ 
package. This can be done automatically just typing \texttt{make ph}, from 
the main \qe\ directory.

Then you need to create the directory \texttt{thermo\_pw} in the
main \qe\ directory. 
%If available for the version of QE that you are using you
%can download the file \texttt{thermo\_pw.tar.gz} and copy it in the
%main \qe\ directory. The command \texttt{tar -xzvf thermo\_pw.tar.gz}
%creates the \texttt{thermo\_pw} directory.
Presently \texttt{thermo\_pw} works only with the SVN version of \qe, 
it is not compatible with the current 5.0.3 version.
The \texttt{thermo\_pw} directory can be downloaded as described at
the web page: \texttt{http://www.qe-forge.org/gf/project/thermo\_pw/scmcvs/?}
\texttt{action=AccessInfo}.

\subsection{Compiling}

In order to compile \thermo\ the main \texttt{Makefile} and a few other
files of \qe\ need to be updated. Just enter in the \texttt{thermo\_pw}
directory and give the command \texttt{make join\_qe}. This command exchanges
the files contained in \texttt{thermo\_pw} with those of the \qe\ package.
Typing \texttt{make thermo\_pw} from the root \qe\ directory, or \texttt{make} 
from the \texttt{thermo\_pw}\ directory, produces the executable
\texttt{THERMO\_PW/src/thermo\_pw.x} that is linked in the 
\qe\ \texttt{bin/} directory. Note that you need to compile \texttt{pw.x},
\texttt{ph.x}, and \texttt{pp.x} before \texttt{thermo\_pw}.
\thermo\ has been written on a PC with Linux operating system using the
\texttt{gfortran} compiler with \texttt{openMPI} parallelization. Any other
combination of computer/operating system has not been tested.

\subsection{Uninstalling}

In order to remove \thermo\ from your \qe\ distribution, just enter in the
\texttt{thermo\_pw} directory and give the command \texttt{make leave\_qe}.
Then just remove the \texttt{thermo\_pw} directory.
   
\subsection{Running \thermo}

In order to use the \thermo\ code you need to create a file called
\texttt{thermo\_control} in your working directory in addition to copy
there the input of \texttt{pw.x} and, if requested by the task, an
input of \texttt{ph.x} that must be called \texttt{ph\_control}.
The input of \texttt{pw.x} can have any name and is given as input to
the \thermo\ code. It is not necessary to specify an \texttt{outdir} 
directory in the \texttt{ph.x} input.

A typical command for running the \thermo\ code could be:
\begin{verbatim}
mpirun -n np thermo_pw.x -ni ni ... < input_pw > output_thermo_pw
\end{verbatim}
where \texttt{np} is the number of processors and \texttt{ni} is the number 
of images. The dots indicate the other parallelization options that
you can find in the \qe\ manual.

One or more \texttt{postscript} files with plots of the material properties
are the output of the \thermo\ code. It produces also files with the data of 
the plot and scripts for the \texttt{gnuplot} program. 
Normally, the user does not need to modify these files, but they allow 
the improvement of the figures if needed.
The plot of the Brillouin zone (BZ) is done with the help of the 
\texttt{asymptote} code. \texttt{Thermo\_pw} produces a script of commands
for the \texttt{asymptote} code and can also run it to produce the pdf
file of the BZ.

The \thermo\ code has a minimal recovery feature. It does not recalculate
the quantities contained in files that are already in the working
directory. Each routine checks if a file with the same name
as the file that it would produce is already in the working directory,
and if this happens, it returns. Note that this feature cannot be 
disabled from input. In order to recalculate a given quantity, just remove
the file that contains it in the working directory.
The \thermo\ code cannot recover automatically a partial phonon calculation, all 
the dynamical matrices of a given geometry must be in the working directory to 
skip the calculation. If one is missing all the phonon calculations are repeated. 
In that case you might want to use the recover feature  of the \texttt{ph.x} code 
by specifying \texttt{recover=.true.} in the input of this code.

\section{Input variables}

The \texttt{pw.x} and \texttt{ph.x} input files are described in the documentation
mentioned above. In this section we discuss only the creation of the file
\texttt{thermo\_control}. This file contains a namelist:  
\begin{verbatim}
&INPUT_THERMO
  what=' ',
  ...
 /
\end{verbatim}
The \texttt{what} variable controls the sequence of calculations made
by \thermo. For each possible value of \texttt{what}, we discuss briefly the
input variables that you can use to control the output plots.

\thermo\ actually writes on file the data to plot and writes a script to plot
these data. The output postscript files are produced by invoking the 
\texttt{gnuplot} program within the code. Usually any modern Linux 
distribution provides a package to install this code, or has it already 
installed by default. If not, you can download it from 
\texttt{http://www.gnuplot.info/}. Three input variables of \thermo\ control 
the use of the \texttt{gnuplot} code:

\begin{verbatim}
lgnuplot    : if .TRUE. gnuplot is called from within the program
              and the postscript files are immediately available
              Default: logical .TRUE.
gnuplot_command  : the command used to call gnuplot.
              Default: character(len=*) 'gnuplot'
flgnuplot   : initial part of the name of the files where gnuplot scripts 
              are written.
              Default: character(len=*) 'gnuplot.tmp'
\end{verbatim}
If you are running \thermo\ on a system that has not \texttt{gnuplot}
you can disable the production of the postscript files and use some other
graphical tools to produce the plots from the output data. 


\subsection{\texttt{what='scf'}}
With this option the code computes only the total energy. This is a single
calculation as if running \texttt{pw.x} with the given input.
No other input variable is necessary.
An example of the use of this option can be found in \texttt{example01}.

\subsection{\texttt{what='scf\_ke'}}
With this option the code makes several self-consistent calculations, 
in parallel on several images, varying the kinetic energy cut-off for 
the wavefunctions and for the charge density. 
In the input of \texttt{pw.x} one specifies the minimum values for these two 
cut-offs. These values are then increased in fixed intervals controlled by the 
following variables. The energy is then plotted as a function of the 
wavefunctions kinetic energy cut-off, a different curve for each value of 
the charge density cut-off.

\begin{verbatim}
nke        : number of kinetic energies tested for the wavefunctions cut-off.
             Default: integer 5
deltake    : delta of wavefunctions kinetic energy cut-off in Ry.
             Default: real 10 Ry
nkeden     : number of kinetic energies tested for the charge density
             cut-off.
             Default: integer 1
deltakeden : delta of charge density kinetic energy cut-off in Ry.
             Default: real 100 Ry.
flkeconv   : name of the file where the data with the total energy as a
             function of the kinetic energy is written.
             Default: character(len=*) 'output_keconv.dat'
flpskeconv : name of the postscript file with the plot of the total energy as
             a function of the kinetic energy cut-off.
             Default: character(len=*) 'output_keconv.ps'
\end{verbatim}
An example of the use of this option can be found in \texttt{example10}.

\subsection{\texttt{what='scf\_nk'}}
With this option the code makes several self-consistent calculations, 
in parallel on several images, varying the size of the {\bf k}-point grid, 
and optionally for metals the smearing parameter \texttt{degauss}. In the 
input of \texttt{pw.x} the minimum value of these parameters is given and 
these values are increased in fixed intervals controlled by the following 
variables. On output the energy is plotted as a function of the mesh size, 
one curve for each smearing parameter.

\begin{verbatim}
nnk        : the number of different values of nk to test
             Default: integer 5
deltank    : the interval between nk values.
             Default: integer 2
nsigma     : the number of smearing intervals.
             Default: integer 1 
deltasigma : the distance between different smearing values.
             Default: 0.005 Ry
flnkconv   : file where the data with the k point convergence is written
             Default: character(len=*) 'output_nkconv.dat'
flpsnkconv : name of the postscript file with the k points convergence plot.
             Default: character(len=*) 'output_nkconv.ps'
\end{verbatim}
An example of the use of this option can be found in \texttt{example11}.

\subsection{\texttt{what='scf\_bands'}}
With this option the code makes a self-consistent calculation followed 
by a band structure calculation. The output of the band structure calculation
is further processed in order to produce a plot of the band structure.
A minimal control of the plot can be obtained with the following variables.
In insulators the zero of the bands is at the highest valence band in the
first {\bf k} point, in metals at the Fermi energy. 

\begin{verbatim}
emin_input : minimum energy for the band dispersion plot.
             Default: real minimum of the bands
emax_input : maximum energy for the band dispersion plot.
             Default: real maximum of the bands
nbnd_bands : the number of bands in the band calculation
             Default: integer nbnd given in pw.x input.
lsym       : if .TRUE. does the symmetry analysis of the bands
             Default: .TRUE.
\end{verbatim}

The path for the bands is determined automatically by the code on the basis
of the Bravais lattice index, but it can also be given at the end of
the \texttt{THERMO\_PW} namelist with the same format as in the \texttt{pw.x}
input. In the input of \texttt{pw.x} the {\bf k} points are those of the
self-consistent calculation. The following variables control how the path
is given:
\begin{verbatim}
q_in_band_form   : only the first and last point of each k path are given.
                 Default: .TRUE.
q_in_cryst_coord : the k - points are given in crystal coordinates.
                 Default: .FALSE.
point_label_type : the label definition (see the BZ manual)
                 Default: SC
\end{verbatim}
An example of the use of this option can be found in \texttt{example02}.
If you give explicitely the path, be careful with options that require
geometry changes (see below). Only automatic paths, or path given through 
letter labels are easily recalculated. The other paths could turn out 
to be correct only for one geometry.

\subsection{\texttt{what='plot\_bz'}}
With this option the code writes a script to make a three dimensional plot
of the Brillouin zone and of its default path. If the input contains a path
after the \texttt{THERMO\_PW} namelist this path is shown in the figure. 
The script must be read by the \texttt{asymptote} code, available at 
\texttt{http://asymptote.sourceforge.net/}. In many Linux distributions
this code is available as a separate package, but it is not installed by
default. The following variables control the plot:
\begin{verbatim}
lasymptote  : if .TRUE. asymptote is called from within the program
              and the pdf file with a plot of the BZ is produced
              Default: logical .FALSE.
flasy       : initial part of the name of the files where asymptore script
              is written, and of the name of the pdf file.
              Default: character(len=*) 'asy_tmp'
asymptote_command  : the command that invokes asymptote and produces the 
              pdf file of the BZ.
              Default: character(len=*) 'asy -f pdf -noprc flasy.asy'
npx         : maximum size of the supercell used to plot the Brillouin
              zone. Used only in the monoclinic case. If the code stops asking
              to double it increase the default until the error disappears.
              Default: integer 8
\end{verbatim}


\subsection{\texttt{what='scf\_ph'}}
With this option the code makes a phonon calculation at a single {\bf q} 
point or on a mesh of {\bf q} points, as specified in the input of 
the \texttt{ph.x} code after a self-consistent \texttt{pw.x} calculation. 
The different representations are calculated in parallel when several images 
are available. No other input variable is necessary. The outputs of this 
calculation are the dynamical matrices files.
An example of the use of this option can be found in \texttt{example03}.

\subsection{\texttt{what='scf\_disp'}}
With this option the code makes a phonon dispersion calculation after the 
self-consistent run at a fixed geometry. The geometry is given in the 
input of \texttt{pw.x}. The dynamical matrices are used to calculate 
the interatomic force constants and to interpolate the phonon dispersions 
along a path in the Brillouin zone. The path can be generated automatically 
on the basis of the Bravais lattice index or given in input as in a band 
structure calculation (see above \texttt{what='scf\_bands'}). The code then 
interpolates the phonon on a uniform mesh of {\bf q} points and computes the 
density of vibrational states with a smearing approach. The density of 
states is used to calculate the harmonic thermodynamical properties: 
vibrational energy, vibrational free energy, vibrational entropy, and 
isochoric heat capacity.
The plotted numerical values are per moles in an Avogadro number of unit 
cells. The same thermodynamical quantities are calculated also by direct 
integration over the Brillouin zone and compared in the plots.
The input variables that control this option are:
\begin{verbatim}
freqmin_input : minimum frequency for phonon dos plot.
                Default: real determined from phonon frequencies
freqmax_input : maximum frequency for phonon dos plot.
                Default: real determined from phonon frequencies
deltafreq     : frequency interval for phonon dos plot.
                Default: real 1 cm^{-1}
ndos          : number of frequency points in the dos plot.
                Default: determined from previous data
nq1_d, nq2_d, nq3_d : thick mesh for phonon dos calculation.
                Default: integer 16, 16, 16
zasr          : type of acoustic sum rule applied to the ifc.
                Default: character(len=*) 'Simple'
tmin          : minimum temperature.
                Default: real 1 K
tmax          : maximum temperature.
                Default: real 800 K
deltat        : interval between two temperatures. Be careful with this value
                because too small or too large values of this parameter could 
                give numerical errors in the temperature derivative used to 
                calculate anharmonic properties.
                Default: real 3 K
ntemp         : number of temperatures
                Default: integer determined from previous data

flfrc         : file where the interatomic force constants are written
                Default: character(len=*) 'output_frc.dat.g1'
flfrq         : file where matdyn writes the interpolated frequencies
                Default: character(len=*) 'output_frq.dat.g1'
fldos         : file where the phonon dos is written
                Default: character(len=*) 'output_dos.dat.g1'
fltherm       : file where the harmonic thermodynamic quantities are written
                Default: character(len=*) 'output_therm.dat.g1'
flpsband      : postscript file of the electronic band structure
                Default: character(len=*) 'output_band.ps'
flpsdisp      : postscript file of the phonon dispersions
                Default: character(len=*) 'output_disp.ps'
flpsdos       : postscript file of the phonon dos
                Default: character(len=*) 'output_dos.ps'
flpstherm     : postscript file of the harmonic thermodynamic quantities
                Default: character(len=*) 'output_therm.ps'
\end{verbatim}
An example of the use of this option can be found in \texttt{example04}.

\subsection{\texttt{what='mur\_lc'}}
With this option the code runs several self-consistent calculations
at different volumes. The runs can be done in parallel when several images 
are available. The total energy as a function of the volume is then 
interpolated with a Murnaghan equation and a plot of the energy
as a function of the volume and of the pressure as a function of the volume is
produced. Presently the volume is changed only by changing \texttt{celldm(1)},
all the other ratios and angles remain fixed at the values given in
the input of \texttt{pw.x}. 
This option can be controlled by the following variables:
\begin{verbatim}
ngeo       : the number of geometries to use.
             The lattice constant of these geometries is calculated from the
             input of pw.x. celldm(1) of this input is used for the central
             geometry. For the others celldm(1) is changed in steps of 
             step_ngeo. ngeo must be odd.
             Default: integer 1 for what=scf_*, 9 for what=mur_lc_*.
step_ngeo  : The step between the lattice constant at different geometries.
             Default: real 0.05 a.u.
ntry       : When the equilibrium lattice constant differs from the lattice 
             constant of the central geometry more than step_ngeo, the code 
             recalculates the Murnaghan equation centering the geometries on 
             the new minimum. This procedure is repeated up to ntry times.
             Default: integer 3
vmin_input : minimum volume for the plot of the energy as a function of volume.
             Default: real 0.98 times the volume of the first geometry.
vmax_input : maximum volume for the plot of the energy as a function of volume.
             Default: real 1.02 times the volume of the last geometry.
deltav     : distance between two volumes in the plot of the energy as a 
             function of the volume.
             Default: real calculated from nvol.
nvol       : number of volumes in Murnaghan plot
             Default : integer 51
flevdat    : file where the Murnaghan equation is written. The results of the
             Murnaghan fit are then written in flevdat.ev.out.
             Default: character(len=*) 'output_ev.dat'
flpsmur    : postscript file of the Murnaghan plot
             Default: character(len=*) 'output_mur.ps'
\end{verbatim}
An example of the use of this option can be found in \texttt{example05}.

\subsection{\texttt{what='mur\_lc\_bands'}}
With this option the code computes the band structure at the geometry 
obtained from the minimum of the Murnaghan equation. See the options 
\texttt{what='scf\_bands'} and \texttt{what='mur\_lc'} for a list of 
the variables that control these two options. They are used in the same 
way with the present option. An example of the use of this option can be 
found in \texttt{example06}.

\subsection{\texttt{what='mur\_lc\_ph'}}
This option is similar to \texttt{what='scf\_ph'} but the phonon calculation
is made at the geometry obtained from the minimum of the Murnaghan equation.
See the options \texttt{what='scf\_ph'} and \texttt{what='mur\_lc'} for a
list of the variables that control these two options. They are used in
the same way with the present option.
An example of the use of this option can be found in \texttt{example07}.

\subsection{\texttt{what='mur\_lc\_disp'}}
This option is similar to \texttt{what='scf\_disp'} but the phonon calculation
is made at the geometry obtained from the minimum of the Murnaghan equation.
See the options \texttt{what='scf\_disp'} and \texttt{what='mur\_lc'} for a
list of the variables that control these two options. 
They are used in the same way with the present option.
An example of the use of this option can be found in \texttt{example08}.

\subsection{\texttt{what='mur\_lc\_t'}}
With this option the code calculates the anharmonic vibrational
properties of the system within the quasi-harmonic approximation. The
outputs of the code are the lattice constant, bulk modulus, and pressure
derivative of the bulk modulus as a function of temperature. Moreover
the isobaric specific heat, the isoentropic bulk modulus, and the average 
Gruneisen parameter are calculated as a function of temperature. Separate
plots of the phonon dispersion along a path specified as described above,
are obtained for all the geometries used in this calculation. For each
geometry the code produces also a plot of the phonon density of states
and of the harmonic thermodynamical quantities. Finally the Gruneisen
parameters interpolated from the three central geometries are
shown on the same path used for the phonon dispersions. The input variables
that control these plot are those described in the option
\texttt{what='mur\_lc'} and \texttt{what='mur\_disp'} in addition to the 
following:

\begin{verbatim}
grunmin_input : minimum y coordinate for the Gruneisen parameter plot.
                Default: real, calculated from the Gruneisen parameters.
grunmax_input : maximum y coordinate for the Gruneisen parameter plot.
                Default: real, calculated from the Gruneisen parameters.
flpgrun       : file where the values of the Gruneisen parameters are written. 
                Default: character(len=*) 'output_pgrun.dat'
flgrun        : file where the Gruneisen parameters in a plotable form are
                written.
                Default: character(len=*) 'output_grun.dat'
flanhar       : file where the anharmonic thermodynamic quantities are written.
                Default: character(len=*) 'output_anhar.dat'
flpsanhar     : postscript file of the anharmonic quantities.
                Default: character(len=*) 'output_anhar.ps'
\end{verbatim}
The output file corresponding to different geometries can be identified
by the presence of the letters \texttt{g1}, \texttt{g2}, ... in the filename.
An example of the use of this option can be found in \texttt{example09}.

\subsection{\texttt{what='elastic\_constants'}}
With this option the code calculates the elastic constants of the solid.
Depending on the Laue class, it calculates the stress tensor for a set of 
strains of the lattice. For each strain, it relaxes the ions to their 
equilibrium positions if \texttt{frozen\_ions=.false.} or keep them
in the strained positions if \texttt{frozen\_ions=.true.}. 
Finally it computes the elastic constants from the numerical derivatives 
of the stress with respect to strain.
The number of independent strains is \texttt{ngeo\_strain}. The input variables
that control this option are:
\begin{verbatim}
frozen_ions: if .true. the elastic constants are calculated 
             keeping the ions frozen in the strained positions. 
             Default: logical .false.
ngeo_strain: the number of strained configurations used. 
             Default: integer 4
delta_epsilon: the interval of strain values between two geometries.
               To avoid a zero strain geometry that might have a
               different symmetry ngeo_strain must be even.
               Default: real 0.002
fl_el_cons: the name of the file that contains the elastic constants
            Default: character(len=*) 'output_el_con.dat'

\end{verbatim}

\subsection{\texttt{what='mur\_lc\_elastic\_constants'}}
As \texttt{what='elastic\_constants'} but the calculation is made at the
geometry that corresponds to the minimum of the Murnaghan equation. The
Murnaghan minimization is made as described for \texttt{what='mur\_lc'}.

\subsection{\texttt{what='piezoelectric\_tensor'}}
With this option the code calculates the piezoelectric tensor 
($g_{\alpha,m}$ $\alpha=1,3\ m=1,6$) of the solid.
Depending on the point group, it calculates the polarization of the
solid for a set of strains of the lattice. For each strain, it relaxes 
the ions to their equilibrium positions when \texttt{frozen\_ions=.false.} 
or keep them in the strained positions when \texttt{frozen\_ions=.true.}. 
Finally it computes the piezoelectic tensor from the numerical derivatives 
of the polarization with respect to strain.
The number of independent strains is \texttt{ngeo\_strain}. The input variables
that control this option the same as those used for the elastic constant:
\begin{verbatim}
frozen_ions: if .true. the piezoelectric tensor is calculated 
             keeping the ions frozen in the strained positions. 
             Default: logical .false.
ngeo_strain: the number of strained configurations used. 
             Default: integer 4
nppl:        the number of k-points per line in Berry phase calculations
             Default: integer 51
delta_epsilon: the interval of strain values between two geometries.
               To avoid a zero strain geometry that might have a
               different symmetry ngeo_strain must be even.
               Default: real 0.002
\end{verbatim}
If a file with the elastic constants is found on disk, the code computes also
the direct piezoelectric tensor $d_{\alpha,m}$ which describes
the polarization induced by an external stress, as $d_{\alpha,m}=
\sum_n g_{\alpha,n} C_{nm}^{-1}$. All these quantities are calculated 
for a vanishing electric field.

\subsection{\texttt{what='mur\_lc\_piezoelectric\_tensor'}}
As \texttt{what='piezoelectic\_tensor'} but the calculation is made at the
geometry that corresponds to the minimum of the Murnaghan equation. The
Murnaghan minimization is made as described for \texttt{what='mur\_lc'}.

\subsection{\texttt{what='polarization'}}
With this option the code calculates the spontaneous polarization.
The code makes a self-consistent calculation followed by three Berry
phase calculations in which the polarization is calculated in the
direction parallel to the three primitive Bravais lattice vectors and 
prints the spontaneous polarization in cartesian coordinates.
The input variables that control this option is:
\begin{verbatim}
nppl: the number of k points per string. In the perpendicular plane the
      number of k-point is that given as input of the pw.x code.
      Default: integer 51
\end{verbatim}

\subsection{\texttt{what='mur\_lc\_polarization'}}
As \texttt{what='polarization'} but the calculation is made at the
geometry that corresponds to the minimum of the Murnaghan equation. The
Murnaghan minimization is made as described for \texttt{what='mur\_lc'}.

\end{document}
