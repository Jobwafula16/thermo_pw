%!
%! Copyright (C) 2014-2016 Andrea Dal Corso 
%! This file is distributed under the terms of the
%! GNU General Public License. See the file `License'
%! in the root directory of the present distribution,
%! or http://www.gnu.org/copyleft/gpl.txt .
%!
\documentclass[12pt,a4paper]{article}
\def\version{0.7.0}
\def\qe{{\sc Quantum ESPRESSO}}
\def\tpw{{\sc THERMO\_PW}}

\usepackage{html}
\usepackage{color}

\usepackage{graphicx}

\definecolor{web-blue}{rgb}{0,0.5,1.0}
\definecolor{coral}{rgb}{1.0,0.5,0.3}
\definecolor{red}{rgb}{1.0,0,0.0}
\definecolor{green}{rgb}{0.,1.0,0.0}

\textwidth = 17cm
\textheight = 24cm
\topmargin =-1 cm
\oddsidemargin = 0 cm

\def\pwx{\texttt{pw.x}}
\def\phx{\texttt{ph.x}}
\def\configure{\texttt{configure}}
\def\PWscf{\texttt{PWscf}}
\def\PHonon{\texttt{PHonon}}
\def\thermo{\texttt{thermo\_pw}}
\def\make{\texttt{make}}

\begin{document} 
\author{Andrea Dal Corso (SISSA - Trieste)}
\date{}

%\def\SissaImage{./sissa_on_white.png}

\title{
%  \includegraphics[width=6cm]{\SissaImage}\\
  \vskip 1cm
  {\color{red} \Huge \tpw\ User's Guide} \\
  \Large (version \version)
}

\maketitle

\tableofcontents

\newpage

\section{\color{coral}Introduction}

This guide covers the installation and usage of the \thermo\ package. 
It assumes that you have some familiarity with the \qe\ package. 
If not please consult the web site: \texttt{http://www.quantum-espresso.org}.

\thermo\ is a \texttt{FORTRAN} code that computes material properties.
It can be considered as a set of drivers that,
at low level, call the \qe\ routines and, at high level, have pre-processing
tools that reduce the number of information required to the user and
post-processing tools that convert the output of \qe\ into plots of material 
properties directly comparable with experiment.

\thermo\ has the following directory structure, contained in a subdirectory 
\texttt{thermo\_pw/} that should be put in the main directory of the \qe\ tree:

\begin{tabular}{ll}
\texttt{Doc/}      & : contains this user\_guide and other documentation\\
\texttt{examples/} & : some running examples \\
\texttt{inputs/}   & : a collection of useful inputs \\
\texttt{pseudo\_test/} & : a collection of inputs to test a pseudopotential library\\
\texttt{space\_groups/} & : a collection of structures for many space groups \\
\texttt{lib/}      & : source files for modules used by \thermo\ \\
\texttt{src/}      & : source files for \thermo\ \\
\texttt{tools/}    & : source files for auxiliary tools \\
\end{tabular}\\

The \thermo\ package can calculate the following quantities:
\begin{itemize}
\item Plot of the Brillouin zone (the structure can be seen by reading the
input of \thermo\ by the \texttt{XCrySDen} program).

\item Plot of the X-rays powder diffraction pattern of the input crystal.

\item Total energy at fixed geometry.

\item Total energy as a function of the kinetic energy cut-off.

\item Total energy as a function of {\bf k}-points and smearing.

\item Electronic band structure at fixed geometry.

\item Electronic density of states at fixed geometry.

\item Electronic heat capacity as a function of temperature (for metals only).

\item Complex dielectric constant as a function of the complex
frequency $\omega$ at fixed geometry.

\item Inverse dielectric constant at a given wavevector {\bf q} as a function 
of the complex frequency $\omega$ at fixed geometry.

\item Phonon frequencies at fixed geometry.

\item Phonon dispersions at fixed geometry and computation of the harmonic
thermodynamic properties: vibrational energy, vibrational free energy,
vibrational entropy, and constant volume heat capacity as a function of
temperature.

\item Frozen ions and total elastic constants at fixed geometry.

\item Fit of the total energy as a function of the lattice parameters with
a quadratic or quartic polynomial and determination of equilibrium lattice 
parameters.  Murnaghan fit for cubic systems.

\item Electronic band structure at the minimum of the total energy.

\item Electronic density of states at the minimum of the total energy.

\item Complex dielectric constant as a function of the complex
frequency $\omega$ at the minimum of the total energy.

\item Inverse dielectric constant at a given wavevector {\bf q} as a function 
of the complex frequency $\omega$ at the minimum of the total energy.

\item Phonon frequencies at the minimum of the total energy.

\item Phonon dispersions and harmonic thermodynamic quantities
at the minimum of the total energy.

\item Frozen ions and total elastic constants at the minimum of the total
energy.

\item Anharmonic properties within the quasi-harmonic approximation: 
lattice parameters, thermal expansion tensor, volume, volume thermal 
expansion, and constant strain heat capacity as a function of temperature; 
phonon frequencies and mode Gr\"uneisen parameters interpolated at a given
geometry or at the equilibrium geometry at a given temperature
(limited to cubic, tetragonal, orthorhombic, and hexagonal systems).
Bulk modulus and pressure derivative of the bulk modulus, isobaric heat 
capacity, isoentropic bulk modulus, and average Gr\"uneisen parameter as 
a function of temperature (limited to cubic systems).

\item Surface band structure identification and plot of the projected bulk
band structure.

\end{itemize}

\thermo\ can run on both serial and parallel machines using all 
the parallellization options of \qe. Moreover, \thermo\ can run using 
several images.
When possible, the image parallelization is used in an asynchronous way.
One processor takes the role of master and distributes the work 
to all the images that carry it out independently. Presently 
the total energies of several geometries for the determination of the 
equilibrium geometry are calculated in parallel when
there are several images. Stresses or total energies at different strained 
geometries needed for the calculation of the elastic constants are 
calculated in parallel. 
The phonon calculations are carried out in parallel, each image doing one 
representation of a {\bf q} point.

\section{\color{coral}People}
The \thermo\ code is designed, written, and maintained by Andrea Dal Corso 
(SISSA - Trieste). It is an open source code distributed, as is, within the GPL
license.  

\section{\color{coral}Installing, Compiling, and Running}

\subsection{\color{web-blue}Installing}

\thermo\ is a package tightly bound to \qe. It cannot be compiled without
it. For instruction on how to download and compile \qe, please 
refer to the general User's Guide, available in file \texttt{Doc/user\_guide.pdf}
under the main \qe\ directory, or in the web site 
\texttt{http://www.quantum-espresso.org}.
When \qe\ is correctly configured, you need to download the \PHonon\ 
package. This can be done automatically typing \texttt{make ph} in
the main \qe\ directory.

At this point you need the source code of the \thermo\ package. 
\texttt{Thermo\_pw} can be downloaded from the 
\texttt{qe-forge} website at \texttt{http://www.qe-} 
\texttt{forge.org/gf/project/thermo\_pw/}, by clicking on the \texttt{Files}
section and downloading one of the \texttt{.tar.gz} files.
The versions of \texttt{thermo\_pw} and of \qe\ must be carefully matched
as illustrated in Table 1. 
There is no \texttt{thermo\_pw} package for the versions of \qe\ not listed 
there. Unpacking the \texttt{.tar.gz} files in the main \qe\ directory, for
instance with the command \texttt{tar -xzvf thermo\_pw.0.5.0.tar.gz}, you
obtain a directory \texttt{thermo\_pw} with the source files. \\
It is also possible to download the \texttt{CVS} version of 
\texttt{thermo\_pw} as described at
the web page: \texttt{http://www.qe-} \texttt{forge.org/gf/project/thermo\_pw/scmcvs/?action=AccessInfo}.
Usually the CVS version of \texttt{thermo\_pw} contains the most recent
features and bug fixes but it might work only with the \texttt{SVN} 
version of \qe\ and its use is not advised.
Please read the web page: 
\texttt{http://people.sissa.it/\textasciitilde dalcorso/thermo\_pw\_help.html} 
to have updated information about the compatibility between the \texttt{CVS} 
version of \texttt{thermo\_pw} and \qe\ or if you have problems downloading
the \texttt{CVS} version of \texttt{thermo\_pw}.
This page contains also information on critical bugs found in the 
\texttt{thermo\_pw} packages and should be consulted before using 
\texttt{thermo\_pw}.

\begin{center}
\begin{table}
\begin{tabular}{ll}
\hline
\hline
\texttt{thermo\_pw} & \qe  \\
\hline
0.7.0 & SVN 17-11-2016 \\
0.6.0 & 6.0.0 \\
0.5.0 & 5.4.0 \\
0.4.0 & 5.3.0 \\
0.3.0 & 5.2.1, 5.2.0 \\
0.2.0 & 5.1.2 \\
0.1.0 & 5.1.1 \\
\hline
\hline
\end{tabular}
\caption{Compatibility between the versions of 
\texttt{thermo\_pw} and of \texttt{QE}.}
\end{table}
\end{center}

\subsection{\color{web-blue}Compiling}

In order to compile \thermo\ the main \texttt{Makefile} and the files
\texttt{install/makedeps.sh} and \texttt{install/plugins\_makefile}
of \qe\ need to be changed. In the \texttt{thermo\_pw}
directory give the command \texttt{make join\_qe}. This command exchanges
the files contained in \texttt{thermo\_pw} with those of the \qe\ package.
Typing \texttt{make thermo\_pw} from the main \qe\ directory, or \texttt{make} 
from the \texttt{thermo\_pw}\ directory, produces the executable
\texttt{thermo\_pw/src/thermo\_pw.x} that can be found in the 
\qe\ \texttt{bin/} directory. A few other tool codes are produced as well
and linked in the \texttt{bin/} directory of \qe.

\thermo\ has been written on a PC with Linux operating system using the
\texttt{gfortran} compiler with \texttt{openMPI} parallelization. It has
been run in parallel on a Linux cluster with several hundreds processors.
It has been tested with \texttt{Intel/15.0} + \texttt{mkl/11.2} +
\texttt{openMPI} and with \texttt{Intel/13.1} + \texttt{mkl/11.2} +
\texttt{openMPI/1.6.5}.
Any other combination of computer/operating system has not been tested, but 
usually \thermo\ runs on the same systems where \qe\ runs. 

\subsection{\color{web-blue}Searching help and reporting bugs}
If you have problems installing \texttt{thermo\_pw} or need help
with some of its features, please subscribe to the \texttt{thermo\_pw-forum}
mailing list to report your problems. Requests for the implementation 
of new features are also welcome. If you think to have found a bug 
in \texttt{thermo\_pw} you can report it to the mailing list or 
write directly to me (\texttt{dalcorso.at.sissa.it}).

\subsection{\color{web-blue}Uninstalling}

In order to remove \thermo, give the command \texttt{make leave\_qe} in the
\texttt{thermo\_pw} directory. Then just remove the directory. 

\subsection{\color{web-blue}Running \thermo}

In order to use the \thermo\ code you need to create a file called
\texttt{thermo\_control} in your working directory in addition to 
the input of \texttt{pw.x} and, if requested by the task, to the
input of \texttt{ph.x} that must be called \texttt{ph\_control}.
The input of \texttt{pw.x} can have any name and is given as input to
the \thermo\ code. It is not necessary to specify an \texttt{outdir} 
directory in the \texttt{ph.x} input.

A typical command for running \thermo\ is:
\begin{verbatim}
mpirun -n np thermo_pw.x -ni ni ... < input_pw > output_thermo_pw
\end{verbatim}
where \texttt{np} is the number of processors and \texttt{ni} is the number 
of images. The dots indicate the other parallelization options that
you can find in the \qe\ manual.

Note that it is very easy to waste resources if the number of 
images is too large. Unused images wait for the working images to complete
their tasks wasting cpu-time in an endless loop. 
Some options 
do not use the image feature, so you have to know how the calculation 
is divided and the number of images must not
be larger than the number of tasks (see below an indication of this number
for each option). If you have doubts on this point use one image.

The outputs of the \thermo\ code are one or more \texttt{postscript} files 
with plots of the material properties. \thermo\ produces also files with 
the data of 
the plot and scripts for the \texttt{gnuplot} program. 
Usually, the user does not need to modify these files, but they allow 
the improvement of the figures when needed.
The plot of the Brillouin zone (BZ) is made with the help of the 
\texttt{asymptote} code. \texttt{Thermo\_pw} produces a script 
for the \texttt{asymptote} code and can also run it to produce the \texttt{pdf}
file of the BZ. 

\newpage

\section{\color{coral}Input variables}

The \texttt{pw.x} and \texttt{ph.x} input files are described in the \qe\ documentation.
In this section we discuss only the creation of the file
\texttt{thermo\_control}. This file contains a namelist:  
\begin{verbatim}
&INPUT_THERMO
  what=' ',
  ...
 /
\end{verbatim}
The \texttt{what} variable controls the sequence of calculations made
by \thermo. For each possible value of \texttt{what}, we discuss briefly the
input variables that you can use to control the output plots. Usually
default values of the input variables are sufficient to carry out  
the basic \thermo\ tasks and you are not supposed to set any variable except
\texttt{what}, but in some cases these input variables give you more control
on the calculation and can be used to tune its accuracy.

\thermo\ actually writes on file the data to plot and writes a script to plot
these data. The output postscript files are produced by invoking the 
\texttt{gnuplot} program within the code. Usually any modern Linux 
distribution provides a package to install this code, or has it already 
installed. If not, you can download it from 
\texttt{http://www.gnuplot.info/}. \\
Three input variables of \thermo\ control 
the use of the \texttt{gnuplot} code:

\begin{verbatim}
lgnuplot    : if .TRUE. gnuplot is called from within the program
              and the postscript files are immediately available
              Default: logical .TRUE.
gnuplot_command  : the command used to call gnuplot.
              Default: character(len=*) 'gnuplot'
flgnuplot   : initial part of the name of the files where gnuplot scripts 
              are written.
              Default: character(len=*) 'gnuplot.tmp'
\end{verbatim}
If you are running \thermo\ on a system that has not \texttt{gnuplot}
you can disable the production of the postscript files and use some other
graphical tools to produce the plots from the output data. 

\subsection{\color{web-blue}Temperature and pressure}
Several quantities in \thermo\ can be calculated as a function of temperature.
Moreover the equilibrium geometry can be searched at fixed pressure 
minimizing the enthalpy instead of the energy.
For the options where these features are active, the values of temperature
and pressure are controlled by the following variables:
\begin{verbatim}
tmin       : minimum temperature.
             Default: real 1 K
tmax       : maximum temperature.
             Default: real 800 K
deltat     : interval between two temperatures. Be careful with this value
             because too small or too large values of this parameter could 
             give numerical errors in the temperature derivatives used to 
             calculate anharmonic properties.
             Default: real 3 K
ntemp      : number of temperatures
             Default: integer determined from previous data
pressure   : The external pressure. The crystal parameters are
             calculated minimizing the enthalpy at this pressure. 
             Given in kbar units. 
             Default: real 0.0 kbar
\end{verbatim}
Note that when you fix the external pressure, the geometries chosen to 
fit the enthalpy must be about the minimum geometry at that pressure.

\subsection{\color{web-blue}Coordinates and structure}
The \texttt{thermo\_pw} code requires the Bravais lattice of the solid. 
Moreover for computing some quantities it assumes that the direct lattice 
vectors are those provided by the routine \texttt{latgen.f90} of the \qe\ 
distribution. 
For this reason it is not recommended to use \texttt{ibrav=0} in the
\texttt{pw.x} input. The preferred method is to give the value of 
\texttt{ibrav} using primitive vectors provided by \qe.
It is also possible to specify the \texttt{space\_group} number and 
the coordinates of the inequivalent atoms. When the 
\texttt{pw.x} input contains the \texttt{ibrav=0} option, 
\texttt{thermo\_pw} writes on output
the values of \texttt{ibrav}, \texttt{celldm}, and of the atomic coordinates
that should be used in the input of \texttt{pw.x} to simulate the same 
solid and stops. There are however two input variables of \texttt{thermo\_pw} 
that can modify this behaviour:
\begin{verbatim}
continue_zero_ibrav : when ibrav=0 in the input of pw.x and this variable 
             is set to .TRUE. thermo_pw runs with ibrav=0 (not recommended
             except when you deal with a supercell). 
             When this variable is .FALSE. and ibrav=0 the behavior 
             depends on find_ibrav.
             Default : logical .FALSE.

find_ibrav : This variable is active only when continue_zero_ibrav=.FALSE..
             When this variable is set to .TRUE. and the input of pw.x
             has ibrav=0, thermo_pw finds the values of ibrav, celldm,
             and of the atomic positions that produce the same crystal 
             and continue the calculation. The geometry 
             used by thermo_pw might be rotated with respect to the input and
             have different primitive vectors.
             When this variable is .FALSE. the code stops after writing in
             output ibrav, celldm, and the atomic positions. These variables 
             can be copied in the pw.x input. Note that the automatic 
             identification of the lattice does not work for supercells.
             Default : logical .FALSE.
\end{verbatim}

\subsection{\color{web-blue}\texttt{what='scf'}}
With this option the code computes only the total energy. This is a single
calculation as if running \texttt{pw.x} with the given input.
No other input variable is necessary.
An example of the use of this option can be found in \texttt{example01}. \\
Number of tasks for this option: \texttt{1}.

\subsection{\color{web-blue}\texttt{what='scf\_ke'}}
With this option the code makes several self-consistent calculations, 
in parallel on several images, varying the kinetic energy cut-off for 
the wavefunctions and for the charge density. 
In the input of \texttt{pw.x} one specifies the minimum values for these two 
cut-offs. These values are then increased in fixed intervals controlled by the 
following variables. The energy is then plotted as a function of the 
wavefunctions kinetic energy cut-off, a different curve for each value of 
the charge density cut-off. \\
The variables that control this option are:

\begin{verbatim}
nke        : number of kinetic energies tested for the wavefunctions cut-off.
             Default: integer 5
deltake    : delta of wavefunctions kinetic energy cut-off in Ry.
             Default: real 10 Ry
nkeden     : number of kinetic energies tested for the charge density
             cut-off.
             Default: integer 1
deltakeden : delta of charge density kinetic energy cut-off in Ry.
             Default: real 100 Ry.
flkeconv   : name of the file where the data with the total energy as a
             function of the kinetic energy is written.
             Default: character(len=*) 'output_keconv.dat'
flpskeconv : name of the postscript file with the plot of the total energy as
             a function of the kinetic energy cut-off.
             Default: character(len=*) 'output_keconv.ps'
\end{verbatim}
An example of the use of this option can be found in \texttt{example10}. \\
Number of tasks for this option: \texttt{nke * nkedens}.

\subsection{\color{web-blue}\texttt{what='scf\_nk'}}
With this option the code makes several self-consistent calculations, 
in parallel on several images, varying the size of the {\bf k}-point grid, 
and optionally for metals the smearing parameter \texttt{degauss}. In the 
input of \texttt{pw.x} the minimum value of these parameters is given and 
these values are increased in fixed intervals controlled by the following 
variables. On output the energy is plotted as a function of the mesh size, 
one curve for each smearing parameter.\\
The variables that control this option are:

\begin{verbatim}
nnk        : the number of different values of nk to test
             Default: integer 5
deltank(3) : the interval between nk values. All three values of nk1, nk2, and
             nk3 are updated simultaneously.
             Default: integer 2  2  2
nsigma     : the number of smearing intervals.
             Default: integer 1 
deltasigma : the distance between different smearing values.
             Default: 0.005 Ry
flnkconv   : file where the data with the k point convergence is written
             Default: character(len=*) 'output_nkconv.dat'
flpsnkconv : name of the postscript file with the k points convergence plot.
             Default: character(len=*) 'output_nkconv.ps'
\end{verbatim}
An example of the use of this option can be found in \texttt{example11}. \\
Number of tasks for this option: \texttt{nnk * nsigma}.

\subsection{\color{web-blue}\texttt{what='scf\_bands'}}
With this option the code makes a self-consistent calculation followed 
by a band structure calculation. There is no image parallelization and no
advantage to use several images. The output of the band structure 
calculation is further processed in order to produce a plot of the band 
structure.
In insulators the zero of the energy is the highest valence band 
of the first {\bf k} point, in metals the Fermi energy. \\
The energy bands plot can be modified by the following variables:

\begin{verbatim}
emin_input : minimum energy for the band dispersion plot (in eV).
             Default: real minimum of the bands
emax_input : maximum energy for the band dispersion plot (in eV).
             Default: real maximum of the bands
nbnd_bands : the number of bands in the band calculation
             Default: integer 2*nbnd, where nbnd is the number of bands 
             given in pw.x input or calculated by pw.x.
only_bands_plot: if the files with the bands and the representations
             are already on files, this option allows to change the 
             parameters of the plot (such as the maximum or minimum energy) and
             do another plot without additional calculation. If the files are
             missing and this variable is .TRUE. an error occurs. Note that
             using this option you cannot change the path.
             Default: logical .FALSE.
lsym       : if .TRUE. does the symmetry analysis of the bands
             Default: .TRUE.
enhance_plot: if .TRUE. writes on the band plot the point group labels,
             and colors with differente colors lines at the zone border.
             Default: .FALSE.
long_path  : if .TRUE. plot the bands in all the Brillouin zone path. 
             Otherwise makes a faster calculation on a short path. 
             Default: .TRUE.
flpsband   : postscript file with the electronic band structure
             Default: character(len=*) 'output_band.ps'
\end{verbatim}
Number of tasks for this option: \texttt{1}.

By default, the bands are plotted along a fixed path in the Brillouin
zone, but the user can modify this behavior giving the path at the end of 
the \texttt{INPUT\_THERMO} namelist with the same format used for 
the \texttt{pw.x} input. The automatic path generation is not available 
for base-centered monoclinic and for triclinic Bravais lattices. For these
lattices the path must be given explicitly.
The following variables control the path:

\begin{verbatim}
q_in_band_form   : only the first and last point of each k path are given.
                 The weight of each k point is an integer, the number of 
                 points in the line that starts at this k point.
                 Default: .TRUE.
q_in_cryst_coord : the k - points are given in crystal coordinates.
                 Default: .FALSE.
point_label_type : the label definition (see the BZ manual)
                 Default: SC
\end{verbatim}
Note that the path cannot be given in the input of \texttt{pw.x} that
must contain the information to generate the mesh of {\bf k} points 
used in the self-consistent calculation. 
An example of the use of this option can be found in \texttt{example02}.
If you give explicitly the path, be careful with options that require
geometry changes (see below). Only automatic paths, or path given through 
letter labels are easily recalculated. The other paths could turn out 
to be correct only for one geometry.

\subsection{\color{web-blue}\texttt{what='scf\_2d\_bands'}}
With this option the code calculates the bands after a self-consistent 
calculation as with the option \texttt{what='scf\_bands'}, but it assumes 
that the cell contains a slab with surfaces perpendicular to the $z$ 
direction. Therefore the two-dimensional Bravais lattice of the surface 
is identified and the default path is chosen on the two-dimensional 
Brillouin zone. 
There are three options: Plot of the projected band structure (PBS); 
plot of the bands of the slab; plot of the bands of the slab above the
projected band structure (the Fermi energies are aligned). In the first 
case the code computes several paths
of {\bf k}-points parallel to the surface (at different {\bf k}$_z$) and does 
not plot the individual bands but selects the energy regions in which 
there are bulk states. The second case is similar to a standard band plot. 
The default path contains only {\bf k}-points parallel to the 
surface (with {\bf k}$_z=0$). The third case assumes that the projected band 
structure has been already calculated and the information to plot it 
can be found on the file \texttt{flpbs}. For the rest it is 
similar to case two. For each direction, bands belonging to different 
irreducible representations of the small group {\bf k} can be plotted 
in the same panel or on different panels.\\
This option is controlled by the following variables:
\begin{verbatim}
lprojpbs : When .TRUE. the projected band structure (PBS) is calculated 
     if nkz > 1 otherwise it is read from file. Usually this variable is
     .TRUE.. Set it to .FALSE. if you do not want to see the PBS, 
     or if you want to see the bands of a bulk projected on the 
     surface Brillouin zone without the PBS.
     Default: logical .TRUE. (forced to .FALSE. if what is not 'scf_2d_bands')
nkz  : The number of k_z values used for the PBS plot. 
     If lprojpbs is .FALSE. a plot of the bulk bands projected on the 
     surface Brillouin zone is produced.
     Default: integer 4 (forced to 1 if what is not 'scf_2d_bands').
gap_thr : minimum size (in eV) of the gaps in the PBS
     Default: double precision 0.1 eV
sym_divide: When .TRUE. the bands belonging to different irreducible 
     representations are plotted in different panels. This option
     can be controlled by variables specified in the path (see below)
     Default : logical .FALSE.
identify_sur: When .TRUE. the surface bands are searched and identified on the
     surface band structure. 
     Default : logical .FALSE.
dump_states: If .TRUE. and identify_sur is .TRUE. dump on the file 
     'dump/state_k_#' the planar averages of the density (and in the 
     noncollinear case also of the magnetization density) of each state. 
     One file for each k point is produced and # is the number of the k points.
     (Use with a small number of k points or it might create quite large files).
     Default: logical .FALSE.
sur_layers: The number of surface layers on which we add the charge density of 
     each state to check if it is a surface state.
     Default : integer 2
sur_thr : the threshold (in percentage) of the charge density that must be
     on the surface layers to identify a state as a surface state.
     Default: calculated from the actual charge density values of the states.
subtract_vacuum: if .TRUE. the charge density of each state on vacuum is
     subtracted (to remove the vacuum states that are confused with surface
     states)
     Default : .TRUE.
force_bands: when .TRUE. the bands are plotted in any case.
     Used to plot the bulk bands on top of the PBS, mainly for debugging.
     Default: logical .FALSE.
only_bands_plot: if the files with the bands, the representations, the pbs
     and the projections are already on files, this option allows to change
     the parameters of the plot (such as the maximum energy or sur_thr) and
     do another plot without any additional calculation. If the files are
     missing and this variable is .TRUE. an error occurs.
     Default: logical .FALSE.
flpbs : the name of the file that contains the information on the projected
     band structure.
     Default: character(len=*) 'output_pbs'
flprojlayer : the name of the file that contains the information of the
     projection of the charge density of each state on each layer.
     Calculated only when identify_sur is .TRUE..
     Default: character(len=*) 'output_projlayer'
\end{verbatim}
The bands and the gnuplot scripts are saved on the same files that would
be used with the option \texttt{what='scf\_bands'}. \\
Number of tasks for this option: \texttt{1}. Image parallelization is
not useful with this option. 

By default the symmetry separation is not carried out. The code plots the
bands of the slab on the same panel with a different color for each 
representation as in the bulk band structure plot (color refer to the 
representations of the slab small group of {\bf k}). 
In order to plot in different panels the different representations
the user can specify \texttt{sym\_divide=} \texttt{.TRUE.}.
By default this option is disabled and its use is rather tricky.
In order to use this option you must indicate explicitly the
path on the two dimensional Brillouin zone using the option
\texttt{q\_in\_band\_form=.TRUE.}. Close to the
starting point of a given line you must indicate the number of 
representations for that line ($0$ means all representations) and which ones.
For instance for a $(111)$ surface of an fcc metal you may
want to plot separately the even and odd states in the direction 
$\bar \Gamma-\bar M$. In order to do so you can specify the path as follows:
\begin{verbatim}
5
gG   30  0
K    30  0
M    30  1  1
gG   30  1  2
M     1  0
\end{verbatim}
Which representation is indicated by its number (in this case $1$ or $2$).
The number of the representation and the small point group of each 
{\bf k}-point can be found in the output of \texttt{thermo\_pw}.
However you must be careful because these representation numbers refer
to the slab small point group of each {\bf k}-point when you
plot the slab band structures and to the surface small point 
group of each {\bf k}-point when you plot a PBS.
For a given {\bf k}-point parallel to the surface, the 
point group of the slab might be larger than the one of the surface, because it 
contain operations that exchange the two surfaces.
In this case the surface point group of a given {\bf k}-point parallel 
to the surface can be found in the projected band structure calculation,
for a {\bf k}-point with the same component parallel to the surface
and a generic {\bf k}$_z$. Actually the point group for this {\bf k}-point 
does not contain operations that exchange {\bf k}$_z$ with $-${\bf k}$_z$.
Therefore the symmetry separation of the PBS is done using the
surface point groups. Some particular values of {\bf k}$_z$, such as 
{\bf k}$_z=0$
might have a different point group than a generic {\bf k}$_z$ and in this
case the representations are transformed into those of the smaller
group of the surface using the group-subgroup relationships and
the symmetry descent of the irreducible representations
(only when \texttt{sym\_divide=.TRUE.}).
It is the user responsibility to specify the same number of panels
for the PBS and for the slab calculation and to assure that the
representations plotted in each panel correspond to each other.
Returning to the example of the $(111)$ surface of an fcc, in the direction
$\bar \Gamma-\bar K$ the slab has $C_2$ symmetry about the $x$-axis, a symmetry
that the surface has not. Therefore you can plot with two different colors
the bands that belong to the $A$ or $B$ representations of the slab,
(states even or odd with respect to a $180^\circ$ rotation about
the $x$ axis, an operation that exchanges the two surfaces)
but you cannot separate the PBS along the $\bar \Gamma-\bar K$ direction into
even or odd states. You might specify two different panels with the
$A$ or $B$ bands in each, but the PBS in the two panels will be the same. 
On the contrary, along the $\bar \Gamma-\bar M$ direction, both the slab and 
the surface have $C_s$ symmetry, so you can separate both the PBS 
and the surface states in two different panels.

There is no input variable to control or change the colors or style of the 
plot. To change the defaults you can modify directly the gnuplot script,
it is written in such a way that a change to a few variables can
control the entire plot.

\subsection{\color{web-blue}\texttt{what='scf\_dos'}}
With this option the code makes a self-consistent calculation followed
by a band structure calculation on a uniform mesh of {\bf k}-points and
computes and plots the electronic density of states. \\
There is no image parallelization and no advantage to use several images. \\
This option is controlled by the following variables:
\begin{verbatim} 
deltae        : energy interval for electron dos plot (in Ry).
                Default: real 0.01 Ry.
ndose         : number of frequency points in the dos plot.
                Default: determined from previous data
nk1_d, nk2_d, nk3_d : thick mesh for dos calculation.
                Default: integer 32, 32, 32
k1_d, k2_d, k3_d : the shift of the k point mesh
                Default: integer 0, 0, 0
sigmae         : the smearing used for dos calculation (in eV).
                If 0.0 uses the degauss of the electronic structure
                calculation in metals and 0.01 Ry in insulators.
                Default: real 0.0 
legauss        : When .true. computes the electronic dos using a gaussian
                smearing. When .false. uses the same smearing of the
                electronic structure calculation in metals or gaussian smearing
                in insulators.
                Default: logical .false.
fleldos        : name of the file that contains the electron dos data
                Default: character output_eldos.dat
flpseldos      : name of the postscript file that contains the electron 
                dos picture
                Default: character output_eldos.ps
fleltherm      : name of the file that contains the electron thermodynamic data 
                Default: character output_eltherm.dat
flpseldos      : name of the postscript file that contains the picture of the
                electron thermodynamic quantities
                Default: character output_eltherm.ps
\end{verbatim}
The minimum and maximum energy, as well as the number of bands,
are specified as with the option \texttt{what='scf\_bands'}.
However with the present option no shift is applied to the bands
and the minimum and maximum energies refer to the unshifted eigenvalues.
Note that after a calculation with \texttt{what='scf\_dos'} you can
run the tool code \texttt{epsilon\_twp.x} to evaluate the frequency
dependent dielectric constant (for insulators only). \\
With this option, in the metallic case, 
the code computes the electronic thermodynamic quantities of
a gas of independent electrons whose energy levels give the calculated 
density of states and produces a postscript file
with the electronic excitation energy, free energy, entropy,
and constant strain heat capacity as a function of temperature. 
The zero of the electronic energy is the energy at the smallest temperature 
required in input when it is lower than $4$ K or $4$ K. 

\subsection{\color{web-blue}\texttt{what='plot\_bz'}}
With this option the code writes a script to make a plot
of the Brillouin zone and of the path (the default one or the one given in 
input). 
The script must be read by the \texttt{asymptote} code, available at 
\texttt{http://asymptote.sourceforge.net/}. In many Linux distributions
this code is available as a separate package, but it is not installed by
default. \\
The following variables control the plot:
\begin{verbatim}
lasymptote  : if .TRUE. asymptote is called from within the program
              and the pdf file with a plot of the BZ is produced
              Default: logical .FALSE.
flasy       : initial part of the name of the file where the asymptote script
              is written and of the name of the pdf file.
              Default: character(len=*) 'asy_tmp'
asymptote_command  : the command that invokes asymptote and produces the 
              pdf file of the BZ.
              Default: character(len=*) 'asy -f pdf -noprc flasy.asy'
npx         : used only in the monoclinic cell, this parameter is needed
              to determine the shape of the Brillouin zone. The default
              value is usually large enough, but for particular shapes of
              the monoclinic Brillouin zone it could be small. If the
              code stops with an error asking to increase npx, double 
              it until the error disappears.
              Default: integer 8
\end{verbatim}
The structure of the solid can be seen using the \texttt{XCrySDen} code
that can read the input file of \texttt{pw.x}. You can find the code at
\texttt{http://www.xcrysden.org/}. \thermo\ produces also a file in the
\texttt{xsf} format called \texttt{prefix.xsf}, where the variable 
\texttt{prefix}
is given in the input of \texttt{pw.x}. This can be useful when
the inequivalent atomic positions and the space group are given in the input
of \texttt{pw.x}. To see an \texttt{xsf} file, give the command
\texttt{xcrysden --xsf file.xsf}.

With this option the code produces also a file with the X-ray
powder diffraction intensities for the solid. A plot shows the 
scattering angles and the relative intensity of each peak. Note 
that this plot is made using 
a superposition of atomic charges, not the self-consistent charge.
By setting the flag \texttt{lformf=.TRUE.} the atomic form factors of all 
the atomic types used to calculate the intensities are plotted. 
By setting the flag \texttt{lxrdp=.TRUE.} the intensities plot is done also 
after the cell optimization and after a self-consistent calculation 
for the options that support it.
The variables that control these plots are:
\begin{verbatim}
lambda      : The X-ray wavelength (in A) used to calculate the scattering 
              angles.
              Default: Cu alpha line 1.541838 A if lambda_element is empty
lambda_elem : The anode element, used to set the X-ray wavelength.
              Supported elements 'Cr', 'Fe', 'Co', 'Cu', 'Mo'.
              NB: lambda must be zero to use lambda_elem, otherwise the value
              of lambda given in input is used.
              Default: character(len=2) ' ' 
flxrdp      : name of the file where the scattering angles and intensities 
              are written.
              Default: character 'output_xrdp.dat'
flpsxrdp    : name of the postscript file with the X-ray diffraction
              spectrum.
              Defaults: character 'output_xrdp.ps'
lxrdp       : if .TRUE. compute the xrdp also after the cell optimization
              with all the options mur_lc_... with the uniformly strained
              atomic positions and after the scf calculation if supported
              by the option. 
              Default: logical .FALSE.
lformf      : if .TRUE. plot also the form factor of each atom type
              present in the solid. Note that the atom type is recognized
              from the atom name in the thermo_pw input. The name
              must coincide with the symbols in the periodic table. 
              (Cu, H, Li, Li1, ... are correct, CU, LI, H1 ... are wrong)
              Default: logical .FALSE.
smin        : minimum value of s used in the atomic form factor plot.
              Default: real 0.0
smax        : maximum value of s used in the atomic form factor plot.
              Default: real 1.0
npoint      : number of points in which the atomic form factor is calculated.
              Default: integer 200
lcm         : when .TRUE. the code uses the Cromer-Mann coefficients form
              the International Tables of Crystallography to
              compute the atomic form factors, otherwise uses the 
              Doyle-Turner or Smith-Burge parameters.
              Default: logical .FALSE.
flformf     : name of the file in which the atomic form factor is written. 
              The code adds a number to each file name and creates a file
              per atom type.
              Defaults: character 'output_formf.dat'
flpsformf   : name of the postscript file with the atomic form factor.  
              The code adds a number to each file name and creates a file
              per atom type.
              Defaults: character 'output_formf.ps'
\end{verbatim}

\subsection{\color{web-blue}\texttt{what='scf\_ph'}}
With this option the code makes a self-consistent calculation followed
by a phonon calculation. The phonon calculation is controlled by the file
\texttt{ph\_control} and can be at a single {\bf q} point or on a mesh of 
{\bf q} points. 
The different representations are calculated in parallel when several images 
are available. No other input variable is necessary. The outputs of this 
calculation are the dynamical matrices files.
An example of the use of this option can be found in \texttt{example03}. \\
Number of tasks for this option: number of parallelizable tasks of the 
phonon code (smaller but of the order of the number of {\bf q} points times 
$3 N_{at}$, where $N_{at}$ is the number of atoms in the unit cell). \\
\texttt{thermo\_pw} adds to the \texttt{ph.x} code the ability to
compute the complex dielectric constant tensor of insulators as a function 
of a complex frequency for the study of optical properties within 
time-dependent density functional perturbation theory (TD-DFPT). 
The TD-DFPT algorithm uses the Sternheimer equation and a self-consistent loop.
The option is activated in the \texttt{ph.x} input by setting 
\texttt{epsil=.TRUE.} and \texttt{fpol=.TRUE.}, but at variance with
the \texttt{ph.x} code, the frequencies must be specified as complex numbers.
The following additional variables can be put in the input of the \texttt{ph.x}
code, to select the frequency range and the number of frequencies to compute:
\begin{verbatim}
freq_line : if this variable is .TRUE. the FREQUENCY keyword has the number 
            of frequency points and the starting and final frequencies. 
            If .FALSE. the number of frequencies and a list of frequencies 
            are given. The frequencies are complex numbers and are given 
            with a real and an imaginary part (in Ry), without parenthesis.
            Default: .FALSE.

delta_freq : When freq_line is .TRUE. instead of giving the last frequency 
            of the line one can give the distance between two frequency points
            delta_freq as a complex number. The last point of the line is 
            calculated using the number of frequencies nfs and the first 
            frequency. When delta_freq is not zero the last frequency 
            is not used and can be omitted.
            Default: complex, (0.0, 0.0).

\end{verbatim}
When in the input of the phonon code a non zero wave-vector {\bf q}
is specified, the previous options produce the inverse of the dielectric 
constant as a function of the frequency at the wave-vector {\bf q}
(this option can be used both for insulators and metals).
Number of tasks for frequency calculations: 1 (images are not active yet).

\subsection{\color{web-blue}\texttt{what='scf\_disp'}}
With this option the code makes a self-consistent calculation followed by
a phonon dispersion calculation at a fixed geometry. The geometry is given in the 
input of \texttt{pw.x}. The dynamical matrices are used to calculate 
the interatomic force constants and to interpolate the phonon frequencies 
along a path in the Brillouin zone. The path can be generated automatically 
or given in input as in a band structure calculation (see above \texttt{what='scf\_bands'}). The code then 
interpolates the phonon frequencies on a uniform mesh of {\bf q} points and 
computes the phonon density of states with a smearing approach. The density of 
states is used to calculate the harmonic thermodynamic properties: 
vibrational energy, vibrational free energy, vibrational entropy, and 
constant strain heat capacity.
The same thermodynamic quantities are calculated also by direct 
integration over the Brillouin zone and compared in the plots.
Note that presently no interpolation formula is used at low temperatures
so \texttt{thermo\_pw} can not be used to obtain thermodynamic 
properties at very low temperatures. 
The plotted numerical values are per moles in an Avogadro number of unit 
cells. If you need them per mole you have to divide by the number 
of formula units in a unit cell. \\
The input variables that control this option are:
\begin{verbatim}
freqmin_input : minimum frequency for phonon dos plot.
                Default: real determined from phonon frequencies
freqmax_input : maximum frequency for phonon dos plot.
                Default: real determined from phonon frequencies
deltafreq     : frequency interval for phonon dos plot.
                Default: real 1 cm^{-1}
ndos          : number of frequency points in the dos plot.
                Default: determined from previous data
nq1_d, nq2_d, nq3_d : thick mesh for phonon dos calculation.
                Default: integer 192, 192, 192
phdos_sigma   : the smearing used for phonon dos calculation (in cm^-1).
                Default: real 2. cm^-1
after_disp    : if .TRUE. the dynamical matrices are supposed to be already
                available in files in the current directory. This option
                is needed to restart when the outdir directory has been
                erased and ph.x cannot be run without redoing the scf 
                calculation. The exact restart point depends on the files
                already available on the current directory.
                Default: logical .FALSE.
fildyn        : the name of the dynamical matrix file, as would be 
                specified in the input of ph. 
                To be used when after_disp is .TRUE.. 
                Default: character ' '
zasr          : type of acoustic sum rule applied to the ifc.
                Default: character(len=*) 'Simple'
ltherm_dos    : if .TRUE. the thermal properties are calculated from the 
                phonon dos.
                Default: logical .TRUE.
ltherm_freq   : if .TRUE. the thermal properties are calculated from the
                direct integration using the phonon frequencies.
                Default: logical .TRUE.
flfrc         : file where the interatomic force constants are written
                Default: character(len=*) 'output_frc.dat.g1'
flfrq         : file where matdyn writes the interpolated frequencies
                Default: character(len=*) 'output_frq.dat.g1'
fldos         : file where the phonon dos is written
                Default: character(len=*) 'output_dos.dat.g1'
fltherm       : file where the harmonic thermodynamic quantities are written
                Default: character(len=*) 'output_therm.dat.g1'
flpsdisp      : postscript file of the phonon dispersions
                Default: character(len=*) 'output_disp.ps'
flpsdos       : postscript file of the phonon dos
                Default: character(len=*) 'output_dos.ps'
flpstherm     : postscript file of the harmonic thermodynamic quantities
                Default: character(len=*) 'output_therm.ps'
\end{verbatim}
An example of the use of this option can be found in \texttt{example04}. \\
Number of tasks for this option: number of parallelizable tasks of the 
phonon code (smaller but of the order of number of {\bf q} points times 
$3 N_{at}$, where $N_{at}$ is the number of atoms in the unit cell).

\subsection{\color{web-blue}\texttt{what='scf\_elastic\_constants'}}
With this option the code calculates the elastic constants of the solid
at the geometry given as input to \texttt{pw.x}. 
There are three different algorithms that at convergence should give the
same results. In two of them, depending on the Laue class, the code 
calculates the nonzero components of the stress tensor for a set of strains
and obtains the elastic constants from the numerical first derivative
of the stress with respect to strain.
The two algorithms \texttt{standard} and \texttt{advanced} differ
only in the choice of the unit cell. In the \texttt{standard} method the
code applies the strain to the primitive vectors of the unstrained solid
and uses \texttt{ibrav=0} and the strained vectors to compute the stress
tensor.
The \texttt{advanced}  method, available only for selected Bravais lattices, 
try to optimize the calculation by choosing strains for which the number
of needed {\bf k}-points is reduced. Moreover it identifies the 
Bravais lattice of the strained solid and recalculates the primitive
vectors with the conventions of \qe. When available this should be the
most efficient method.
The third algorithm is called \texttt{energy}.
Using the \texttt{energy} algorithm the elastic constants are calculated 
from the quadratic fit of the total energy as a function of strain
without computing stress. This option is available only for selected 
lattices and usually requires more independent strains. It can be
used when stress calculation is not implemented in \qe.\\
For all methods the number of strains is \texttt{ngeo\_strain}
for each independent strain. 
For each strain, the code relaxes the ions to their equilibrium 
positions if \texttt{frozen\_ions=.FALSE.} or keeps them
in the strained positions if \texttt{frozen\_ions=.TRUE.}. 
Note that elastic constant calculations with \texttt{frozen\_ions=.FALSE.}
might require smaller force convergence threshold than standard calculations. 
The default value of \texttt{forc\_conv\_thr} must be changed in the 
\texttt{pw.x} input.
At finite pressure all methods give the elastic constants that
relate linearly stress and strain. \\
The input variables that control this option are:
\begin{verbatim}
frozen_ions: if .TRUE. the elastic constants are calculated 
             keeping the ions frozen in the strained positions. 
             Default: logical .FALSE.
ngeo_strain: the number of strained configurations used to calculate each
             derivative. 
             Default: integer 4 ('standard' and 'advanced'), 6 ('energy')
elastic_algorithm: 'standard', 'advanced', or 'energy'. See discussion above.
             Default: character 'standard' 
delta_epsilon: the interval of strain values between two geometries.
             To avoid a zero strain geometry that might have a
             different symmetry ngeo_strain must be even.
             Default: real 0.005
epsilon_0:   a minimum strain. For small strains the ionic relaxation 
             routine requires a very small threshold to give the correct 
             internal relaxations and sometimes fail to converge. In this 
             case you can increase delta_epsilon, but if delta_epsilon 
             becomes too large you can reach the nonlinear regime. In 
             this case you can use a small delta_epsilon and a minimum 
             strain. (To be used only for difficult systems).
             Default: real 0.0
poly_degree: degree of the polynomial used to interpolate stress or energy.
             ngeo_strain must be larger than poly_degree+1
             Default: 3 ('standard', 'advanced', 2 if ngeo_strain < 6), 
                      4 ('energy', 3 if ngeo_strain < 6).
fl_el_cons:  the name of the file that contains the elastic constants
             Default: character(len=*) 'output_el_con.dat'
\end{verbatim}
The three algorithms are equivalent only at convergence both with
{\bf k}-point sampling and with the kinetic-energy cut-off, but 
large differences between the elastic constants obtained with the 
\texttt{standard} and \texttt{advanced} algorithms might point to 
insufficient {\bf k}-point sampling. Large differences between the 
elastic constants obtained with the \texttt{energy} algorithm with respect 
to the other two might point to insufficient kinetic-energy cut-off. \\
Number of tasks for this option: \texttt{ngeo\_strain} times the number of
independent strains. \\

Using the elastic constants tensor the code can calculate and print
a few auxiliary quantities:
the bulk modulus, the poly-crystalline averages of the Young modulus,
of the shear modulus, and of the Poisson ratio. Both the Voigt and the
Reuss averages are printed together with the Hill average.
The Voigt-Reuss-Hill average of the shear modulus and of the bulk modulus are 
used to compute average sound velocities. The average of the Poisson ratio and
the bulk modulus allow the estimation of the Debye 
temperature. The Debye temperature is calculated also with the exact 
formula evaluating the average sound 
velocity from the angular average of the sound velocities calculated 
for each propagation direction solving the Christoffel wave equation.
The exact Debye temperature is used within the Debye model to calculate the
Debye's vibrational energy, free energy, entropy and constant strain heat
capacity. These quantities are plotted in a postscript file as a function
of temperature.

%\subsection{\color{web-blue}\texttt{what='scf\_piezoelectric\_tensor'}}
%With this option the code calculates the piezoelectric tensor 
%($g_{\alpha,m}$ $\alpha=1,3,\ m=1,6$) of the solid.
%Depending on the point group, it calculates the polarization of the
%solid for a set of strains of the lattice. For each strain, it relaxes 
%the ions to their equilibrium positions when \texttt{frozen\_ions=.FALSE.} 
%or keep them in the strained positions when \texttt{frozen\_ions=.TRUE.}. 
%Finally it computes the piezoelectric tensor from the numerical derivatives 
%of the polarization with respect to strain.
%The number of strains used to compute each derivative 
%is \texttt{ngeo\_strain}.\\
%The input variables
%that control this option are the same as those used for the elastic constant:
%\begin{verbatim}
%frozen_ions: if .TRUE. the piezoelectric tensor is calculated 
%             keeping the ions frozen in the strained positions. 
%             Default: logical .FALSE.
%ngeo_strain: the number of strained configurations used. 
%             Default: integer 4
%nppl:        the number of k-points per line in Berry phase calculations
%             Default: integer 51
%delta_epsilon: the interval of strain values between two geometries.
%               To avoid a zero strain geometry that might have a
%               different symmetry ngeo_strain must be even.
%               Default: real 0.002
%\end{verbatim}
%If a file with the elastic constants is found on disk, the code computes also
%the direct piezoelectric tensor $d_{\alpha,m}$ which describes
%the polarization induced by an external stress, as $d_{\alpha,m}=
%\sum_n g_{\alpha,n} C_{nm}^{-1}$. All these quantities are calculated 
%for a vanishing electric field. \\
%Number of tasks for this option: \texttt{ngeo\_strain} times the number of
%independent strains. \\
%This feature is still incomplete and experimental.
%
\subsection{\color{web-blue}\texttt{what='mur\_lc'}}
With this option the code runs several self-consistent calculations
at different geometries. The runs can be done in parallel when several images 
are available. This option has two working modes controlled by the 
logical variable \texttt{lmurn}. When \texttt{lmurn=.TRUE.} the total energy as 
a function of the volume is interpolated by a Murnaghan equation and 
a plot of the energy as a function of the volume and of the pressure 
as a function of the volume is produced. The volume is changed by 
changing only \texttt{celldm(1)}. \texttt{celldm(2)...celldm(6)} remain fixed
at the values given as input of \texttt{pw.x}.
When \texttt{lmurn=.FALSE.} the energy is calculated in a uniform
grid of parameters composed of \texttt{ngeo(1)} $\times$ \texttt{ngeo(2)...}
$\times$ \texttt{ngeo(6)} points.
The energies are fitted with a quadratic or quartic polynomial of
$N_k$ variables, where $N_k$ is the number of independent crystal
parameters for the given crystal system. A plot of the energy as 
a function of the lattice constant is produced for cubic systems.
For solids of the hexagonal, tetragonal, and trigonal systems
contour plots of the energy as a function of the two crystal parameters
($a$ and $c/a$ or $a$ and $\cos\alpha$)
are plotted. For orthorhombic systems contour plots of the energy as a function
of $a$ and $b/a$ are plotted for each value of $c/a$. 
Presently no graphical tool is implemented to plot the energy
for monoclinic and triclinic crystal systems. 
When \texttt{lmurn=.FALSE.} the bulk modulus
is not calculated. To obtain it, you can calculate the elastic constants at the
minimum geometry (see the option \texttt{what='mur\_lc\_elastic\_constants'}). 
With this option the pressure control is active. You can specify a 
finite pressure and the enthalpy is minimized instead of the
energy. Note however that if the minimum is distant from the starting
configuration its associated error can be large, larger for the
quadratic than for the Murnaghan interpolation. For this
reason the present option should be used starting from the minimum found by
\texttt{pw.x} using the \texttt{vc-relax} option and the pressure 
should not be too different from the pressure used for \texttt{vc-relax}.\\
This option can be controlled by the following variables:
\begin{verbatim}
ngeo(1),...,ngeo(6) : the number of geometries to use for each celldm parameter.
             The lattice constant of these geometries is calculated from the
             input of pw.x. celldm(1),...,celldm(6) of this input is used 
             for the central geometry. For the others celldm(1),...,celldm(6),
             are changed in steps of step_ngeo(1),...,step_ngeo(6). 
             ngeo(1) must be odd. Only the values of celldm relevant for
             each Bravais lattice are actually changed.
             Default: integer 1,1,1,1,1,1 for what=scf_*, 9,1,1,1,1,1 for 
             what=mur_lc_* and lmurn=.TRUE. or for cubic systems, 5 on all 
             the relevant celldm parameters when lmurn=.FALSE. and the system
             is not cubic.
step_ngeo(1),...,step_ngeo(6) : The step between the lattice constants at 
             different geometries. step_ngeo(1) is, in atomic units, the change
             of a, step_ngeo(2), step_ngeo(3) are dimensionless and are the
             changes of the ratios b/a, c/a, step_ngeo(4), step_ngeo(5), &
             step_ngeo(6) are the changes in degree of the angles alpha,
             beta, and gamma. The cosine of the angle is calculated by the
             program.
             Default: real 0.05 a.u., 0.02, 0.02, 0.5, 0.5, 0.5
lmurn       : if .TRUE. the Murnaghan fit is done. Only ngeo(1) values of
             the energy are fitted, the other values of ngeo are not used. 
             if .FALSE. use a quadratic or quartic function to interpolate 
             the energy as a function of all celldm parameters. The number of 
             self-consistent calculations is ngeo(1) x ngeo(2) x ngeo(3)
             x ngeo(4) x ngeo(5) x ngeo(6). In this case only the 
             minimum energy and the optimal celldm are given in output. 
             Default: .TRUE. 
reduced_grid: if .TRUE. the number of energies used for the quadratic fit 
             are reduced using the variables start_geo and jump_geo. 
             This option is used only when lmurn=.FALSE.
             Default: logical .FALSE.
start_geo  : The first geometry of the reduced_grid.
             Default: integer 1
jump_geo   : jump_geo-1 geometries are not computed when reduced_grid
             is .TRUE. For instance when jump_geo=3 and start_geo=1 the
             computed geometries are 1,4,7,10 etc.
             Default: integer 1
show_fit   : if .TRUE. show the contour plot of the fitted energy instead
             of the energy. Used by default when reduced_grid is .TRUE.
             Default: logical .FALSE.
vmin_input : minimum volume for the plot of the energy as a function of volume.
             Default: real 0.98 times the volume of the first geometry.
vmax_input : maximum volume for the plot of the energy as a function of volume.
             Default: real 1.02 times the volume of the last geometry.
deltav     : distance between two volumes in the plot of the energy as a 
             function of the volume.
             Default: real calculated from nvol.
nvol       : number of volumes in Murnaghan plot
             Default : integer 51
lquartic   : if .TRUE. fit the energy with a quartic polynomial.
             Default : logical .TRUE.
lsolve     : choose the algorithm used to fit the quartic polynomial parameters.
             Allowed values:
             1 explicitly minimize chi^2
             2 Use the QR algorithm to minimize chi^2 (lapack routine dgels)
             3 Use the SVD algorithm to minimize chi^2 (lapack routine dgelss)
             Default: integer 2
flevdat    : file where the Murnaghan equation is written. The results of the
             Murnaghan fit are then written in flevdat.ev.out.
             Default: character(len=*) 'output_ev.dat'
flpsmur    : postscript file of the Murnaghan plot
             Default: character(len=*) 'output_mur.ps'
ncontours  : the number of contours in the energy plot. These levels can
             be determined automatically by the code or defined by the user.
             The energy levels can be defined after the INPUT_THERMO namelist 
             but before the path, as a list
             energy_level(1)      color(1)
             ...
             energy_level(ncontours)   color(ncontours) 
             Color is a string of the type color_red, color_green, etc.
             The list of available colors is at the beginning of each gnuplot
             script.
             energy_level is in Ry units.
             Default: integer 9
flenergy   : name of the file that contains the energy in a form that
             can be used by gnuplot to make contour plots.
             Defaults: character(len=*) 'output_energy'
flpsenergy : file with the contour plots of the energy as a function of the
             crystal parameters.
             Default: character(len=*) 'output_energy.ps'
\end{verbatim}
An example of the use of this option can be found in \texttt{example05}.\\
Number of tasks for this option: \\
\texttt{ngeo(1)} when \texttt{lmurn=.TRUE.}, \\
\texttt{ngeo(1)}$\times$\texttt{ngeo(2)}$\times$\texttt{ngeo(3)}$\times$\texttt{ngeo(4)}$\times$\texttt{ngeo(5)}$\times$\texttt{ngeo(6)} when 
\texttt{lmurn=.FALSE.} and \texttt{reduced\_grid=.FALSE.}\\
\texttt{ngeo(1)}$+$\texttt{(ngeo(2)-1)}$+$\texttt{(ngeo(3)-1)}$+$\texttt{(ngeo(4)}-1)$+$\texttt{(ngeo(5)-1)}$+$\texttt{(ngeo(6)-1)} when 
\texttt{lmurn=.FALSE.} and \texttt{reduced\_grid=.TRUE.}.


\subsection{\color{web-blue}\texttt{what='mur\_lc\_bands'}}
With this option the code computes the band structure at the geometry 
that minimizes the energy. See the options 
\texttt{what='scf\_bands'} and \texttt{what='mur\_lc'} for a list of 
the variables that control these two options. 
With this option the pressure control is active. You can specify a
finite pressure and the enthalpy is minimized instead of the
energy to find the equilibrium geometry. The bands are calculated
at the geometry that corresponds to the external pressure.
An example of the use of this option can be found in \texttt{example06}. \\
Number of tasks for this option: see \texttt{what='mur\_lc}.

\subsection{\color{web-blue}\texttt{what='mur\_lc\_ph'}}
This option is similar to \texttt{what='scf\_ph'} but the phonon calculation
is made at the geometry that minimizes the energy.
See the options \texttt{what='scf\_ph'} and \texttt{what='mur\_lc'} for a
list of the variables that control these two options. 
With this option the pressure control is active. You can specify a
finite pressure and the enthalpy is minimized instead of the
energy to find the equilibrium geometry. The phonons are calculated
at the geometry that corresponds to the external pressure.
An example of the use of this option can be found in \texttt{example07}. \\
Number of tasks for this option: Maximum between the number of tasks 
needed by the \texttt{what='mur\_lc'} option and the number
of tasks of the phonon code (see above the option \texttt{what='scf\_ph'}).

\subsection{\color{web-blue}\texttt{what='mur\_lc\_disp'}}
This option is similar to \texttt{what='scf\_disp'} but the phonon calculation
is made at the geometry that minimizes the energy.
See the options \texttt{what='scf\_disp'} and \texttt{what='mur\_lc'} for a
list of the variables that control these two options. 
With this option the pressure control is active. You can specify a
finite pressure and the enthalpy is minimized instead of the
energy to find the equilibrium geometry. The phonons are calculated
at the geometry that corresponds to the external pressure.
An example of the use of this option can be found in \texttt{example08}. \\
Number of tasks for this option: Maximum between the number of tasks  
needed by the \texttt{what='mur\_lc'} option and the number
of tasks of the phonon code (see above the option \texttt{what='scf\_ph'}).

\subsection{\color{web-blue}\texttt{what='mur\_lc\_elastic\_constants'}}
As \texttt{what='scf\_elastic\_constants'} but the calculation is made at the
geometry that minimizes the energy. The energy minimization is 
done as described for \texttt{what='mur\_lc'}. 
With this option the pressure control is active. You can specify a
finite pressure and the enthalpy is minimized instead of the
energy to find the equilibrium structure. Note however that if the 
minimum is distant from the starting configuration its associated error 
can be large, larger for the quadratic than for the Murnaghan interpolation. 
An example of the use of this 
option can be found in \texttt{example13}. \\
One additional input variable controls this option.
\begin{verbatim}
do_scf_relax : if .TRUE. the code makes a self-consistent relax calculation at
               the equilibrium geometry to find the optimized atomic 
               coordinates, before computing the elastic constants.
               This step is needed only for solids that have internal
               degrees of freedom in the unstrained configuration. 
               If .FALSE. the coordinates of the input geometry are strained 
               uniformly to the equilibrium geometry.
               Default: logical .FALSE. 
\end{verbatim}
Number of tasks for this option: Maximum between the number of tasks
needed by the \texttt{what='mur\_lc'} option and the number of tasks
needed for the \texttt{what='scf\_elastic\_constants'} option.

%\subsection{\color{web-blue}\texttt{what='mur\_lc\_piezoelectric\_tensor'}}
%As \texttt{what='scf\_piezoelectric\_tensor'} but the calculation is made at the
%geometry that minimizes the energy. The
%energy minimization is made as described for \texttt{what='mur\_lc'}. \\
%This feature is still incomplete and experimental.

\subsection{\color{web-blue}\texttt{what='mur\_lc\_t'}}
With this option the code calculates the anharmonic 
properties within the quasi-harmonic approximation. 
The outputs of the code are the values of crystal parameters 
(\texttt{celldm}) as a function of temperature. This calculation is done 
by computing the phonon dispersions on all the geometries specified as in
\texttt{what='mur\_lc'} (or on a subset of these geometries) and 
minimizing the Helmholtz free energy.
Separate plots of the phonon dispersions are obtained for all the 
calculated geometries.
For each geometry the code produces also plots of the phonon density
of states and of the harmonic thermodynamic quantities.
From \texttt{celldm} as a function of temperature the code computes the thermal
expansion tensor, the volume, and the volume thermal expansion as a 
function of temperature. 
The frequencies at all the calculated geometries are interpolated
by quadratic or quartic polynomials of the crystal parameters
and can be shown at crystal parameters given in input or at those that
correspond to a given temperature. The interpolated frequencies  
are shown on the same path used for the phonon dispersions.
In addition to the frequencies the code produces also several plots of
the derivatives of the frequencies with respect to the crystal parameters
multiplied by the crystal parameters. \\
When the Murnaghan equation is used to interpolate the
Helmholtz free energy (\texttt{lmurn=.TRUE.}), in addition to the volume, 
the bulk modulus and the pressure derivative of the bulk modulus are 
plotted as a function of 
temperature. Moreover the isobaric heat capacity, the isoentropic 
bulk modulus, and the average Gr\"uneisen parameter are calculated as 
a function of temperature.
The mode Gr\"uneisen parameters are calculated with cubic interpolations of the
phonon frequencies. Using the variable \texttt{with\_eigen} one can 
calculate these parameters as derivatives of the phonon frequencies 
(default) or as expectation values of the derivatives of the dynamical 
matrix on the central geometry eigenvectors (might require a lot of RAM). 
The mode Gr\"uneisen parameters are used to calculate the volume
thermal expansion and the result is compared with the volume thermal expansion
derived from the numerical derivative of the equilibrium volume obtained
from the minimization of the Helmholtz free energy. 
When the Helmholtz free energy is interpolated with a quadratic or cubic
polynomium (\texttt{lmurn=.FALSE.}), by default, the code computes only 
the temperature dependence of the lattice parameters and of the volume, the 
volume thermal expansion, and the thermal expansion tensor. However if a file 
with the elastic constants is found in the working directory 
and \texttt{lb0\_t=.FALSE.} the bulk modulus is calculated and 
assumed independent from the temperature so that also the isobaric specific 
heat, the isoentropic bulk modulus, and the average Gr\"uneisen parameter 
are calculated as a function of temperature. The derivatives of the 
frequencies are used to calculate the thermal expansion tensor which 
is compared with that obtained from the numerical derivatives of the 
crystal parameters obtained from the minimization of the Helmholtz free energy.
If many files with the elastic constants, one for each geometry, as
produced with the option \texttt{elastic\_constants\_t}, are found
in the working directory, the bulk modulus and the elastic constants
are computed as a function of temperature within the "quasi-static
approximation" and are used
to calculate the other thermodynamic properties. In addition to the
quantities plotted for cubic solids, the code plots also the elastic
constants as a function of the temperature and the difference between
elastic constants calculated at constant temperature or at constant
entropy. These possibilities are implemented only for cubic, tetragonal, 
hexagonal, trigonal, and orthorhombic systems. \\
With this option the pressure control is active. You can specify a
finite pressure and the Gibbs energy is minimized instead of the
Helmholtz free energy. Note however that if the minimum is distant from 
the starting configuration its associated error can be large, larger for the
quadratic than for the Murnaghan interpolation. \\
The input variables that control these plots are those described in the option
\texttt{what='mur\_lc'} and \texttt{what='mur\_lc\_disp'} in addition to the 
following:
\begin{verbatim}
grunmin_input : minimum y coordinate for the Gruneisen parameter plot.
                Default: real, calculated from the Gruneisen parameters.
grunmax_input : maximum y coordinate for the Gruneisen parameter plot.
                Default: real, calculated from the Gruneisen parameters.
volume_ph     : The frequencies and Gruneisen parameters interpolated 
                at this volume are plotted on a postscript file. 
                When volume_ph=0.0 the volume is calculated from temp_ph.
                This option is available only for cubic solids. Otherwise
                use celldm_ph.
                Default : 0.0 (in (a.u.)**3)
celldm_ph     : The frequencies and Gruneisen parameters interpolated 
                at this crystal parameters are plotted on a postscript file. 
                If this is 0.0 the celldm are calculated from temp_ph.
                To have accurate Gruneisen parameters and interpolated
                frequencies set the central geometry as close as possible 
                to celldm_ph. When all nstep are odd, the central geometry 
                is the one given in the input of pw.x.
                Default : 0.0 (celldm(1) in a.u., celldm(2-6) dimensionless)
temp_ph       : The frequencies and Gruneisen parameters interpolated at 
                the volume (cubic systems) or at celldm (anisotropic systems)
                that corresponds to this temperature are plotted 
                on a postscript file (only when volume_ph=0.0 or 
                celldm_ph(1)=0.0).
                Default : real tmin (in K)
with_eigen    : if .TRUE. use the eigenvectors of the dynamical matrix to
                calculate the Gruneisen parameters. For anisotropic solids
                this option cannot be changed. It is .TRUE. for the plotted
                Gruneisen bands, but Gruneisen parameters are not used to
                calculate anharmonic properties.
                Default: logical .FALSE. 
lquartic_ph   : if .TRUE. fit the vibrational free energy with a fourth
                order polynomial, otherwise with a second order polynomial
                Default: logical .FALSE.
lv0_t         : if .TRUE.  the calculation of the thermal expansion with 
                Gruneisen parameters uses the equilibrium volume as a 
                function of temperature computed from the free energy 
                minimization, otherwise the volume at T=0 K.
                Default: logical .TRUE.
lb0_t         : if .TRUE. the calculation of the thermal expansion with 
                Gruneisen parameters uses the bulk modulus as a function of 
                temperature computed from the free energy minimization, 
                otherwise the bulk modulus computed at T=0 K. For anisotropic
                solids only lb0_t=.FALSE. is implemented.
                Default: logical .TRUE.
flpgrun       : file where the values of the Gruneisen parameters are written. 
                Default: character(len=*) 'output_pgrun.dat'
flgrun        : file where the Gruneisen parameters in a plotable form are
                written.
                Default: character(len=*) 'output_grun.dat'
flpsgrun      : name of the postscript file with the Gruneisen parameters plot.
                The frequencies are written in the file with the same name 
                plus the string _freq.
                Default: character(len=*) 'output_grun.ps'
flanhar       : file where the anharmonic thermodynamic quantities are written.
                Default: character(len=*) 'output_anhar.dat'
flpsanhar     : postscript file of the anharmonic quantities.
                Default: character(len=*) 'output_anhar.ps'
fact_ngeo(1)...fact_ngeo(6) : With these factors the vibrational free energy 
                is interpolated using a smaller number of geometries 
                with respect to the total energy. The phonons are always 
                calculated at geometry 1, then fact_ngeo(i)-1 geometries 
                are not calculated and so on. The last calculated geometry 
                must be ngeo(i). This happens when fact_ngeo(i) divides
                ngeo(i)-1. For even ngeo(i), fact_ngeo(i) must be 1.
                For odd ngeo(i) the following table gives a few examples
                ngeo   fact_ngeo      calculated geometries
                3         2            1,3
                5         2            1,3,5
                7         2            1,3,5,7
                7         3            1,4,7
                9         2            1,3,5,7,9
                9         4            1,5,9
                11        2            1,3,5,7,9,11
                11        5            1,6,11
                Defaults: integer 1,1,1,1,1,1
                This option is not active when reduced_grid=.TRUE.
\end{verbatim}
The output files corresponding to different geometries can be identified
by the presence of the letters \texttt{g1}, \texttt{g2}, ... in the filename.
To exploit all the features of this option please write the dynamical matrices
in \texttt{.xml} format (using a \texttt{fildyn} with the \texttt{.xml}
extension).
An example of the use of this option can be found in \texttt{example09}. \\
Number of tasks for this option: Maximum between the the number of tasks  
needed by the \texttt{what='mur\_lc'} option and the number
of tasks of the phonon code (see above the option \texttt{what='scf\_ph'}). \\

\subsection{\color{web-blue}\texttt{what='elastic\_constants\_t'}}

With this option the code computes the elastic constants for a set of
geometries, the same geometries that are chosen with the option 
\texttt{what='mur\_lc\_t'}, and writes them on separate files, one for
each geometry. The temperature dependence of the elastic constants is
not computed by this option. In order to plot the elastic constants
as a function of temperature it is necessary to make a calculation
with \texttt{what='mur\_lc\_t'} having on files the elastic constants
calculated with the present option. In this case \texttt{thermo\_pw}
will be able to calculate the anharmonic properties using temperature
dependent elastic constants and bulk moduli obtained by interpolating 
(within the "quasi-static approximation") the elastic constants computed 
by this option at the crystal parameters 
found at each temperature from the minimization of the free energy. 
The parameters that control this option are the same as those used with  
\texttt{what='mur\_lc'} for the choice of the geometries and with
\texttt{what='scf\_elastic\_constants'} for the control of the elastic
constants. Note however that only \texttt{elastic\_algoritm='standard'}
is presently working with this option.

%\subsection{\color{web-blue}\texttt{what='scf\_polarization'}}
%With this option the code calculates the spontaneous polarization.
%The code makes a self-consistent calculation followed by three Berry
%phase calculations in which the polarization is calculated in the
%direction parallel to the three primitive reciprocal lattice vectors and 
%prints the spontaneous polarization in Cartesian coordinates.
%The input variable that controls this option is:
%\begin{verbatim}
%nppl: the number of k points per string. In the perpendicular plane the
%      number of k-point is that given as input of the pw.x code.
%      Default: integer 51
%\end{verbatim}
%This feature is still incomplete and experimental.
%
%\subsection{\color{web-blue}\texttt{what='mur\_lc\_polarization'}}
%As \texttt{what='scf\_polarization'} but the calculation is made at the
%geometry that corresponds to the minimum of the Murnaghan equation. The
%Murnaghan minimization is made as described for \texttt{what='mur\_lc'}.\\
%This feature is still incomplete and experimental.
%

\newpage
\section{\color{coral}Restarting an interrupted run}

There are several situations that might require the restart of the \thermo\ 
code. We must distinguish two different cases:
\thermo\ stopped while running \qe\ routines, because the code reached
the maximum cpu time or because some external event stopped the run,
or \thermo\ stopped after doing some post-processing task. This second 
case comprises also the normal termination of
\thermo\ and the necessity to change some details of the plot rerunning
the post-processing tools without redoing the \qe\ calculations.
Support for the first case is based on the recover features provided by 
\qe\ routines. This restarting method needs files in the \texttt{outdir}
directory and works only with one image. In this case \thermo\ behaves as \qe\ 
except for the fact that \texttt{max\_seconds} in the input of \texttt{pw.x} or
of \texttt{ph.x} is not active. To run \texttt{thermo\_pw}
for a fixed number of seconds \texttt{max\_seconds} must be set in the
\texttt{THERMO\_CONTROL} namelist. If the code stopped inside \texttt{pw.x},
\texttt{restart\_mode} must be set to \texttt{'restart'} in the input of 
\texttt{pw.x} while if the code stopped inside \texttt{ph.x} routines
\texttt{recover} must be set to \texttt{.TRUE.} in the input of \texttt{ph.x}.

Instead when you are running \texttt{thermo\_pw} with several images the option 
\texttt{max\_seconds} is disabled, \texttt{restart\_mode} must 
be set to \texttt{'from\_scratch'} in the input of \texttt{pw.x} and
\texttt{recover} must be set to \texttt{.FALSE.} in the input of \texttt{ph.x}.
This does not mean that \texttt{thermo\_pw} cannot recover any of the previous
calculations but a few calculations must be repeated. In general \thermo\ 
does not recalculate the quantities contained in files that are already 
in the working directory. If you are running \thermo\ in tasks that do 
self-consistent calculations followed by phonon calculations, the 
self-consistent calculation is not repeated 
when completed and also completed phonon calculations at a given
geometry for which the dynamical matrices are available are
not redone. It is possible to stop \texttt{thermo\_pw} after the 
calculation of the phonon dispersions for a fixed number of geometries 
by setting the input variable \texttt{max\_geometries} in the 
\texttt{THERMO\_CONTROL} namelist. 
Each routine checks if a file with the same name
as the file that it would produce is already in the working directory,
and if this happens, it reads its content and returns. This feature cannot be
disabled from input. In order to recalculate a given quantity, just remove
the file that contains it from the working directory.

In an anharmonic calculation, if you have already all the dynamical 
matrices for all the geometries and you do not have any more the
\texttt{outdir} directory, it is possible to skip entirely
the phonon calculations and the reading of the files produced by
\texttt{pw.x} by setting the variable
\texttt{after\_disp=.TRUE.} and giving the name of the dynamical matrices file
using the variable \texttt{fildyn} in the \texttt{THERMO\_CONTROL} namelist. 
In this case \thermo\ can compute the anharmonic properties with a 
different set 
of temperatures, or with a different sampling on the phonon frequencies, 
etc.. You need to erase the output files that contain
the phonon dos, or the thermal properties from a previous calculation, keeping 
the dynamical matrices files and the \texttt{restart} directory and 
rerun \thermo.
Similarly if the files containing the bands energy eigenvalues are already
in the working directory, it is possible to set the input variable
\texttt{only\_bands\_plot} to change the bands plot without redoing the
bands calculation. Note however that in this case it is not possible 
to change the Brillouin zone path. \\

\newpage
\section{\color{coral}Tools}

The directory \texttt{tools} contains a few tools that can be useful to build
structures of solids, surfaces, ribbons, and nanowires. Moreover it 
contains some miscellaneous codes that give additional information
on the internal conventions of \thermo\ or further process its output. 
Currently it contains the following programs:

\begin{itemize}

\item \texttt{bravais\_lattices.x} tests the
module \texttt{lattices.f90} of the library. Presently it can read three
primitive lattice vectors of a Bravais lattice and find the \texttt{ibrav}
code and the \texttt{celldm} parameters of the input lattice. It
can also read two sets of primitive vectors and decide if they describe
the same Bravais lattice. In the positive case it gives the orientation
of one lattice with respect to the other.

\item \texttt{crystal\_point\_group.x} is a crystal point group calculator.
It can give several information about the crystallographic point groups,
such as the list of symmetry operations, the product of two rotation
matrices, the product table, the class structure, the character tables of 
the irreducible representations of the point group and of the double
point group and the projective representations of the point group. It
gives the list of subgroups and supergroups of a given group and 
the compatibility tables of a given group with its subgroups. It can 
also decompose the Kronecker product representations. Finally it
can list the cojugate groups.

\item \texttt{elastic.x} reads the elastic constants of a
solid and computes the elastic compliances, the bulk modulus, and a few
poly-crystalline averages. It uses the \thermo\ library so the output is the
same. 

\item \texttt{epsilon\_tpw.x} generalizes the routine
\texttt{epsilon.f90} of the \qe\ distribution. It calculates the 
complex dielectric
constant of a solid as a function of the frequency for independent electrons
using the LDA or GGA eigenvalues. It is limited to insulators, but supports
norm-conserving, ultrasoft, and PAW pseudopotentials. It supports both scalar
relativistic and fully relativistic pseudopotentials and it uses the point
group symmetry of the solid to reduce the number of {\bf k}-points.

\item \texttt{gener\_nanowire.x} reads a two dimensional (2D)
Bravais lattice index and atomic coordinates an generates a sheet of type
$(m,n)$. A sheet contained between the two vectors 
{\bf C}$ = m$ {\bf a}$_1 + n ${\bf a}$_2$ and {\bf T} $= p ${\bf a}$_1 + q ${\bf a}$_2$
can be also generated and wrapped about {\bf C} in a nanotube form 
({\bf a}$_1$ and {\bf a}$_2$ are the primitive lattices of the 2D Bravais 
lattice). 
For lattices that allow it, $p$ and $q$ can be determined automatically so that 
{\bf T} is perpendicular to {\bf C}.

\item \texttt{gener\_2d\_slab.x} reads a two dimensional 
Bravais lattice index and atomic coordinates an generates an infinite ribbon
perpendicular to {\bf G} $= m$ {\bf b}$_1 + n ${\bf b}$_2$, where 
{\bf b}$_1$ and {\bf b}$_2$ are the primitive reciprocal lattice vectors 
of the 2D Bravais lattice. The number of rows of the ribbon, and the number of
atoms per row are given as input variables.

\item \texttt{gener\_3d\_slab.x} reads a three dimensional
Bravais lattice index and atomic coordinates and generates an infinite slab
perpendicular to {\bf G} $= m ${\bf b}$_1 + n ${\bf b}$_2 + o ${\bf b}$_3$, 
where {\bf b}$_1$, {\bf b}$_2$ and {\bf b}$_3$ are the primitive reciprocal 
lattice
vectors of the Bravais lattice. The number of layers of each slab, and the
size of the surface unit cell are given as input parameters.

\item \texttt{hex\_trig.x} reads the values of $a$ and $c$
of the conventional hexagonal cell of a rhombohedral lattice (in \AA ngstrom),
and gives as output the size $a_r$ (in a.u.) and the cosine of the angle
$\alpha$ of the rhombohedral cell. This information can
be written in the input of \texttt{pw.x} for this type of cells. It
is used to convert the structural information contained in a CIF file
to the \texttt{pw.x} input.

\item \texttt{kovalev.x} writes the correspondence between point group
symmetry operations defined in the Kovalev tables and those used by \qe.

\item \texttt{plot\_sur\_states.x} reads the dump file produced
by \thermo\ in a \texttt{what='scf\_2d\_bands'} calculation that
contains the planar averages of the all the states, and plots
the states with the {\bf k} point and the band numbers requested in input.

\item \texttt{space\_groups.x} gives several information on space groups.
It can give the names of the space group given the number reported in
the International Tables for Crystallography (ITA), or
the number given one of the names, translate the names between different
editions of the ITA tables or the Sh\"onflies name. It gives the list 
of coset representatives of each space group and the list of symmorphic 
space groups.

\item \texttt{supercell.x} reads a three dimensional
Bravais lattice index and the atomic coordinates of the atoms inside a unit
cell and produces a supercell with $n1 \times n2 \times n3$ cells of the original
unit cell. For centered cells there is the option to consider $n1$, $n2$,
and $n3$ in the conventional or in the centered cell.
The input unit cell can be specified also by giving the space group and
the coordinates of the inequivalents atoms.
It can be useful to study defects or to calculate all the 
atomic positions starting from the space-group and the 
inequivalents positions.
It is also possible to give the input Bravais lattice by using
\texttt{ibrav=0} and the three principal vectors. In this case,
before generating the supercell, the code rotates the Bravais
lattice so that it has the same orientation of the vectors described
in the \texttt{thermo.pdf} guide.

\item \texttt{test\_colors.x} produces a postscript file with the 
\texttt{gnuplot} colors that can be used in the plots.   

\item \texttt{translate.x} reads a set of atomic positions and a translation
vector and translates the atomic positions. It can read also a rotation matrix 
and roto-translate the atomic positions.

\end{itemize}

For a detailed description of the input variables please look at the beginning 
of the \texttt{fortran} sources of each code.

\newpage
\section{\color{coral}Examples, inputs, pseudo\_test, space\_groups}

The directories \texttt{examples}, \texttt{inputs}, \texttt{pseudo\_tests}, 
\texttt{space\_groups} contain a set
of examples that can be studied in order to learn how to use the \thermo\ 
package. The \texttt{examples} directory contains inputs that run in a few
seconds but do not give converged results. These examples can be studied 
to see how the \thermo\ code works in the different cases. The
reference directory of each example contains all the output
files produced by the run. A one-to-one comparison with the output
produced by running the example script is however not possible due to the
asynchronous nature of the runs. The plotted physical quantities however
should be the same. \\
The directory \texttt{inputs} contains a set
of realistic inputs and reasonably converged results. Not all
output files are reported in the reference directory of each run.
The \texttt{inputs} examples are divided according to the structure type
and many material properties are calculated for each structure.
This directory can be seen as a gallery of the results that can be
obtained by the \thermo\ code, or as a source of information for the
construction of a particular input geometry. \\
The directory \texttt{pseudo\_test} contains a set of inputs that can
be used to test a pseudopotential library. It illustrates how to use \thermo\ 
for high-throughput calculations. \\
The directory \texttt{space\_groups} contains a collection of structures
ordered by the space group number that are used to test the space
groups routines. These inputs are also examples of the use of the
keyword \texttt{space\_group} and of the Wyckoff positions to give the
atomic coordinates in the \texttt{pw.x} input. You can also use these
strutures for your calculations, but note that the cut-off energies and
the {\bf k}-point meshes used in these inputs are not converged.

\newpage
\section{\color{coral}Documentation}

In addition to this user's guide, this directory contains the following
documents:

\begin{itemize}

\item
\texttt{tutorial.pdf} : a short guide that indicates where to find the
information needed to perform a given task.

\item
\texttt{point\_groups.pdf} : a description of the crystallographic point 
groups, character tables of the irreducible representations of groups and 
of the double point groups and tables of the projective representations, 
for the interpretation of the color codes in the band and phonon dispersion 
plots.

\item
\texttt{thermo.pdf} : some notes on the thermodynamic expressions implemented
in \texttt{thermo\_pw}.

\item
\texttt{developer\_guide.pdf} : some notes on the internal logic of
\texttt{thermo\_pw}.

\end{itemize}

\newpage
\section{\color{coral}Color codes}

In this section we briefly summarize the color codes of some of the figures
that can be obtained from \texttt{thermo\_pw}.

\begin{itemize} 
\item
Total energy versus kinetic energy. This is a single figure of the total
energy versus wave-functions kinetic energy cut-offs. When the test
requires several charge density cut-offs there is a different curve
for each charge density cut-off. The curve corresponding to the lowest
charge density cut-off is \texttt{red}, the one corresponding to the
highest is \texttt{blue}, all the others are \texttt{green}.
Note that the total energy of the last configuration (highest wave function
and charge density cut offs) is subtracted from all energies.

\item
Total energy versus size of the {\bf k}-point mesh. This is a single
figure of the total energy as a function of the size of the {\bf k}-point
mesh. When the test requires several values of \texttt{degauss}, there
is a curve for each \texttt{degauss}. The curve corresponding to the
first \texttt{degauss} is \texttt{red}, the one corresponding to the
last is \texttt{blue}, all the others are \texttt{green}.          
Note that the total energy of the last configuration (highest number of
points and lowest \texttt{degauss}) is subtracted from all energies.

\item
Total energy as a function of volume (\texttt{lmurn=.TRUE.}). 
This plot is composed by two figures. Total energy as a function of volume 
and pressure as a function of volume. Both curves are \texttt{red}.   
The points on the first curve are the calculated energies, the continuous
curve is the fit.

\item
Total energy as a function of one or two crystallographic parameters
(\texttt{lmurn=.FALSE.}). When there is a single parameter the curve is
\texttt{red} as in the case \texttt{lmurn=.TRUE.}. When there are two
parameters a contour plot of the energy as a function of two parameters
is shown. The contour levels, their number and their colors can be
given in input. By default the code shows nine levels with three
colors. From the lowest to the highest levels, the colors are \texttt{red},
\texttt{green}, and \texttt{blue}. The energy value of each level is 
written on output. When the user requests more levels without specifying
their colors, the code continues with three \texttt{yellow} levels, 
then \texttt{pink}, \texttt{cyan}, \texttt{orange}, \texttt{black}, and
when more than $24$ levels are requested the sequence of colors is repeated.
For crystal systems with more crystallographic parameters, this figure is
not available.

\item
Energy bands. In this figure the bands have the color of
their irreducible representation. Each line of the path can have a different
point group and representations. See the \texttt{point\_groups.pdf} file
for the list of representations and their color code. 
When the symmetry analysis is not done all the bands are \texttt{red}.
                              
\item
Electron density of states. This is a plot composed by two figures,
the first contains the electron density of states, the second the integral
of the density of states up to that energy. The dos is \texttt{red}.
In the local spin density case, the dos for spin up is \texttt{red} the
one for spin down is \texttt{blue} and with a negative sign.          
The integrated density of states is \texttt{blue}. In the spin polarized
case, the curve shows the integral of the sum of the up and down density 
of states.

\item
Electronic energy, free energy, entropy, and isochoric heat capacity 
(metals only). This plot is composed by four pictures one for each
quantity. There is a single \texttt{blue} curve per plot.      

\item
Phonon dispersions. In this figure the phonon dispersions have the color 
of their irreducible representations. The same comments made for the plot of 
the band structure apply here.                       

\item
Phonon dos. There is one picture with a single \texttt{red} curve.  
   
\item
Vibrational energy, free energy, entropy, and isochoric heat capacity. This plot
is composed by four figures each one showing one quantity. In \texttt{red} the 
quantities obtained using the phonon density of states, 
in \texttt{blue} those obtained from three dimensional integrals over 
the Brillouin zone. In some cases the \texttt{red} curve is not visible 
because it is exactly below the \texttt{blue} one.

\item
Debye vibrational energy, free energy, 
entropy, and isochoric heat capacity. This plot is composed by four 
figures each one showing one quantity. The curves are in \texttt{blue}
and the word Debye appears in the $y$ axis label.
                                            
\item
Equilibrium volume, bulk modulus, pressure derivative of the bulk
modulus, isochoric heat capacity, isobaric heat capacity,
thermal expansion, isobaric-isochoric difference of the heat capacity,    
isoentropic-isothermal difference of the bulk modulus, and 
average Gr\"uneisen parameter as a function of T (\texttt{lmurn=.TRUE.}). 
This is a plot composed by nine figures, each one containing a different
quantity in the order given above. In \texttt{red} the quantities 
calculated using the phonon dos, in \texttt{blue} those
obtained by a three dimensional integral over the Brillouin zone.
Thermal expansion and average Gr\"uneisen parameter can also be 
calculated using the mode Gr\"uneisen parameters and the result is plotted in
\texttt{green}.

\item
Crystallographic parameters, volume, thermal expansion tensor, and 
volume thermal expansion as a function of temperature 
(\texttt{lmurn=.FALSE.}). The number of
pictures in this plot depends on the crystal system. It shows $a$, 
$b/a$ (for orthorhombic solids) and $c/a$ (for tetragonal, hexagonal, and 
orthorhombic solids) as a function of temperature. Then it shows the
volume, the thermal expansion tensor, and the volume thermal expansion as
a function of temperature. All quantities calculated using the phonon 
density of states are in \texttt{red}, those calculated from integration
over the Brillouin zone are in \texttt{blue} with the exception of the
thermal expansion tensor. This tensor is in \texttt{red} or \texttt{blue}
as above, or it might be \texttt{green} if computed from Gr\"uneisen
parameters. For hexagonal and tetragonal solids $\alpha_{xx}$ is \texttt{red}
if computed from phonon dos, \texttt{blue} if computed from three-dimensional
frequency integration and \texttt{green} if computed from Gr\"uneisen 
parameters. $\alpha_{zz}$ is \texttt{pink}, \texttt{cyan}, and \texttt{orange}
in the same three cases.
In the orthorhombic case $\alpha_{xx}$ and $\alpha_{zz}$ have the same
colors, while $\alpha_{yy}$ is \texttt{gold}, \texttt{olive}, and 
\texttt{light-blue} in the three cases described above.
Only $a$, the volume, and the volume thermal expansion are shown 
for all the other crystal systems.
When the working directory contains a file with the elastic constants and
the flag \texttt{lb0\_t=.FALSE.} the code uses the elastic constants to
compute the bulk modulus and plots the difference between isobaric and
isochoric heat capacities and the difference between isobaric and isothermal
bulk modulus. In each plot there are three curves in \texttt{red} or
\texttt{blue} that indicate quantities calculated
from the phonon density of states or from direct integration over the
Brillouin zone. The \texttt{green} curves use the 
mode Gr\"uneisen parameters to calculate the thermal expansion and
the elastic constants to calculate the bulk modulus. The volume instead
is the same as in the \texttt{blue} curve if \texttt{ltherm\_freq=.TRUE.}
or as in the \texttt{red} curve if \texttt{ltherm\_freq=.FALSE.}.
When both \texttt{ltherm\_freq=.FALSE.} and \texttt{ltherm\_dos=.FALSE.}
the volume is kept fixed at the equilibrium volume at $T=0$ K.
The difference $C_\sigma-C_\epsilon$ of the heat capaticities computed at
constant stress or constant strain is computed also using the anisotropic
formula, without introducing the bulk modulus. In this case it is
plotted in \texttt{yellow} (\texttt{orange}) when the crystallographic 
parameters are calculated from the minimization of the free energy 
obtained from phonon density of states (direct integration over the 
Brillouin zone).

\item
Mode Gr\"uneisen parameters. In this plot the mode Gr\"uneisen parameters have
the color of the irreducible representation of the phonon dispersion curve
of which they are the derivative.
The same comments made for the band structure plot apply here.

\item
Phonon dispersions at the geometry that corresponds to a given temperature. 
The color are assigned on the basis of the irreducible representation of
each mode. The same comments made for the band structure plot apply here.

\item
Temperature dependence of the elastic constants within the 
`Quasi-static Approximation'. There is a plot for each non-zero
elastic constant and the number depends on the Laue class. Elastic constants 
interpolated at the geometry computed using the phonon density of states 
are in \texttt{red}, those calculated from integration over the 
Brillouin zone are in \texttt{blue}.

\end{itemize}

\end{document}
