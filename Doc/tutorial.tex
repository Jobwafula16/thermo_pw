%!
%! Copyright (C) 2016 Andrea Dal Corso 
%! This file is distributed under the terms of the
%! GNU General Public License. See the file `License'
%! in the root directory of the present distribution,
%! or http://www.gnu.org/copyleft/gpl.txt .
%!
\documentclass[12pt,a4paper]{article}
\def\version{0.7.0}
\def\qe{{\sc Quantum ESPRESSO}}
\def\tpw{{\sc THERMO\_PW}}

\usepackage{html}
\usepackage{color}

\usepackage{graphicx}

\definecolor{web-blue}{rgb}{0,0.5,1.0}
\definecolor{coral}{rgb}{1.0,0.5,0.3}
\definecolor{red}{rgb}{1.0,0,0.0}
\definecolor{green}{rgb}{0.,1.0,0.0}

\textwidth = 17cm
\textheight = 24cm
\topmargin =-1 cm
\oddsidemargin = 0 cm

\def\pwx{\texttt{pw.x}}
\def\phx{\texttt{ph.x}}
\def\configure{\texttt{configure}}
\def\PWscf{\texttt{PWscf}}
\def\PHonon{\texttt{PHonon}}
\def\thermo{\texttt{thermo\_pw}}
\def\make{\texttt{make}}

\begin{document} 
\author{Andrea Dal Corso (SISSA - Trieste)}
\date{}

%\def\SissaImage{./sissa_on_white.png}

\title{
%  \includegraphics[width=6cm]{\SissaImage}\\
  \vskip 1cm
  {\color{red} \Huge \tpw\ Tutorial} \\
  \Large (version \version)
}

\maketitle

\tableofcontents

\newpage

\section{\color{coral}Introduction}

This guide gives a brief overview of the possibilities of the \thermo\ package. 
It is a sort of HOWTO that gives you the most important information needed 
to accomplish a given task.
It assumes that you have already downloaded and compiled \qe\ and \thermo\  
and that you have some task to do and want to known where to start. 
The ultimate reference for the use of \thermo\ is its user's guide,
however the capabilities of the code are rapidly expanding and the 
user's guide is becoming more and more complex, so if you do not want
to read it entirely you can find here some indications about where
to find the information that you need.
This tutorial tells you also in which file to find the quantity that you need.

\section{\color{coral}People}
This tutorial has been written by Andrea Dal Corso (SISSA - Trieste). 

\section{\color{coral}Overview}

In order to make a calculation with \thermo\ you need to be able to 
produce an input file for the \texttt{pw.x} code of \qe. This input file
requires mainly five information:
\begin{itemize}
\item The Bravais lattice.

\item The position of the atoms inside the unit cell.

\item The type of atoms and the pseudopotentials files that you want to use.

\item The cut-off energies.

\item The {\bf k}-point mesh that the code should use to make integrations over
the Brillouin zone. The smearing parameter for metals. 
\end{itemize}

The Bravais lattice is specified by an integer number \texttt{ibrav} and by the
crystal parameters \texttt{celldm} (up to six numbers). You can find 
information on the \texttt{ibrav} code and on the required crystal parameters
in the file \texttt{PW/Doc/INPUT\_PW} of the \qe\ distribution. 
In \qe\ you can use \texttt{ibrav=0} and give the primitive
lattice vectors of the Bravais lattice. Presently \thermo\ needs to
known the Bravais lattice number so this form of input is not recommended. 
If you use it, \thermo\ writes on output the \texttt{ibrav}, the 
\texttt{celldm} and the atomic coordinates needed to simulate
the same cell and stops.
You can therefore just cut and paste \texttt{ibrav}, \texttt{celldm} 
and the atomic coordinates in the input of \texttt{pw.x}. Alternatively
you can set the flag \texttt{find\_ibrav=.TRUE.} in the \thermo\ input
and \thermo\ will make the conversion for you and run the job. 
After setting the correct
\texttt{ibrav} and \texttt{celldm}, \texttt{thermo\_pw} might still tell you
that the Bravais lattice is not compatible with the point group. This
can happen, for instance, if you have isolated molecules, amorphous solids,
or defects. In these cases you can still continue but symmetry will not be 
used to reduce the number of components of the tensors that represent 
the physical quantities. In order to use the residual symmetry, you have to
use one of the suggested \texttt{ibrav}, adjusting the \texttt{celldm} to
the parameters of your cell. For instance if you have a cubic cell, but
the symmetry requires a tetragonal lattice, you have to use a tetragonal
lattice with \texttt{celldm(3)=1.0}.
In rare cases, with lattices such as the face-centered orthorhombic some
symmetry operations might be incompatible with the FFT grid found by 
\texttt{pw.x}. The choice made in \qe\ is to discard these symmetries making
the lattice incompatible with the point group. In these cases the code needs 
\texttt{nr1=nr2=nr3}. Set these three parameters in the \texttt{pw.x} input 
equal to the largest one. 

The positions of the atoms inside the unit cell are defined by an integer
number \texttt{nat} (the number of atoms) and by \texttt{nat} 
three-dimensional vectors as specified in the file \texttt{PW/Doc/INPUT\_PW}.
You can use several units, specify the coordinates in Cartesian or on the
crystal basis or you can give the space group number and the
coordinates of the inequivalent atoms. 
These options are supported by \thermo. See the \texttt{pw.x} manual
for details. \\

The number of different types of atoms is given by an integer number 
\texttt{ntyp} and for each atomic type you need to specify a 
pseudopotential file. Pseudopotential files depend on the exchange and 
correlation functional and can be found in many 
different places. There is a pseudopotential page in the \qe\ website, or 
you can consider generating your pseudopotentials with the \texttt{pslibrary} 
inputs. You can consult the web page \texttt{http://www.qe-forge.org/gf/project/pslibrary/} for more information. \\

The kinetic energies cut-offs depend on the pseudopotentials
and on the accuracy of your calculation. You can 
find some hints about the required cut-offs inside the pseudopotentials files,
but you need to check the convergence of your results with the cut-off 
energies. \\

The {\bf k}-point mesh is given by three integer numbers and possible
shifts (0 or 1) in the three directions. The convergence of the results
with this mesh should be tested. \\

Once you have an input for \texttt{pw.x}, in order to run \thermo\ you
have to write a file called \texttt{thermo\_control} that contains a single
namelist called \texttt{INPUT\_THERMO}. This namelist contains a keyword
\texttt{what} that controls the type of calculation performed by 
\thermo. Ideally you should set only \texttt{what} and make a calculation
similar to \texttt{pw.x} calling instead the executable \texttt{thermo\_pw.x}
and giving as input the input prepared for \texttt{pw.x}.

\section{\color{coral}How do I make a self-consistent calculation?}
Use \texttt{what='scf'}. See \texttt{example01}. The calculation is
equivalent to a call to \texttt{pw.x} and is controlled by
its input. In particular in the input of \texttt{pw.x} you can choose
a single self-consistent calculation using \texttt{calculation='scf'}, 
an atomic relaxation using \texttt{calculation='relax'}, or a cell relaxation 
using \texttt{calculation='vc-relax'}.
The use of \texttt{calculation='nscf'} and \texttt{calculation}
\texttt{='bands'} is
not supported by \thermo\ and could give unpredictable results.
There is no advantage to use \thermo\ to do a molecular dynamic
calculation. 

\section{\color{coral}How do I plot the band structure?}
Use \texttt{what='scf\_bands'}. See \texttt{example02}.
With this option \thermo\ calls twice
\texttt{pw.x} making first a self-consistent calculation with the parameters
given in the \texttt{pw.x} input and then a band calculation along a 
path chosen by \thermo, or along a path given by the user
after the \texttt{INPUT\_THERMO} namelist. The path can be given as
in the \texttt{pw.x} input, but see the user's guide for additional details.
A few other parameters control the band plot. The most useful are 
\texttt{emin\_input} and \texttt{emax\_input} that allow you to plot the 
bands in a given energy range. The bands can be found 
in a file called \texttt{output\_band.ps} if you do not specify the name 
of the postscript file in the \thermo\ input.

\section{\color{coral}How do I plot of the electronic density of states?}
Use \texttt{what='scf\_dos'}. See \texttt{example18}. With this option
\thermo\ calls twice \texttt{pw.x} making first a self-consistent calculation
followed by a non self-consistent calculation on a regular mesh of 
{\bf k} points.
This mesh can be specified in the \thermo\ input (if none is given \thermo\ 
uses the default values). The density of
states can be found in a file called \texttt{output\_eldos.ps} if you do not
specify the name of the postscript file in the \thermo\ input.

\section{\color{coral}How can I see the crystal structure?}
\thermo\ is not a crystal structure viewer, but you can use a code such
as \texttt{XCrySDen} that is able to read the \texttt{pw.x} input and
see the crystal structure. If you use \texttt{what='plot\_bz'}, 
\thermo\ produces a \texttt{.xsf} file that contains the structure given in
input and can be read directly by \texttt{XCrySDen}. This could
be useful if you have the space group and the inequivalents atomic positions.
The generated \texttt{.xsf} file contains all the symmetry equivalent 
atomic positions. For the same purpose you could also use the output
of \texttt{pw.x}.

\section{\color{coral}How can I see the Brillouin zone?}
Use \texttt{what='plot\_bz'}. See \texttt{example12}. With this option
\thermo\ does not call \texttt{pw.x}, it just produces a script for
the \texttt{asymptote} code with the instructions to plot the Brillouin
zone and the standard path (or the path that you have given in the \thermo\ 
input).

\section{\color{coral}How can I plot the X-ray powder diffraction spectrum?}
Use \texttt{what='plot\_bz'} to see the spectrum corresponding to
the geometry given in the \texttt{pw.x} input. You can also see the
spectrum corresponding to a relaxed structure using for instance
\texttt{what='scf'}, asking for an atomic (cell) relaxation in the \texttt{pw.x}
input and using \texttt{lxrdp=.TRUE.} variable in the \thermo\ input.
The X-ray powder diffraction spectrum is shown in a file called 
\texttt{output\_xrdp.ps} if the name of the postscript file is not
given in the \thermo\ input. The scattering angles and intensities
are also written in a file called \texttt{output\_xrdp.dat} if the filename
is not given in the \thermo\ input.

\section{\color{coral}How can I find the space group of my crystal?}
Use \texttt{what='plot\_bz'} and look at the output. The space group is
identified. In the case you have a structure with \texttt{ibrav=0} and
the primitive lattice vectors use the option \texttt{find\_ibrav=.TRUE.} 
in the \texttt{thermo\_pw} input (See the user guide in the subsection
{\it Coordinates and structure}).

\section{\color{coral}How do I plot the phonon dispersions?}
Use \texttt{what='scf\_disp'}. See \texttt{example04}. In this case you
have to prepare an input for the \texttt{ph.x} code that must be
called \texttt{ph\_control}. The required information in this input
is the {\bf q}-point mesh used to interpolate the dynamical matrices and
the name of the dynamical matrix files.
See the \texttt{ph.x} guide if you need information on this point.
The phonon dispersions are found in a file called \texttt{output\_disp.ps.g1}
if the name of the postscript file is not given in the \thermo\ input.
The vibrational density of states is found in a file called 
\texttt{output\_dos.ps.g1} if the name of the postscript file in not
given in the \thermo\ input.

\section{\color{coral}How do I calculate the vibrational energy, free energy, entropy, and heat capacity?}
Use \texttt{what='scf\_disp'}. See \texttt{example04}. These quantities
are calculated after the phonon dispersion calculation for the default 
temperature range ($1$ K - $800$ K) if this range
is not given in the \thermo\ input. These quantities 
as a function of temperature are found in the file \texttt{output\_therm.ps.g1}.

\section{\color{coral}How do I calculate the elastic constants?}
Use \texttt{what='scf\_elastic\_constants'}. See \texttt{example13}. The
elastic constants appear in the output of \thermo\ and also in a file
called \texttt{output\_el\_cons.dat} if the name of the file is not
specified in input. This file can be read by \thermo\ for the options
that require the knowledge of the elastic constants.

\section{\color{coral}How do I calculate the Debye temperature?}
Use \texttt{what='scf\_elastic\_constants'}. See \texttt{example13}. The
Debye temperature appears in the output of \thermo. A file called
\texttt{output\_therm.ps\_debye} contains a plot of the vibrational
energy, free energy, entropy and heat capacity computed within the 
Debye model.

\section{\color{coral}How do I calculate the equilibrium structure?}
Use \texttt{what='mur\_lc'}. See \texttt{example05}. Refer to the user's
guide if you have an anisotropic solid. The crystal
parameters are written in the \thermo\ output file. Note that the structure is
searched interpolating with a polynomial or with the Murnaghan
equation the energy calculated for several geometries close to the geometry 
given in the input of \texttt{pw.x} so the closer this structure is to the 
actual equilibrium structure the better the fit is and the
closer the value found by \thermo\ to the real minimum. If the 
equilibrium structure is very different from the one given in the 
\texttt{pw.x} input you might want to repeat the calculation starting 
from the new equilibrium structure. You can also check with the file
\texttt{output\_mur.ps} (when \texttt{lmur=.true.}) or 
\texttt{output\_energy.ps} (when \texttt{lmur=.false.}) if the minimum
found is within the range of calculated structures. If this is not 
the case the minimum has probably a non negligible error.
Note also that almost all options can be specified using
\texttt{what='mur\_lc\_...'} instead of \texttt{what='scf\_...'}.
In this case they are calculated at the equilibrium geometry instead of
the geometry given in the \texttt{pw.x} input. 
This holds also if a finite pressure is specified in the \thermo\ input 
and the equilibrium structure corresponds to the given pressure (See below).

\section{\color{coral}How do I calculate the equilibrium structure 
at a given pressure?}
Use \texttt{what='mur\_lc'} and specify \texttt{pressure=value} in kbar in the
\thermo\ input. Note that in the input of \texttt{pw.x} you should 
give a geometry which is as close as possible to the equilibrium value
found by the code at the given pressure (see above).

\section{\color{coral} How do I specify the interval of temperatures that 
I need?}
See the user's guide in the subsection {\it Temperature and Pressure}.

\section{\color{coral} How do I calculate the crystal parameters as a function
of temperature?}
Use \texttt{what='mur\_lc\_t'}. See \texttt{example09}. Note that
for this option you need to give also the \texttt{ph.x} input.
For anisotropic solids you need to specify \texttt{lmurn=.FALSE.},
otherwise you calculate only the volume as a function of temperature
varying \texttt{celldm(1)} but keeping all the other crystal parameters
constant. The crystal parameters are plotted as a function of temperature
in the standard temperature range if not given in input, in the file 
\texttt{output\_anharm.ps} for cubic solids and
\texttt{output\_anharm.ps\_celldm} for anisotropic
solids if the name of the postscript file is not given in input.

\section{\color{coral} How do I calculate the thermal expansion?}
Use \texttt{what='mur\_lc\_t'}. See \texttt{example09}. The thermal expansions
are shown as a function of temperature in the file \texttt{output\_anharm.ps}.

\section{\color{coral} How do I calculate the bulk modulus as a function of 
temperature?}
Use \texttt{what='mur\_lc\_t'} and the option \texttt{lmurn=.TRUE.}.
This approach is rigorously valid only for cubic solids, for anisotropic
solid it is only an approximation in which only \texttt{celldm(1)} is
changed while the other crystal parameters are kept constant.

\section{\color{coral} How do I calculate the constant pressure heat capacity?}
Use \texttt{what='mur\_lc\_t'} and the option \texttt{lmurn=.TRUE.}.
This approach is rigorously valid only for cubic solids, for anisotropic
solid it is only an approximation in which only \texttt{celldm(1)} is
changed while the other crystal parameters are kept constant.

\section{\color{coral} How do I calculate the electronic heat capacity
of a metal?}
Use \texttt{what='scf\_dos'}. See \texttt{example18}. In the metallic
case in addition to a plot of the density of states this option produces
also a plot of the electronic energy, free energy, entropy, heat capacity
and chemical potential in the standard temperature range if this range is
not given in the \thermo\ input. These quantities are found in the file
\texttt{output\_eltherm.ps} if the name of the postscript
file is not given in the \thermo\ input.
Please keep the default value of \texttt{deltae} or the electronic
thermodynamic properties could be wrong at low temperatures.

\section{\color{coral} Which is the meaning of the colors in
the electronic bands and phonon dispersions plots?}
Different colors correspond to different irreducible representations of 
the small point group of the {\bf k} or {\bf q} wavevector. To see the
correspondence color-representation see the \texttt{point\_groups.pdf} 
manual. The point group used for each {\bf k} or {\bf q} point is
written in the \thermo\ output and also in the plot if you set
\texttt{enhance\_plot=.TRUE.}. In the output you can also find,
close to each band energy or phonon frequency value, the name of the
irreducible representation. Relevant character tables are given in the
\texttt{point\_groups.pdf} manual, in the \thermo\ output, or by the
\texttt{crystal\_point\_group.x} tool.

\end{document}
